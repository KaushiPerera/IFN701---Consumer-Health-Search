\documentclass[]{article}
\usepackage{graphicx}

%opening
\title{Annotated Bibliography}
\author{Kaushi Perera}

\begin{document}
	
\maketitle
	
\section{Inthiran, A., Alhashmi, S. M., \& Ahmed, P. K. (2016). Describing health querying behavior. In Proceedings of the 2nd SIGIR workshop on Medical Information Retrieval (MedIR).} 

This study investigated the querying behaviour of laypeople when searching for health information. The querying process of a whole search process is important because it represents the searchers' initial understanding of the health aspect they search for. The improvement of this understanding can also be understood by evaluating the query reformulation patterns. Therefore, the authors aimed at evaluating qualitative querying and reformulation patterns when performing health tasks with different levels of difficulty. 20 laypeople (undergraduate students, postgraduate students and university staff members) participated in the study. Searching was performed on MedlinePlus; two simulated situations were given to each participant; search session time was unlimited. Information, such as socio-demographic details, general search experience, health search experience were gathered via a pre-experiment interview. Clinical scenarios were used because laypeople tend to perform such health searches. Simulated situation 1 was related to searching reasons for experiencing kidney enlargement followed by urine retention and whether there are any alternative treatments or is surgery the only option available to treat this condition. Simulated situation 2 was related to searching treatment options for a swollen neck on the left side which cannot be moved left or right and the reasons for experiencing this condition. At the end of each task, participants were asked to rate their perception of the task difficulty via a post-experiment interview. Querying patterns were classified as informational directed (related to knowing about a particular topic) and informational undirected (related to knowing everything/ anything about a topic). Query reformulation patterns were analysed using semantic analysis methods.

For simulated situation 1, 5 participants rated it as easy, another 5 rated as neutral and 10 rated as difficult. For simulated situation 2, 15 rated it as easy, another 5 rated as neutral and 5 rated as difficult. When performing easy tasks, participants started search process with informational directed queries (searching answers for specific questions) and then moved to informational undirected queries (searching for broader aspects). When performing neutral tasks, participants started search process with informational undirected queries and moved to informational directed queries (from broader topics to more narrower aspects). When performing difficult tasks, participants started search process with informational directed queries, then moved to informational undirected queries and ended the search session with informational directed queries (no much focus). This behaviour was called as participants performing 'unsystematic movements'. The authors used a model based on a triangle to illustrate querying patterns. According to this model, broadening the query was represented by moving towards the base of triangle and narrowing the query was represented by moving towards the tip of the triangle. Two query reformulation patterns were demonstrated by the participants when performing easy tasks. One pattern was to start the reformulation strategy with switching topic and end it with specialization. The other pattern was to start the reformulation strategy with specialization and end it with parallel movement (previous and the new queries having partial overlaps). When performing neutral tasks, again two query reformulation patterns were demonstrated by the participants. One was to start with parallel movement and end with specialization. The other was using switching topics. When performing difficult tasks also, participants tended to use switching topics. Therefore, participants tended to use varying query reformulation strategies when performing much easier tasks and did not change the reformulation strategies (switching topic only) when performing more difficult tasks. 

In conclusion, based on these results, two recommendations were proposed, so as to aid laypeople to perform well in querying and reformulating (high querying efficacy). One recommendation is to 'develop an algorithm to detect and classify query and reformulation patterns' and the other is to 'provide personalized query recommendations'. As participants performed different querying patterns for tasks with different levels of difficulty, querying patterns are useful to detect task difficulty. Therefore, such an algorithm will be able to detect task difficulty based upon query patterns. In addition, when an algorithm detects a query or a reformulation pattern related to a difficult task, health domains can be used to provide relevant assisting features. Hence, if a switching topic pattern was detected, query suggestions could be provided to laypeople so as to help them in keeping the focus on the search goal. With the use of such query suggestions, laypeople will be able to issue efficient and relevant queries. Health domains can also be used to personalize assisting features according to the health search task performed.      
 
\section{Hu, R., Lu, K., \& Joo, S. (2013). Effects of topic familiarity and search skills on query reformulation behavior. Proceedings of the Association for Information Science and Technology, 50(1), 1-9.}

This study investigated the effect of topic familiarity and search skills on query reformulation when searching for health information. Four research questions were considered in this study; 1. 'Does users' topic familiarity influence their selection of query reformulation types?'; 2. 'Does users' search skills influence their selection of query reformulation types?'; 3. 'Does users' topic familiarity influence the frequencies of query reformulations in a session?'; 4. 'Is there a significant difference in the time spent on each type of query reformulation?'. An experimental IR system of health information was designed with a Google-like search interface. Six search topics were selected from the medical information database MEDLINE to be used in the user studies. 45 graduate students participated in the study. Users were instructed to perform two pre-experiment search tasks. First one was to 'provide a definition of a key biomedical term selected from each search topic'. The second one was to 'find the relationships between biomedical concepts and to provide an answer for each search topic'. The use of search system was not allowed while answering questions. Participants' demographic information, familiarity with the six topics and information about their major was collected. Each participant performed three tasks using health information search system. Data, such as user queries, timestamps of search submissions, document IDs viewed and the search task were collected. 135 search sessions were collected. These sessions were categorized into two groups by considering users' familiarity with the topic. Therefore, the two categories were expert (topic familiarity greater than 2) sessions (34 out of 135 sessions) and novice (topic familiarity less than or equal to 2) sessions (101 out of 135 sessions). 45 participants were divided into two groups based on their major. Therefore, the two groups were Library and Information Science (LIS) major (with more skills in searching information and included 28 out of 45 participants) and other major (17 out of 45 participants). Query reformulations were categorized into six types. 4 main facets were considered for the analysis. 

\begin{enumerate}
	\item Content changed: included 4 sub-facets; (i) specification: adding or using more specific terms to specify a query; (ii) generalization: replacing specific terms with general terms to generalize a query; (iii) parallel movement: shifting query terms to another aspect of the topic.
	\item Content unchanged: included 1 sub-fact; (i) synonym: replacing terms with more common terms which have the same meaning of the previous terms.
	\item  Format: format changes were performed, such as term variations and search operators (abbreviation, preposition, Boolean operators etc.).
	\item  Error: correcting typos and wrong formats in previous terms. 
\end{enumerate}
    

The authors observed that the average topic familiarity before performing searches was approximately 2 (from a nine-level Likert scale). More than 65\% of the participants occasionally search health information online and about 11\% of them never searched health information online. 112 out of 135 sessions had at least one query reformulation action. A total of 334 query reformulation actions were performed. The reformulation types 'Specification', 'Generalization' and 'Parallel movement' were performed in 78\% of the query reformulations. 'Format', 'Error' and 'Synonym' were performed in 16.8\%, 3.9\% and 0.3\% of the query reformulations respectively. Specification was the most frequently used query reformulation type and synonym was the least frequently used query reformulation type. During expert sessions the most frequently applied query reformulation types were 'Generalization' (22.4\%) and 'Format' (22.4\%). Users with higher topic familiarity tended to start the search process with specific terms and extended search results with the use of 'Generalization' reformulation type. During novice sessions the most frequently applied query reformulation types were 'Error' (4.1\%), 'Parallel movement' (25.6\%) and 'Specification' (35.3\%). This revealed that users with low topic knowledge are more prone to make mistakes (misspelling and typo) because of the lack of health topic related knowledge. They also started their search process with general terms and gradually had to move to more specific terms in order to obtain precise and relevant information. However, according to the statistical analysis there was no significant association between topic familiarity (domain knowledge) and the selection of query reformulation types. LIS major group more frequently applied 'Format'(17.7\%), 'Generalization' (21.7\%) and 'Specification' (35.9\%) during their search process. LIS major users tended to change the format of the queries very frequently so as to retrieve more precise search results and made comparatively less errors. The other major group more frequently applied 'Error' (7.4\%) and 'Parallel movement' (27.4\%) during their search process. Hence, this group was unable to issue efficient queries to obtain relevant results and was not good at moving from one aspect to another while searching. However, according to the statistical analysis there was no significant association between search skills and the selection of query reformulation types.  The average query reformulations per session was lower for the 'expert sessions' group (less than 2 query reformulations) compared to the 'novice sessions' group (approximately 2.7 query reformulations). Therefore, users with higher topic knowledge were able to complete search tasks with lesser effort in query reformulation. According to the statistical analysis a significant association between topic familiarity and the frequencies of query reformulations in a session was identified (users with higher topic familiarity applied fewer query reformulations). In terms of time spent on each type of query reformulation, users spent comparatively longer time when correcting queries. However, according to the statistical analysis there was no significant difference in the time spent on each type of query reformulation.   

In conclusion, topic familiarity has the ability to influence the frequencies of query reformulations and the time spent on each type of query reformulation. Users with higher topic familiarity made less spelling errors and preferred applying general terms to modify their queries. Therefore, they tended to issue more correct search statements, and initiated their search processes with more specific terms and gradually moved on to issuing more general terms to obtain broader search results (improve search results with higher recall rate). In addition, these users performed less number of query reformulations to complete their search tasks (more efficient). In contrast, users with lower topic familiarity were less likely to apply 'Format' and 'Generalization' query reformulation types when modifying queries. They initiated their search processes with more broader terms and gradually moved on to issuing more specific terms to obtain more precise search results. Because these users obtained irrelevant search results with their inadequate initial queries, they applied a 'Parallel reformulation' to move from one search aspect to another aspect. Therefore, such users had more difficulty in selecting proper search strategies when initiating a search process. As a result, they performed more number of query reformulations to complete their search tasks (less efficient). Users with higher search skills (LIS major) made less errors and applied 'Format', 'Generalization' or 'Specification' query reformulation types. 'Error' and 'Parallel movement' query reformulation types were not much used by these users. In contrast, users from other major group did not very frequently apply any of the 'Format', 'Generalization' or 'Specification' query reformulation types and made more mistakes when issuing search terms to the system. Therefore, users with higher search skills applied more various query reformulation types. Overall, users spent more time when conducting 'Error' query reformulation type because they had to correct terms from previous queries. Therefore, it is said that query expansion techniques, query suggestion techniques, controlled vocabularies and auto filling techniques can be used in IR systems to aid users with lower topic familiarity.                   
 
\textbf{Limitations}

Authors reported that 135 search sessions from 45 participants were not enough to generalize the findings. Topic familiarity was measured on the basis of user perception, so the information might not be reliable. Information about user search skills were not objectively measured for the study (although LIS major participants were assumed to have advanced search skills compared to other participants).    
 



\end{document} 