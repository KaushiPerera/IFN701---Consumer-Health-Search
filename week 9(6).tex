
\documentclass[]{article}
\usepackage{graphicx}

%opening
\title{Analysis 6}
\author{Kaushi Perera}

\begin{document}

\maketitle


\textbf{Techniques to map user queries to medical symptoms}


In particular, when users want to search for information about medical symptoms, it is really important that they are able to identify proper terms to address these symptoms, in order to retrieve accurate information about these symptoms. Researchers have used different techniques in their studies to determine the impact of those techniques on accurately mapping user queries to medical symptoms. Therefore, according to (A)'s observations, users are able to generate more successful queries to search for the name of a symptom, when they are presented with images and videos of different medical symptoms. This indicated that users are more likely to correctly describe a symptom when they actually experience it. In addition, these researchers also observed that, from health information search systems' side, the use of features, such as Encyclopedias, anatomy (body parts), synonyms, medical dictionaries, Greek and Latin roots and paraphrases is useful when attempting to match a user query with its corresponding symptom name. Therefore, they have concluded that, the use of previously mentioned features has the ability to improve the performance when matching queries with symptom names compared to randomly guessing symptoms names from user queries.



\end{document}