
\documentclass[]{article}
\usepackage{graphicx}

%opening
\title{Analysis 1}
\author{Kaushi Perera}

\begin{document}
	
\maketitle
	
	
\textbf{User search behaviour}


\section{Complexity of health information search tasks} 

The first main aspect of user search behaviour is the complexity of health information search tasks. The term 'task complexity' represents the difficulty, mental effort required and the time needed to successfully complete a health information search task. According to (C) and (H), the session length of a search task and the number of web pages visited increases with the complexity of the health information search task. (C) also state that, task complexity has the ability to influence interaction strategies because both searching and browsing strategies were used by consumers when performing more complex search tasks. As (C) further mention, not only interaction strategies but also search patterns were influenced by the complexity of search tasks. This fact has also been supported by (J) stating that,  consumers start with specific queries, then move to more broader aspects, and again move back to specific queries at the end of search tasks, when performing difficult tasks. In addition to this, (J) also noticed that, the only query reformulation pattern used while performing complex tasks was 'switching topics'. Furthermore, (C) and (G) reported that, consumers use resources, such as Encyclopedia and dictionary to gain a preliminary understanding when performing more complex health search tasks and it is challenging to plan the search for more complex tasks. In contrast, (J) mentioned that, to accomplish much easier tasks, consumers started with more specific queries and moved towards more broader aspects, and to accomplish neutral tasks, consumers started with broader topics and moved towards more narrower aspects. According to (G)'s observations, search planning was also identified as less challenging for easier health search tasks. Summary of how user search patterns varies with the task complexity is reported in Table 1. 

\begin{table}[t!]
	\includegraphics[width=1.0\textwidth]{Taskcomplexity.png}
	\caption{The influence of task complexity on user search patterns\label{tabel1}}
\end{table} 


\section{What do users search for, the purpose of the search and the use of information}


The second aspect of user search behaviour focuses on the content users search for, purposes of the searches and the use of retrieved information. The observations are presented based on three different studies. (A) have studied the purposes of consumer health information searches and have identified that, in addition to seeking information to gain general knowledge about different diseases and symptoms, treatment options, diagnostic information etc. users also tend to search health information to obtain confirming and novel information. However, according to (B)'s observations, some of these information searches were performed by consumers with the intention of confirming their own incorrect initial assumptions. (G) have stated that, online information is mainly used by consumers to improve communications with physicians. 



\section{Health search queries and query reformulations}


The third aspect of user search behaviour is the health search queries and query reformulations, performed by users when searching for health-related information. The impact of users' queries on health searches, the reasons for unsuccessful health searches and different query reformulation patterns performed by users were identified based on five different investigations. According to (D), formulation of good queries and retrieval of well-designed, standard and reliable results lists are the most important factors for a successful health search. However, (B) argue that, most of the consumers still issue imprecise health search queries and this has become one of the main reasons for unsuccessful health information searches. (J) noted that there are mainly two query reformulation patterns, such as moving from switching topic to specialization and moving from specialization to parallel movement which are being used by consumers when performing much easier tasks and also there are two main query reformulation patterns, such as moving from parallel movement to specialization and the use of switching topics which are being used for more difficult tasks. (K) mention that, users spend comparatively a longer time when performing 'Error' (correcting queries) query reformulation type and according to (C), most of the query reformations are performed by conducting conceptual changes. (C) also state that, in addition to performing query reformulations, consumers also tend to perform query re-executions as query iterations. Examples of different query reformulation types and an example of query iterations is presented in Table 2.

\begin{table}[t!]
	\includegraphics[width=1.0\textwidth]{queryreformulations.png}
	\caption{Examples of query reformulations and query iterations\label{tabel2}}
\end{table} 

\section{Patterns of accessing resources}

The fourth aspect of user search behaviour is the different patterns followed by consumers when accessing resources. Basically, three main patterns were identified based on three different studies. According to (H), search engines were the most popular starting point of the searches. As (A) state, in general (irrespective of task complexity), users tend to start their search processes with general concepts and gradually move towards other aspects of those general concepts. In addition, (F) mention that, the majority of users directly access low-level pages (View Study/ View Results) via web-based search engines or consumer health sites, and therefore, it is recommended to place links to background information and other search features on low-level pages.


\section{How consumers evaluate health search results}

The fifth aspect of user search behaviour is, how consumers evaluate health search results. The key factors, such as trustworthiness, quality, features of websites etc. which were used by consumers for the selection and determination of usefulness of search results were identified based on six different studies. According (A), search results are selected based on rankings and familiarity on top of trustworthiness, quality and usefulness of the information. However, as (D) state, credibility, reliability and trustworthiness are also important factors when selecting web pages from results lists. (G) mention that websites/ search results which contained credible information, personalized content or information written by similar people to consumers were assessed as trustworthy. In addition, according to (A)'s observations, both medical-specific and general web sites were used by consumers when performing health information searches. According to (G)'s arguments, consumers also use their prior knowledge when determining the usefulness of search results. Furthermore, according to (A), (F), (G) and (B), some other factors, such as websites which were named with the medical condition searched, websites with visible local maps, design and content of the websites, and explicitness of information in relating lay and professional terms were important when evaluating the usefulness of the search results. Moreover, as (D) and (B) state, the search engine technology used and the organization of search results lists were taken into account when assessing the efficiency of consumer health information systems. According to these observations (C), (D) have concluded that, all these factors should be considered when designing consumer health information websites/ systems, because as (D) and (B) state, the information design of a web page influences the time taken to select from its results pages, users tend to make erroneous decisions by only considering the web page appearances and poor web resource configurations lead to unsuccessful health information searches. Therefore, once a consumer health information system is well designed, it can lead users to successful health information searches. 

\section{The impact of domain knowledge and search skills}


The sixth aspect of user search behaviour is the impact of domain knowledge and search skills on health information searches. The observations of researchers are presented based on eight different investigations. According to (D) and (I), users’ domain knowledge and familiarity with health topics influence consumers' keyword searches and health information search behaviours. This has been confirmed by (B) stating that, despite their web experience, users with incorrect and imprecise domain knowledge tend to search health information on irrelevant websites and are unsuccessful in evaluating retrieved health information. In contrast, (B) also mention that users with higher education levels tend to judge the authoritativeness of search results prior determining them as useful. The impact of domain knowledge and familiarity with health topics has been further supported by (E) stating that consumers with lower eHealth literacy and familiarity with health topics spend a longer time to evaluate the quality of search results and they focus on the main text content of a webpage. On the other hand, (E) also state that consumers with higher eHealth literacy and  familiarity with health topics spend much lesser time to evaluate the quality of search results and their focus is scattered all-over the webpage. In addition to these observations, (C), (I), (K) state that users with higher health topic familiarity and domain knowledge also search for other concepts associated with the main concept, use more specific and more varied vocabulary/query keywords, and start search processes with specific terms and move towards using more general terms. The behaviour of users with lower topic familiarity was further discussed by (K), (H) and (A) stating that they make more mistakes (spelling) when issuing query terms, start search processes with more general terms and move towards more specific terms, quickly get satisfied with the search results, submit short queries and tend to select results from the first results page. Furthermore, according to (B), resource knowledge is important for navigational actions and according to (C), in general users prefer to use natural language while performing health searches. As (B) state, users with better online search skills are able to perform efficient health information searches. This fact has been supported by (K), mentioning that users with more search skills change query format more frequently to obtain relevant results and make less mistakes when issuing queries. Opposed to this, (K) also mention that, users with less search skills are unable to issue effective queries to obtain relevant results and make more mistakes when issuing queries. In terms of query reformulations, (I) mention that, different query reformulation patterns, such as dynamic query reformulation pattern (unfamiliar tasks) and parallel query reformulation pattern (familiar tasks) are being used by the consumers depending on the familiarity level of the tasks. The behaviour of query reformulations has also been discussed by (K) stating that topic familiarity influences the frequencies of query reformulations where users with higher topic familiarity perform a lower number of query reformulations per session and users with lower topic familiarity perform a higher number of query reformulations per session. According to (K), topic familiarity also influences the time spent on each type of query reformulation. Summaries of user behaviours and characteristics with respect to domain knowledge, topic familiarity and search skills are presented in Figure 1, Figure 2 and Figure 3 respectively.   

\begin{figure}[t!]
	\includegraphics[width=1.0\textwidth]{domainknowledge.png}
	\caption{User behaviours and characteristics with respect to domain knowledge\label{fig1}}
\end{figure}  

\begin{figure}[t!]
	\includegraphics[width=1.0\textwidth]{topicfamiliar.png}
	\caption{User behaviours and characteristics with respect to topic familiarity\label{fig2}}
\end{figure} 

\begin{figure}[b!]
	\includegraphics[width=1.0\textwidth]{searchskills.png}
	\caption{User behaviours and characteristics with respect to search skills\label{fig3}}
\end{figure} 


\section{Overall view of consumer health information searches}
 

The seventh aspect of user search behaviour is the overall view of consumer health information searches. The observations of two different studies are presented by highlighting two main factors of consumer health searches which are query formulation and selection of appropriate search results. According to (H), consumers still have problems in formulating queries which is also supported by (D) stating that overall consumer health information search, including query formulation and efficient selection of appropriate search results still remains as a challenging task to many of the consumers.      


  

  

\section{What can be done to support lay people when searching for health information}

The final aspect of user search behaviour is the actions that could be taken in order to support lay people when searching for health information. Three highly useful methods and one moderately useful method were identified based on two different investigations. According to (B), laypeople with imprecise domain knowledge can be supported via information portals, individual websites and education tools. However, (D) state that directory categories were not that useful when performing health information searches because consumers were not familiar with what a category contains and they did not like the fact that some categories having endless sets of category levels. 
   



\end{document}



































 



     
            
      
  
