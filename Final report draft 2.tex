


\documentclass[]{article}
\usepackage{graphicx}

%opening
\title{A literature review on Consumer Health Search}
\author{Kaushi Perera}

\begin{document}
	
\maketitle
	
\textbf{Unit: IFN 701}
\textbf{Credit Points: 24 credit points}	
\textbf{Student ID:n9789511}	
\textbf{Student Name: D. F. Kaushalya S. Perera}	
\textbf{Project Supervisor: Dr. Guido Zuccon}	
\textbf{Project Coordinator: }		
	
\textbf{Abstract}
	
	purpose, 
	scope, 
	research method for the Literature review, 
	key outcomes and 
	recommendations from the literature review completed
	
\textbf{Introduction}    
	
\textbf{Background and/or context}
	
At present online resources are increasingly used by lay people to search different types of information. A significant portion of these searches are performed to retrieve health information. Hersh (2015) state that, the use of Information Retrieval systems by consumers to retrieve health information has become ubiquitous. For example, as Zuccon, Koopman and Palotti (2015) have highlighted in their paper, an analysis conducted using three web search engines, has been able to reveal that, approximately 10\% of user queries issued to these search engines are health related.  Hersh (2015) also mentions that, at present, most scientific papers are published electronically. These profound changes in publishing information is also identified as a main reason for this escalated use of online resources by consumers to seek health information.  
	
Consumers search health information online to obtain various types of information, such as medical information, details about health professionals, self-diagnosis information and to decide the efficiency of medical treatments (Pogacar et al., 2017). However, when lay people perform health information searches, there is a higher possibility of them retrieving inaccurate, irrelevant and unreliable information, because they are more likely to use lay terms to describe a medical condition. For instance, a lay person might issue a query, such as ‘my head is pounding’ except typing the proper medical term (cephalalgia) when searching for medical information. This issue is known as the ‘circumlocutory in medical queries’ (Stanton, Ieong \& Mishra, 2014). In addition, consumers also tend to express health conditions using different words which might make it hard to retrieve information containing words that matches those consumers’ words. As Croft, Metzler and Strohman (2010) mention, this issue has been identified as the ‘vocabulary mismatch problem’. In other words, this basically means that, although the main purpose of information retrieval models is to fetch (retrieve) information which satisfies users’ information needs, the information that is being retrieved highly depend on issued user queries. Consumers are highly likely to have language and knowledge gaps, such as not knowing adequate vocabulary related to health domain and might also be unfamiliar with most of the proper medical terms (Soldaini et al., 2016). 
	
Therefore, consumers’ increased use of online resources to retrieve health-related information can also increase the potential problems associated with the retrieval process, such as retrieving inaccurate, unreliable and useless information. For example, as Pogacar et al., (2017) state, users are highly likely to be influenced by inaccurate and unreliable information, when searching for the efficacy of medical treatments, and might end up having harmful impact on their own lives. Hence, it is crucial for health information search systems to satisfy users’ health information needs by retrieving authoritative, accurate and useful information.  
	
Because of the importance of this field, researchers have focused more on to improving health information retrieval systems as a means of making sure that these retrieval systems are able to satisfy consumers’ information needs. As a result, researches have been conducted to cover various aspects, such as user search behaviours and retrieval models of consumer health information searches (Toms \& Latter, 2007 and Jimmy, Zuccon \& Koopman, 2018).    
	
\textbf{Aims, objectives and the anticipated significance of this literature review project}
	
The aim of this literature analysis is to contribute to the various aspects of consumer health information searches, investigated in recent researches, by synthesizing this information and presenting it as a literature analysis. Therefore, this literature analysis aims to cover various aspects, such as user search behaviours, consumer health information retrieval models, strategies to improve user queries and different problems faced by consumers when searching for health information online. This synthesized knowledge and information then will be reported as an analysis of different aspects covered in each research paper, methodologies which are used for each study and their corresponding results in a well-structured literature review paper. Prior preparing the final literature analysis, this project also prepares an annotated bibliography of all the research papers covered and a table of topics summarising the claims of each research paper. Therefore, this analysis is really useful for health information search system designers because they can find heaps of useful information and tips which can be embedded when designing such systems. In addition, this information is also important for researchers who are interested in conducting researches on ‘consumer health information search’, because this literature analysis highlights any uncovered aspects of consumer health information searches. In other words, this literature analysis highlights areas which are pertinent for future studies.   
	
	
\textbf{a brief overview of the method(s) have applied }
	
The methodology of this literature analysis consists of a few main steps.
	
Step 1: Defining a protocol for the literature analysis. The main purpose of defining a protocol is to retrieve more relevant and useful research papers to conduct this literature analysis 
	
Step 2: Reading and analysing the chosen research papers to gather knowledge and information related to consumer health information searches including:
	
	(i)	User search behaviours
	(ii) Retrieval models
	(iii) Strategies used to improve user queries
	(iv) Problems faced by consumers when searching for health information online
	
Step 3: Preparing an annotated bibliography based on all the covered research papers
	
Step 4: Preparing a table of topics summarising the claims in each research paper
	
Step 5: Preparing the literature analysis by highlighting different topics (aspects), methods and findings covered in each research paper. In addition, any uncovered aspects of consumer health information searches will also be highlighted in this literature analysis
	
More details on the methodology which has been applied to conduct this literature analysis is presenter under the section 'Literature Review Methodology'. 
	
\textbf{Recap the scope of the project}
	
This literature review only focuses on recent research papers which were published 2005 onwards. The knowledge and information in these research papers also need to contribute well to at least one of the main aspects of consumer health information searches as listed below.
	1.	User search behaviours
	2.	Consumer health retrieval models
	3.	Strategies to improve user queries 
	4.	Problems faced by consumers when searching for health information  
	
\textbf{Briefly summarise the key deliverables (results and outcomes}
	
This literature analysis consists of a few main deliverables.
	
1.	An annotated bibliography: This will include an annotation for each of the chosen research paper. Each annotation will highlight key factors, such as the aim of the study, the methodology which has been followed, key findings and the importance of those findings as presented in each research paper.  
	
2.	A table of topics and claims: This table will summarise different topics and claims covered in each of the chosen research paper. For example, an excel spreadsheet will be used to note down different aspects, such as user search behaviours and retrieval models covered in each research paper with their claims/ key findings. Then these findings will be analysed to identify patterns within each aspect.  
	
3.	A presentation highlighting the key findings of the analysis: The final presentation which was prepared and presented in week 12, included information, such as different aspects of consumer health information search covered in each research paper, the key findings of each research paper and the importance of each of the finding.  
	
4.	A literature review paper which includes a thorough analysis of the chosen research papers: The final literature review paper consists of an analysis which includes, different aspects of consumer health information searches covered in each chosen research paper and their corresponding findings. Hence, this literature analysis will be a synthesis of the key findings of each research paper depending on which aspect of consumer health information search it covers. In addition, this literature analysis will also highlight any uncovered aspects of consumer health information search by the chosen research papers.
	
\textbf{Literature Review Methodology}  
	
The methodology of this literature analysis consists of a few main steps.
	
Step 1: Defining a protocol for the literature analysis
	
The main target of this step is to define a protocol (Knopf, 2006) for this literature analysis, which will be used to search and determine the relevance and usefulness of research papers. The relevance and usefulness of research papers will be determined based on a few main factors which have been described below. The researchers paper which are identified as highly relevant and useful will be chosen to conduct this literature analysis. 
	
(1)	The keywords/phrases and other techniques used to search for relevant and useful research papers 
	
a. Consumer health search
b. Consumer health information searching behaviour
c. Retrieval models for consumer health search
d. Strategies for improving user queries 
e. Problems for consumers when searching health information online
	
The use of proper keywords or phrases for searching relevant and useful research papers is important because the retrieval of research papers via search engines is highly impacted by those keywords and phrases. Hence, all the keywords and phrases should be carefully chosen to search for relevant and useful research papers. In addition to executing these queries, the links presented in Google Scholar, such as ‘Related articles’ and ‘Cited by’ will also be used to retrieve relevant and useful research papers.   
	
(2)	Inclusion and exclusion criteria 
	
A few factors are taken into account when determining the usefulness and relevance of a research paper. The major area which will be covered by this literature analysis is ‘Consumer Health Search’. Hence, this is a broader topic, this literature analysis specifically aims at reviewing research papers which cover at least one of the main aspects, such as user search behaviours, retrieval models, strategies for improving user queries or problems for consumers when searching health information online of consumer health information search. Hence, all the research papers which are chosen for this literature analysis must be directly relevant to at least one of the previously mentioned aspects. In addition, all the chosen research papers for this literature analysis must be recent research papers. Therefore, this literature review will only analyse research papers which were published 2005 onwards. Moreover, all the chosen research papers must also be academic and peer-reviewed research papers. 
	
(3)	Search services used
	
Mainly two services which are the Google Scholar and QUT library databases will be used in this literature review project to retrieve relevant and useful research papers. The main reasons for choosing these search services to search for useful and relevant research papers are because these services are able to retrieve information sources which contain highly reliable (academic and peer-reviewed) and complete information on prior research work.   
	
Step 2: Searching for relevant research papers by issuing the queries and analysing those research papers  
	
The main goal of this step is to issue queries as defined in step 1 or use links, such as ‘Related articles’ and ‘Cited by’ on Google Scholar website to retrieve recent research papers that contain information on at least of the aspects, such as user search behaviours, retrieval models, strategies for improving use queries and problems for consumers when searching for health information online. The next step is to read and analyse the chosen research papers to gather knowledge and information they have contributed to different aspects of consumer health information search as previously mentioned. It is crucial to analyse and understand existing research work related to consumer health search, because the foundation of this literature analysis is formed based on prior related research work. In addition, literature analysis also aims at covering approximately five to ten related research papers per week, depending on the length of each research paper.             
	
Step 3: Preparing an annotated bibliography based on all the covered research papers 
	
In order to analyse all the chosen research papers an annotated bibliography (The Writing Center, 2018) will be prepared and used as the analysis technique for this literature review. The tool TeXstudio which is a LaTeX editor will be used to prepare this annotated bibliography. An annotation will be written for each of the research paper covered in this literature analysis and then will be included in the final annotated bibliography. Each annotation will include factors, such as the aim of the study, the methodology which has been used, key findings of the study and the importance of those findings as presented in each research paper. 
	
Step 4: Preparing a table of topics summarising the claims in each research paper
	
At this step, this project aims at preparing an excel spreadsheet as a table of topics using the annotations written for each of the research paper in step 3. Hence, this table of topics will include all the aspects (user search behaviours and retrieval models) covered in each of the chosen research paper, as a categorization of the themes and summaries of their claims. Most importantly these findings then will be analysed to identify any patterns within each aspect. Therefore, this categorization and the analysis of patterns will be the basis of the literature review conducted as the final step of this project.  Hence, this table of topics which summarises the claims of each research paper is really important because the content of final literature review will depend on the knowledge and information gained in this analysis. 
	
Step 5: Preparing the final literature review by synthesizing all the covered topics (aspects), methods, findings in each research paper, and by highlighting any uncovered aspects
	
This is the final and the main step of this literature review project. In this step a literature review will be prepared with the use of knowledge and information gained in the thorough analysis which was conducted in the previous step. Therefore, the final literature analysis will most importantly present any identified patterns within each covered aspect by highlighting the corresponding findings from each research paper. Hence, this literature analysis can also be seen as a synthesis of the key findings of each research paper which are categorized according to their common features. In addition, this literature review will also emphasize any uncovered aspects of consumer health information search by the chosen research papers. The final literature analysis will be prepared and presented by follow the literature review guidelines presented in the paper ‘Writing Integrative Literature Reviews: Guidelines and Examples’ (Torraco, 2005). Moreover, the tool TeXstudio which is a LaTeX editor will also be used to prepare the literature review and the final report.
	
\textbf{Project Management Methodology}  
	
The Dynamic Systems Development Method (DSDM) was the project management methodology which was used to complete this literature review project. The main reason for using this project management approach is because DSDM is an agile project management approach (The DSDM Agile Project Framework, 2014). With the use of this project management approach, the main deliverables of the project, such as the annotated bibliography, the table of topics and the final literature analysis can be implemented as increments. Most importantly, this methodology provides. room to obtain continuous feedback from the supervisor and to identify any potential issues or risks associated with the project before it is too late. Therefore, it is possible to fix any identified issue as soon as they have been identified and then continue with the rest of the project. In addition, in a situation where there is a possibility of timeline slippage, it is possible to adjust the number of research papers read in each week, so as to align project tasks with their deadlines. For example, if time does not permit to read at least five papers in one week, it is possible to read 4 papers in that week and read an extra paper in the following week. Furthermore, another importance of adhering to DSDM as an agile project management approach is its assurance to deliver the project with all the essential requirements in it. Hence, the use of an agile project management methodology, such as DSDM will guarantee a high-level of quality for the final deliverables and will also assist in delivering them on time. 
	
\textbf{Literature Review Results}

\textbf{User search behaviour}


\textbf{1. Complexity of health information search tasks} 

The first main aspect of user search behaviour is the complexity of health information search tasks. The term 'task complexity' represents the difficulty, mental effort required and the time needed to successfully perform and complete a health information search task. According to (C) and (H), the session length of a search task and the total number of search results, for instance web pages, selected by users for viewing increases with the complexity of the health information search task. (C) also state that, task complexity has the ability to influence interaction strategies because both searching and browsing strategies were used by consumers when performing more complex search tasks. As (C) further mention, not only interaction strategies but also search patterns were influenced by the complexity of search tasks. This fact has also been supported by (J) stating that,  consumers start with specific queries, then move to more broader aspects, and again move back to specific queries at the end of search tasks, when performing difficult tasks. In addition to this, (J) also noticed that, the only query reformulation pattern used while performing complex tasks was 'switching topics'. Furthermore, (C) and (G) reported that, consumers use resources, such as Encyclopedia and dictionary to gain a preliminary understanding when performing more complex health search tasks and it is challenging to plan the search for more complex tasks. In contrast, (J) mentioned that, to accomplish much easier tasks, consumers started with more specific queries and moved towards more broader aspects, and to accomplish neutral tasks, consumers started with broader topics and moved towards more narrower aspects. According to (G)'s observations, search planning was also identified as less challenging for easier health search tasks. Summary of how user search patterns varies with the task complexity is reported in Table 1. 

\begin{table}[t!]
	\includegraphics[width=1.0\textwidth]{Taskcomplexity.png}
	\caption{The influence of task complexity on user search patterns\label{tabel1}}
\end{table} 


\textbf{2. What do users search for, the purpose of the search and the use of information}


The second aspect of user search behaviour focuses on the content users search for, purposes of the searches and the use of retrieved information. The observations are presented based on three different studies. (A) have studied the purposes of consumer health information searches and have identified that, in addition to seeking information to gain general knowledge about different diseases and symptoms, treatment options, diagnostic information etc. users also tend to search health information to obtain confirming and novel information. However, according to (B)'s observations, some of these information searches were performed by consumers with the intention of confirming their own incorrect initial assumptions. (G) have stated that, online information is mainly used by consumers to improve communications with physicians. 



\textbf{3. Health search queries and query reformulations}


The third aspect of user search behaviour is the health search queries and query reformulations, performed by users when they search for health information. The impact of user queries on health searches, the reasons for unsuccessful health searches and different query reformulation patterns performed by users were identified based on five different investigations. According to (D), formulation of good queries and retrieval of well-designed, standard and reliable results lists are the most important factors for a successful health search. However, (B) argue that, most of the consumers still issue imprecise health search queries and this has become one of the main reasons for unsuccessful health information searches. (J) noted that there are mainly two query reformulation patterns, such as moving from switching topic to specialization and moving from specialization to parallel movement which are being used by consumers when performing much easier tasks and also there are two main query reformulation patterns, such as moving from parallel movement to specialization and the use of switching topics which are being used for more difficult tasks. (K) mention that, users spend comparatively a longer time when performing 'Error' (correcting queries) query reformulation type and according to (C), most of the query reformations are performed by conducting conceptual changes. (C) also state that, in addition to performing query reformulations, consumers also tend to perform query re-executions as query iterations. An example of a query iteration and a few examples of different query reformulation types are presented in Table 2.  

\begin{table}[t!]
	\includegraphics[width=1.0\textwidth]{queryreformulations.png}
	\caption{Examples of query reformulations and query iterations\label{tabel2}}
\end{table} 

\textbf{4. Patterns of accessing resources}

The fourth aspect of user search behaviour is the different patterns followed by consumers when accessing resources. Basically, three main patterns were identified based on three different studies. According to (H), search engines were the most popular starting point of the searches. As (A) state, in general (irrespective of task complexity), users tend to start their search processes with general concepts and gradually move towards other aspects of those general concepts. In addition, (F) mention that, the majority of users directly access low-level pages (View Study/ View Results) via web-based search engines or consumer health sites, and as a result, they have recommended to make important information, such as background information and other search features easily accessible to users by placing their links on low-level pages. 


\textbf{5. How consumers evaluate health search results}

The fifth aspect of user search behaviour is, how consumers evaluate health search results. The key factors, such as trustworthiness, quality, features of websites etc. which were used by consumers for the selection and determination of usefulness of search results were identified based on six different studies. According (A), search results are selected based on rankings and familiarity on top of trustworthiness, quality and usefulness of the information. However, as (D) state, credibility, reliability and trustworthiness are also important factors when selecting web pages from results lists. (G) mention that websites/ search results which contained credible information, personalized content or information written by similar people to consumers were assessed as trustworthy. In addition, according to (A)'s observations, both medical-specific and general web sites were used by consumers when performing health information searches. According to (G)'s arguments, consumers also use their prior knowledge when determining the usefulness of search results. Furthermore, according to (A), (F), (G) and (B), some other factors, such as websites which were named with the medical condition searched, websites with visible local maps, design and content of the websites, and explicitness of information in relating lay and professional terms were important when evaluating the usefulness of the search results. Moreover, as (D) and (B) state, the search engine technology used and the organization of search results lists were taken into account when assessing the efficiency of consumer health information systems. According to these observations (C), (D) have concluded that, all these factors should be considered when designing consumer health information websites/ systems, because as (D) and (B) state, the information design of a web page influences the time taken to select from its results pages, users tend to make erroneous decisions by only considering the web page appearances and poor web resource configurations lead to unsuccessful health information searches. Therefore, once a consumer health information system is well designed, it can lead users to successful health information searches. 

\textbf{6. The impact of domain knowledge and search skills}


The sixth aspect of user search behaviour is the impact of domain knowledge and search skills on health information searches. The observations of researchers are presented based on eight different investigations. According to (D) and (I), users’ domain knowledge and familiarity with health topics influence consumers' keyword searches and health information search behaviours. This has been confirmed by (B) stating that, despite their web experience, users with incorrect and imprecise domain knowledge tend to search health information on irrelevant websites and are unsuccessful in evaluating retrieved health information. In contrast, (B) also mention that users with higher education levels tend to judge the authoritativeness of search results prior determining them as useful. The impact of domain knowledge and familiarity with health topics has been further supported by (E) stating that consumers with lower eHealth literacy and familiarity with health topics spend a longer time to evaluate the quality of search results and they focus only on the central text content of a resulting webpage. On the other hand, (E) also state that consumers with higher eHealth literacy and  familiarity with health topics spend much lesser time to evaluate the quality of search results and their focus is scattered all-over the webpage. In addition to these observations, (C), (I), (K) state that users with higher health topic familiarity and domain knowledge also search for other concepts associated with the main concept, use more specific and more varied vocabulary/query keywords, and start search processes with specific terms and move towards using more general terms. The behaviour of users with lower topic familiarity was further discussed by (K), (H) and (A) stating that they make more mistakes (spelling) when issuing query terms, start search processes with more general terms and move towards more specific terms, quickly get satisfied with the search results, submit short queries and tend to select results from the first results page. Furthermore, according to (B), resource knowledge is important for navigational actions and according to (C), in general users prefer to use natural language while performing health searches. As (B) state, users with better online search skills are able to perform efficient health information searches. This fact has been supported by (K), mentioning that users with more search skills change query format more frequently to obtain relevant results and make less mistakes when issuing queries. Opposed to this, (K) also mention that, users with less search skills are unable to issue effective queries to obtain relevant results and make more mistakes when issuing queries. In terms of query reformulations, (I) mention that, different query reformulation patterns, such as dynamic query reformulation pattern (unfamiliar tasks) and parallel query reformulation pattern (familiar tasks) are being used by the consumers depending on the familiarity level of the tasks. The behaviour of query reformulations has also been discussed by (K) stating that the rate (frequency) at which query reformulations are performed is influenced by topic familiarity, where users with higher topic familiarity perform a lower number of query reformulations per session and users with lower topic familiarity perform a higher number of query reformulations per session. According to (K), topic familiarity also influences the time spent on each type of query reformulation. Summaries of user behaviours and characteristics with respect to domain knowledge, topic familiarity and search skills are presented in Figure 1, Figure 2 and Figure 3 respectively.   

\begin{figure}[t!]
	\includegraphics[width=1.0\textwidth]{domainknowledge.png}
	\caption{User behaviours and characteristics with respect to domain knowledge\label{fig1}}
\end{figure}  

\begin{figure}[t!]
	\includegraphics[width=1.0\textwidth]{topicfamiliar.png}
	\caption{User behaviours and characteristics with respect to topic familiarity\label{fig2}}
\end{figure} 

\begin{figure}[b!]
	\includegraphics[width=1.0\textwidth]{searchskills.png}
	\caption{User behaviours and characteristics with respect to search skills\label{fig3}}
\end{figure} 


\textbf{7. Overall view of consumer health information searches}


The seventh aspect of user search behaviour is the overall view of consumer health information searches. The observations of two different studies are presented by highlighting two main factors of consumer health searches which are query formulation and selection of appropriate search results. According to (H), consumers still have problems in formulating queries which is also supported by (D) stating that health-related information searches, including query formulations and efficient selections of appropriate search results still remain as challenging to many of the consumers.  

\textbf{8. What can be done to support lay people when searching for health information}

The final aspect of user search behaviour is the actions that could be taken in order to support lay people when searching for health information. Three highly useful methods and one moderately useful method were identified based on two different investigations. According to (B), laypeople with imprecise domain knowledge can be supported via information portals, individual websites and education tools. However, (D) state that directory categories were not that useful when performing health information searches because consumers were not familiar with what a category contains and they did not like the fact that some categories having endless sets of category levels. 

\textbf{Query improvements}


\textbf{1. How query recommendations, clarifications, suggestions and expansions impact health information searches}


The first aspect of query improvements is how query recommendations/ clarifications impact health information searches. The different ways in which query recommendations and clarifications impact health searches have been identified based on four different studies. According to (A), the use of query recommendations, increases the number of successful queries and query suggestions provided by different systems are beneficial for consumers when searching health related information. This fact has been supported by (C) mentioning that, query suggestions lead to more successful searches.  However, (A) also state that, the impact of query recommendations on successful completion of a predefined health search task and the overall user satisfaction is not significant. The importance of query clarifications when performing health information searches was highlighted by (B) stating that, the number of accurately answered medical questions increases with the trustworthiness of the web pages used to answer the questions and with the use of clarified queries, consumers are able to retrieve more trustworthy resources as results. According to (B), implicit query clarifications which are performed on websites without the involvement of users, are also highly useful and most importantly, users do not need to know the correct medical terminologies which are being added to user queries while clarifying them. In addition, (C) argue that, multilingual and multi-terminology suggestions are also useful to retrieve more relevant documents. (D) investigated how query expansion aids laypeople in improving initial queries and the overall retrieval performance. However, they have not been able to make clear conclusions because of the lack of information about relevance and readability assessment of the test collection they have used to evaluate retrieved documents. 

\textbf{2. The impact of different synonym mappings on query clarification}

The second aspect of query improvements is the impact of different synonym mappings on query clarification. The observations are presented based on one study which utilized three main synonym mappings for the analysis. According to (B), Behavioural synonym mappings which mapped user expressions to medical symptoms is the best performing synonym mapping compared to MedSyn mappings (mappings focused on diseases and symptoms, and removed terms which were not related to UMLS semantic types) and DBpedia  mappings (mapping lay terms to expert terminology based on Wikipedia redirect pages), because users are more likely to accurately answer health-related questions with the use of results retrieved using queries clarified with Behavioural mapping and Behavioural mapping is the least preferred mapping for incorrectly answered questions. In addition, (B) also state that, 'simple multinomial logistic regression classifier' (an ensemble of three query clarification methods and no query clarification method) is the best performing query clarification method compared to all individual synonym mappings and any unclarified queries. Furthermore, according to (B)'s observations, the selection of the most appropriate query clarification method (synonym mapping) to clarify each query, further improves the clarification process. 

\textbf{3. Consumers' preference for query recommendations, clarifications and suggestions when performing health information search tasks} 

The final aspect of query improvements is consumers' preference for query recommendations, clarifications and suggestions when performing health information search tasks. Consumers' different preferences are presented based on two different investigations. According to (B), comparatively lay people prefer query clarifications more than experts because lay people are able to retrieve  more useful information to successfully complete a health-related search. (C) identified consumers' higher preference for query suggestions and they have mentioned this as good acceptance of health query suggestions from the tool (Google Chrome extension) was observed. In addition, according to (B)'s observations, a strong correlation has been identified between lay people and experts in terms of the success rate of each health information search task,  because some tasks were more difficult for both groups when compared to other health information search tasks, no matter which query clarification method has been used. However, (B) also have identified that, the difference in success rates between lay people and experts is not significant. 

\textbf{The use of different methods to enhance consumers' understandability of health information and retrieval models}


\textbf{1. The importance of methods which enhance consumers' understandability of health information}

Studies have been conducted in order to investigate the importance of different methods which enhance consumers' understandability of health information. The observations are presented based on both system side approaches and user side approaches which have been used to enhance consumers' understandability of health information. According to (A), the use of supportive information to thoroughly explain a health concept (contextualization of health information), significantly increases lay users' understanding of health information by raising users' health literacy and reducing communication complexity. In addition, (A) mention that, users' cognitive style, including the way they organize, filter, transform and process information is also a significant factor in thoroughly understanding health content. Apart of that, as (A) mention, consumers also use their own contexts, such as social context and psychological context as a support for understanding health information well.        


\textbf{2. The impact of retrieval models on consumer health information search} 


Researches have also been conducted to study the impact of retrieval models on consumer health information search. The observations are presented based on a study which was conducted to investigate the performance of two different knowledge bases. According to B's observations, UMLS Knowledge Base has the ability to effectively expand more queries and retrieve more reliable and useful documents compared to Wikipedia Knowledge Base. In addition, (B) also state that, relevance feedback improves the effectiveness of these query expansions.    
	
\textbf{Current web search engines and problems when searching health information online}

\textbf{1. Problems faced by consumers when searching for self-diagnosis information, symptoms and efficacy of health treatments}


Researchers have conducted studies in order to understand different problems faced by lay people when they search for health/ medical information, as a means of improving health information search systems. The problems faced by consumers when searching for self-diagnosis information, symptoms and treatments are presented in this section based on four studies. According to (A)'s observations, consumers face different problems when using self-diagnosis websites' information to diagnose symptoms. They state these problems as experiencing anxiety after reading inaccurate or false diagnosis information about serious conditions, unreliability of available information because there is no involvement of a health professional, the possibility of making users buy prescribed drugs from Internet pharmacies, and the potential to ignore professional assistance and diagnosis by believing false reassurance provided by self-diagnosis websites. This has also been confirmed by (C) stating that, there is a higher possibility of users retrieving misleading advice and irrelevant information when searching for self-diagnosis information which will then lead to inaccurate self-diagnosis decisions and eventually cause harm to people. One of the main reasons for users obtaining irrelevant health information has also been highlighted by (C) mentioning that, current retrieval techniques are not able to retrieve relevant health information when queries which describe symptoms in a circumlocutory manner are issued. Therefore, (A) have concluded that information available on self-diagnosis websites should be well structured, reliable, evidence-based and presented in lay language. In addition to problems faced by consumers when searching for self-diagnosis information and symptoms, according to (D), consumers also face similar problems when searching and determining the efficacy of health treatments. They have mainly observed that users' decisions are highly dependent upon search results' bias. Therefore, when search results are bias towards correct information, users were able to accurately determine the efficacy of health treatments. However, when search results were bias towards incorrect information users’ made more inaccurate and harmful decisions which were even worse than the decisions made without any search results. Other than that, users who were less confident with the efficacy of the treatments labelled them as inconclusive. (D) also claim that the impact of search results bias on the portion of accurate decisions and harmful decisions is statistically significant. A few of the main reasons for users making inaccurate decisions regarding the efficacy of health treatments were identified. They are, users less likely behaviour to label a health treatment as unhelpful because they are willing to see positive information in search results, users' prior health knowledge which make them think that most of the medical treatments are useful, users' bias towards clicking on the top ranked (rank 1) search result and the lower number of search results interacted by users. Therefore, they have concluded that, search results significantly affect people’s decisions of the effectiveness of medical treatments and have a great potential to both help and harm users. Furthermore, as (B) have observed, providing supportive information (contextualization) has the ability to reduce communication complexity and increase users' understanding of different health concepts. However, they have also identified that there is a gap between this additional/ supportive information, which is known as 'informational context', provided to users by health information websites and users' own personal contexts, such as social context and psychological context which are used to support the better understanding of health-related information.  

\textbf{The use of different methods to evaluate the relevance of health-related search results}

Retrieved results for a health search task are evaluated in order to determine whether those results are relevant to what a user has searched for. Researchers have used different methods in their studies to decide the relevance of retrieved search results. (A) recruited human assessors to evaluate the relevance of retrieved search results. According these assessors, documents which contained information focused on the symptoms searched by users, such as descriptions of the symptoms and causes of those symptoms which are also supported by photographic material were considered as ‘Highly relevant’. Therefore, they have observed that, when seeking for highly relevant information, on average, only three out of top ten results contain highly useful information to self-diagnose medical symptoms. In addition, approximately four or five results of the top ten results, which contained little information that was helpful to people, were considered as relevant (provide useful information to people in order to self-diagnose themselves). Furthermore, documents which contained information not only about relevant symptoms but also about other symptoms were considered as ‘somewhat relevant’. Moreover, ‘on topic but unreliable’ documents were considered as irrelevant in this evaluation. (B) have used a Decision Tree approach in their study to categorize retrieved health search results as relevant or irrelevant based on users' queries. According to their observations, the method with feature selection (selected the most important features which were strongly related to relevance) performs well and has a higher accuracy when determining the relevance of the search results compared to the method without feature selection. Therefore, they concluded that, this feature selection model can be used to filter out irrelevant documents prior presenting search results to consumers, so as to aid them in retrieving more relevant health related documents.     

\textbf{Techniques to map user queries to medical symptoms}


In particular, when users want to search for information about medical symptoms, it is really important that they are able to identify proper terms to address these symptoms, in order to retrieve accurate information about these symptoms. Researchers have used different techniques in their studies to determine the impact of those techniques on accurately mapping user queries to medical symptoms. Therefore, according to (A)'s observations, users are able to generate more successful queries to search for the name of a symptom, when they are presented with images and videos of different medical symptoms. This indicated that users are more likely to correctly describe a symptom when they actually experience it. In addition, these researchers also observed that, from health information search systems' side, the use of features, such as Encyclopedias, anatomy (body parts), synonyms, medical dictionaries, Greek and Latin roots and paraphrases is useful when attempting to match a user query with its corresponding symptom name. Therefore, they have concluded that, the use of previously mentioned features has the ability to improve the performance when matching queries with symptom names compared to randomly guessing symptoms names from user queries.   
	
	
\textbf{Discussion}

This section critically analyses the significance of this literature review, key findings, limitations associated with this literature reivew, and key recommendations and steps that could be taken to further develop this work. The anticipated significance of this literature analysis was to contribute to the existing various aspects of consumer health search, investigated in recent studies, by synthesizing this existing information and presenting it as a literature analysis. The different consumer health search aspects which were considered in this literature analysis, include user search behaviours, strategies used to improve user queries, methods used to enhance consumer understandability of health information, retrieval models, problems faced by consumers when searching health information online, methods used to evaluate the relevance of health search results and techniques used to map user queries to medical symptoms. In addition to different aspects of consumer health search, this analysis has also highlighted several issues which are uncovered by the considered set of studies.   


\textbf{Key findings:}

The key findings of this literature analysis is presented with respect to the seven main aspects of consumer health search which were identified by analysing ........ recent research papers. 

The first key aspect of consumer health search identified was user search behaviours. According to the findings, the complexity or difficulty of a consumer health search influences the overall health search process, including session length, number of web pages visited, interaction strategies, search patterns, query reformulation patterns and search planning. When considering the purposes of consumer health searches, it was identified that, in addition to searching information about diseases, symptoms, treatments etc., users also search for confirming and novel information, and some of this information is also used by consumers to improve communications with physicians. In addition, formulation of effective queries and the use of most appropriate query reformulation types lead consumers to successful health information searches. In relation to the starting point of a search, it was identified that, at the beginning of a search process consumers search for more general concepts via search engines. Moreover, consumers tend to evaluate the usefulness of search results based on factors, such as trustworthiness, quality, features of websites, reliability, ranking, familiarity, personalized content and prior knowledge. The organization of the search results lists is important, because consumers use it to assess the efficiency of health information systems. Furthermore, topic familiarity, domain knowledge and search skills have a higher impact on health information searches performed by consumers. Therefore, consumers who have higher levels of domain knowledge, topic familiarity and more search skills are able to perform successful health searches. Hence, health information search still remains as a challenging task to most of the consumers because consumers have problems with query formulations and selection of useful search results. As a solution, information portals, individual websites and education tools are suggested to be used as a support for consumers when searching for health information. 

The second key aspect of consumer health search identified was strategies used to improve user queries. According to the findings of this aspect, query recommendations, clarifications and suggestions, all are able to increase the number of successful queries and therefore, positively impact consumer health searches. The use of different synonym mappings, such as Behavioural, MedSyn and DMpedia has the ability to improve query clarification processes with varying levels of performance. In addition, query clarification process can be further improved by using a logistic regression classifier which ensemble all three synonym mapping methods and a method with no query clarification. Furthermore, lay people prefer to have query recommendations, clarifications and suggestions while searching for health information, because they are able to retrieve more useful information with the assistance of these techniques. 

The third key aspect of consumer health search identified was methods used to enhance consumers' understandability of health information. The findings of this aspect relieved that, both user side approaches, such as users’ cognitive style and the use of their own contexts (social and psychological), and system side approaches, such as contextualisation of health information (providing supportive information to thoroughly explain a health concept) which are used to enhance consumers’ understandability of health information have a positive influence on overall consumer health information searches, because they assist to increase consumers’ understandability of health information.

The fourth key aspect of consumer health search identified was retrieval models. According to the findings of retrieval models, the use of most suitable retrieval model is able to effectively expand more queries and retrieve highly useful information. In addition, relevance feedback can be used to further improve the effectiveness of these query expansions.

The fifth key aspect of consumer health search identified was problems faced by consumers when they search for health-related information online. The findings of this aspect revealed that, consumers face various problems when searching for different types of health-related information. When they search for self-diagnosis information they can face many problems, such as experiencing anxiety after reading inaccurate or false diagnosis information, unreliability of available information, making users buy drugs from Internet pharmacies, ignore professional assistance by believing everything on self-diagnosis websites, and making erroneous and harmful self-diagnosis decisions. In addition, they also face problems when determining the efficacy of health treatments, because their decisions are highly dependent on search results' bias. In other words, if search results are bias towards inaccurate information consumers tend to make erroneous and harmful decisions about the efficacy of health treatments. Furthermore, there is a gap between the supportive information provided to users by health information websites and users' own personal contexts (social and psychological) which they use to support their understanding of health information.  

The sixth key aspect of consumer health search identified was methods used to assess the relevance of health search results. The findings of the methods used to assess the relevance of health search results revealed that, search results which contain information directly relevant to what user has searched are considered as highly relevant by human assessors. In addition, when using an approach, such as Decision Trees to determine the relevance of health search results it is important to use a method with feature selection, because it improves the accuracy of decisions made on the relevance of search results.

The final key aspect of consumer health search identified was techniques used to map user queries to medical symptoms. According to the findings of techniques used to map user queries to medical symptoms, it is understood that, users are able to correctly describe and search for a symptom, when they actually experience it. In addition, the use of features, such as Encyclopedias, anatomy (body parts), synonyms, medical dictionaries etc. is highly useful because these features improve the performance of health information search systems, when matching user queries with symptom names.      

\textbf{Uncovered issues/ aspects:}  

There are a few uncovered issues which were noticed while analysing recent research papers. One of the main issues is that, studies have not been conducted by performing the same consumer health information search task in several different consumer health information websites. This is important in order to understand how different architectures, structures and interfaces of websites influence, consumer health information searches. In another study, which aimed at investigating how query expansions impact on improving initial user queries and the overall retrieval performance, authors were not able to provide clear conclusions, because of the unavailability of some other important information. For instance, relevance assessments for the test collection used for the study.   


\textbf{Limitations of this lit analysis:}

only a few papers covered for some aspects


\textbf{Key recommendations arising from my work: ornext steps that must be taken to move the project forward/ develop the work further (propose a research agenda base on this literature analysis): }

Based on this literature analysis, it can be concluded that, it is really important for consumer health information system designers to consider above mentioned findings as much as possible when designing these systems. It will be really helpful for consumers with low levels of domain knowledge to retrieve more useful health information as required by formulating effective queries and selecting the most useful health search results. 

\textbf{Appendix} 

\section{Zuccon, G., Koopman, B., \& Palotti, J. (2015, March). Diagnose this if you can. In European Conference on Information Retrieval (pp. 562-567). Springer.}

This study investigated the effectiveness of current web search engines in retrieving useful and relevant information for diagnostic medical circumlocutory queries. ‘Circumlocutory queries’ are queries in which the user observed symptoms are described in a long winded form, rather than using the specific medical terms for the symptoms. In this study the authors mainly investigated eight symptoms. In order to investigate these symptoms, they considered three or four queries per each symptom. As a result, in total twenty six queries were investigated. The  medical circumlocution diagnostic queries were generated based on a method proposed in a prior study. These diagnostic queries represented query terms that may be issued by users when searching for self-diagnosis information.  Along with the queries, the names of the symptoms that users referred were recorded in a table. This information was used for relevance assessment. This study used Google and Bing search engines to issue all the twenty six queries and to obtain the top ten search results for each of these queries. URLs of all these top-ranked results were also recorded. The authors recruited researchers from Queensland University of Technology and also another eight higher degree students to evaluate the relevance of each of these retrieved result. Most importantly, none of these participants were medical experts. As a result, the authors were able to observe the actual behaviours of lay people (no medical knowledge) when they search for health information online. Each participant was shown web pages only related to a single symptom. 

Then participants determined the relevance of each web page by considering whether a user is able to properly self-diagnose (find the accurate medical term of the symptom) a medical symptom by using the information on that web page. Participants were given four labels, which are Not relevant, On topic but unreliable, Somewhat relevant and Highly relevant, and were instructed to assign one label to each of the search result. The effectiveness of Google and Bing search engines was assessed using precision (relevance of documents) and nDCG (ranks of relevant documents). The authors observed that on average only about four or five results of the top ten results, which were retrieved using Google and Bing, consisted of useful information for people in order to self-diagnose medical symptoms. In particular, if highly relevant information was sought, on average, only three out of top ten results contained highly useful information to self-diagnose medical symptoms. When analysing the documents/results which were judged as ‘Somewhat relevant’, it revealed that these documents contained information not only about the relevant symptom but also it contained information about other symptoms as well. For instance, some documents which were categorized as ‘Somewhat relevant’ had definitions of several symptoms including the definition of the targeted symptom. When analysing ‘highly relevant’ document category, it revealed that, those documents contained information focused on the symptoms searched by users, such as descriptions of the symptoms and causes of those symptoms which were also supported by photographic material. The category of documents 'on topic but unreliable' was considered as irrelevant in this evaluation.  

In conclusion, according to the results of this study, it was revealed that currently used retrieval techniques in search engines are not able to retrieve useful results for queries issued with circumlocutory or colloquial descriptions of symptoms. Hence, there are possible risks associated with lay people when they search for self-diagnosis information online, because they are more likely to retrieve misleading information that would lead them to inaccurate self-diagnosis decisions and eventually cause harm to them.   

\textbf{Limitations}: Only a small amount of queries were considered in this study. 
During the evaluation performed by each participant, only one query was taken into account, but in reality it is important to consider the whole search session, including all the health queries issued by users. Other factors, such as reliability and understandability of the obtained information which are necessary to evaluate the relevance of search results were not considered in this study.

\section{Stanton, I., Ieong, S., \& Mishra, N. (2014, July). Circumlocution in diagnostic medical queries. In Proceedings of the 37th international ACM SIGIR conference on Research \& development in information retrieval (pp. 133-142). ACM.}

The objective of this study was to find the corresponding professional medical terms for medical signs and symptoms that are referred by user generated 'colloquial health search queries'. For instance, the colloquial name for ‘cephalalgia’ is headache and a circumlocutory for ‘cephalalgia’ is 'my head is pounding'. The solution suggested in this paper ignored query terms that were not helpful to determine the proper medical symptom and picked up the terms which were more likely to matter. Therefore, this study aimed at solving three main problems; 1. how to generate training data for this experiment (queries which colloquially explain medical signs or symptoms), 2. how to automatically represent the similarity between a user query and a medical symptom and 3. how to automatically infer the actual symptom name from a circumlocutory expression. 

The authors used a reversed approach of crowdsourcing in conjunction with images and videos to obtain colloquial variants for a medical symptom (training data). Then the main aim was to map these colloquial variants and documents to a particular symptom name in order to improve search engines' ability to retrieve more useful content. Colloquial variants for a medical symptom were collected by presenting fixed symptoms to participants and asking them to generate queries that they would run to search and then determine the health problem. Because users tend to describe a symptom according to what they see and feel, a set-up was used in this study to provide participants with simulated experiences of having a given symptom. Several knowledge sources, such as Greek and Latin roots, medical dictionaries, Encyclopedias, synonyms, paraphrases, anatomy (body parts) and colours were used to understand a user's colloquial query. Each query and its associated symptom were converted to two vectors based on the feature selected. A vector for a particular query was built by considering the query terms. The cosine similarity was measured to compare the similarity between a user query and its associated symptom. The labelled data was obtained by crowdsourced labels and Wikipedia redirects (approximately twelve redirects for each medical symptom).  Two approaches were followed during the experiments.In approach 1 training and testing both were performed on the same fixed set of symptoms where images representing 31 symptoms and videos representing 10 symptoms were used by the classifier to choose the most appropriate class for a query. In approach 2 training (crowdsourced data) and testing (Wikipedia redirects) were performed on two different sets of symptoms.     

The authors observed that, during the first approach the improvement to the micro-average (total fraction of correct symptoms) was 61\% when images were presented to participants for generating queries (33\% improvement over the baseline). The improvement to the micro-average (total fraction of correct symptoms) was 85\% when videos were presented to participants for generating queries (26\% improvement over the baseline). During the second approach overall accuracy was obtained as 59\%. Therefore, overall this approach exhibited improvements over random guessing. The features, such as Encyclopedias, anatomy (body parts), synonyms, medical dictionaries, Greek and Latin roots and paraphrases were identified as top ranked features because they were able to match queries with their corresponding symptom name. 

\textbf{Limitations:} 

The authors observed several limitations of this study. Firstly, the classifier was unable to relate some of the truly positive query, symptom combinations. Secondly, this classifier was unable to better perform on some redirects from professional to highly professional language and on foreign language redirects. Thirdly, crowdsourcing participants of this study did not properly represent a sample of Internet users. Finally, it is kind of impossible for someone to correctly describe how a symptom really feels when that person is actually not experiencing the symptom. 

\section{Pogacar, F. A., Ghenai, A., Smucker, M. D., \& Clarke, C. L. (2017, October). The Positive and Negative Influence of Search Results on People's Decisions about the Efficacy of Medical Treatments. In Proceedings of the ACM SIGIR International Conference on Theory of Information Retrieval (pp. 209-216). ACM.}

This paper investigated the impact of search results on consumers' decisions of the effectiveness of medical treatments. 60 people participated in the study. They were presented with results which were biased towards either being correct or incorrect. Search results pages which were biased towards correct information had 8 out of 10 results correct. Search results pages which were biased towards incorrect information had 8 out of 10 results incorrect. Each search result/ document was presented with its title, url, and a snippet. In addition, the top most rank of a correct result was also controlled (either 1 or 3) to analyse the effect of ranks. The levels 1 and 3 were chosen because, according to prior research the mostly viewed search results were the first two results. Participants were asked to decide the effectiveness (efficacy) of ten medical treatments either with the help of search results or in a controlled condition with no search results. The two independent variables considered were the search results bias (correct and incorrect) and the topmost accurate search result's rank (level 1 or level 3). The two dependent variables considered were the portion of accurate decisions and the portion of inaccurate (harmful) decisions made by the participants. Data was also collected via questionnaires and as feedback on each decision made. Three categories, such as Helps (effective treatment), Inconclusive (unsure about the medical treatment) and Does not help (ineffective treatment) were used to categorise the efficacy of each treatment. Participants who performed under the control condition had to decide the effectiveness of two out of ten treatments. 

Two models, such as a complete model and a null model (no variable of interest) were used to analyse the significance of the dependent variables. The authors observed that when the rank 1 document was correct and the whole results page was biased in favour of accurate information the accuracy of decisions increased approximately up to 70\% and lowered the percentage (from 20\% to 6\%) of harmful decisions. This accuracy was significantly higher than the accuracy obtained while determining the efficacy of treatments under the control condition. In contrast, when results were biased in favour of incorrect information it notably reduced the accuracy of decisions (from 43\% to 23\%) and doubled the rate of harmful decisions. This performance with the presence of incorrect results was even worse than determining the effectiveness of treatments in the control condition (no search results). The impact of search results bias on the portion of accurate decisions and inaccurate (harmful) decisions was identified as statistically significant. The impact of the topmost correct rank on the portion of inaccurate (harmful) decisions was nearly statistically significant. While performing the tasks under the control condition, participants tended to categorize truly unhelpful treatments as inconclusive treatments. This clarified that participants were looking for positive information and did not want to classify a treatment as unhelpful. Hence, it was understood that search results with incorrect information can influence users' decision making. A positive correlation was identified between participants' prior knowledge of different health problems and medical treatments. It was also observed that more prior knowledge of the health issues led users to make more harmful decisions because users with prior knowledge were less likely to classify a medical treatment as inconclusive. Participants who classified medical treatments as inconclusive were identified as less confident of their final decision compared to classifying a treatment as unhelpful or helpful. Rank 1 search result of a results page was identified as important because some users clicked on it multiple times during the same session. It was also observed that, the more participants interact with the search results the more accurate decisions they make.

In conclusion, searchers performed better when they were exposed to correct information. On the other hand, they performed worse than not having search results, when they were exposed to incorrect information. Hence, it was concluded that search results notably affect consumers' decisions of the effectiveness of medical treatments. Therefore, the results of this study demonstrated that search engines are able to both assist and harm users. 

\section{Soldaini, L., Yates, A., Yom-Tov, E., Frieder, O., \& Goharian, N. (2016). Enhancing web search in the medical domain via query clarification. Information Retrieval Journal, 19(1-2), 149-173.}

This study investigated the usefulness of ‘query clarification’ when bridging the gap between lay terms and expert vocabularies. Query clarification was performed by adding the most suitable expert expressions to search queries issued by lay people. Three different types of synonym mappings, such as Behavioural mappings (mapping user expressions to medical symptoms), MedSyn mappings (mappings focused on diseases and symptoms, and removed terms which were not related UMLS semantic types) and DBpedia  mappings (mapping lay terms to expert terminology based on Wikipedia redirect pages) were used to automatically improve user queries. Each synonym mapping mapped lay terms to expert expressions. For each of the user query three clarified queries were generated using the three different types of synonym mappings. Bing was used to retrieve search results for each of the four versions of the query. The most appropriate clarification (the clarification which best represented the medical concept expressed in user query) for a user query was selected by identifying the query clarification which was most likely (maximum probability) to appear in health-related Wikipedia pages. The Wikipedia pages with information boxes containing medically-related identification codes, such as MedlinePlus, DiseasesDB, eMedicine, MeSH and OMIM were considered as health related Wikipedia pages. In a situation where it was possible to map multiple query expressions to expert terms, the longest one was selected assuming that it entirely captured the information requirements of the user. In a situation where it was possible to map multiple query expressions with the same length, the one with the 'highest conditional probability' was selected.  

The similarities and differences of each of the three different synonym mappings were analysed based on three factors, such as the size of each synonym mapping, common terms between each set of synonyms and overlap between the results obatined by issuing each query prior and after query clarification.  The authors observed that Behavioural was the smallest in size and DBpedia was the largest in size. The synonym list of Behavioural mapping was almost completely contained within the DBpedia which had the largest synonym list.  Both Behavioural and DBpedia synonym mappings performed similar query clarifications. Queries clarified with MedSyn synonym mapping and unclarified queries retrieved quite similar results. Therefore, Behavioural and DBpedia mapping were identified as the most similar synonym mappings. Eighty lay people and  twelve medical experts participated in the study.  Fifty clarified queries were generated for this analysis. Each of these queries consisted of a symptom name, a drug name or a disease name. Each participant was instructed to answer twenty medical related multiple choice questions which were prepared based on each clarified query. Participants were also instructed to answer questions based on search results which were obtained by issuing clarified  and unclarified queries.

The authors observed that Behavioural synonym mapping was identified as the best performing synonym mapping because users were more likely to correctly answer questions with the use of results which were retrieved using queries clarified with Behavioural mapping. In addition, Behavioural mapping was the least preferred for incorrectly answered questions. Lay people preferred query clarifications more than experts because lay people were able to retrieve  more useful information to correctly answer health-related questions. In contrast, experts preferred to have their own original queries when retrieving information. However, the difference in success rates between the lay people and experts was identified as not significant. A strong correlation was also been identified by authors, between lay people and experts in terms of the success rate of each health information search task. The reason for this was because some questions were more difficult for both laypeople and experts when compared to other questions. The number of accurately answered medical questions increased with the trustworthiness of the web pages used to answer the question. In addition, clarified queries were able to retrieve more trustworthy resources as results. A logistic regression classifier was used to identify the optimal synonym mapping to perform query clarification. Furthermore, the authors also used a 'simple multinomial logistic regression classifier' which was an ensemble of the methods of different query clarification types (synonym mappings) and the method with no query clarification. This was considered as an additional baseline and according to the observations of the authors, this logistic regression classifier performed the best when compared to all individual synonym mappings and unclarified queries. Hence, it was concluded that by selecting the most appropriate query clarification method (synonym mapping) to clarify each query, the clarification process can be further improved. 

In conclusion, users are able to retrieve more reliable results by clarifying their initial queries. These reliable results can then be used to correctly answer health related questions. Implicit query clarifications are also highly useful and most importantly, users do not need to know the correct medical terminologies which are being added to user queries while clarifying them.        
    	
\section{Zeng, Q. T., Crowell, J., Plovnick, R. M., Kim, E., Ngo, L., \& Dibble, E. (2006). Current Challenge in Consumer Health Informatics retrieval with query recommendations. Journal of the American Medical Informatics Association, 13(1), 80-90.}


This study investigated how the use of a novel system known as 'the  Health Information Query Assistant (HIQuA)' impact on consumers' health information retrieval by providing supplementary query terms, which are related to consumers' initial queries, to construct more efficient queries. Three sources, such as consumer usage patterns, medical vocabularies and concept co-occurrences in medical publications were used to identify related terms which were then suggested to consumers. In order to provide query recommendations, the system mapped a user query to a concept defined by the 'Unified Medical Language System (UMLS)' and then identified other concepts which are related to the initial concept. These related concepts were identified based on three sources, such as 'medical vocabulary' source, 'literature co-occurrence' source and 'concept co-occurrences in consumer HIR sessions'. The degree of relevance was calculated by considering the frequency of concept co-occurrences in publications or frequency of relation occurrences in different medical vocabularies. The 'semantic distance' between two health-related concepts was calculated using their co-occurrence frequency. The concepts with shortest semantic distances to the initial query concept were considered as related concepts. In addition to suggesting concepts based on consumers' initial query concepts, the system also suggested concepts based on the 'type' of initial query concepts, such as disease and procedure.      

Two hundred and thirteen (213) people participated in the study. They were instructed to perform one of the two pre-defined health search tasks. The first task was related to finding five factors which increase the possibility of having a heart attack and the second task was related to finding three treatments for baldness. 50\% were randomly assigned with the first search task and the other 50\% were assigned to the second search task. Then 50\% were provided with the query recommendation function and the rest were not provided with it. Each participant was instructed to perform a health search task which was defined by themselves and search for a health related question according to their task definition. At the end of searches, data regarding participants' satisfaction of the search tasks were collected based on a scale of 1 to 5. All the queries issued by participants and their recommendation selections were recorded. The impact of query recommendation function was evaluated by comparing three outcomes between the groups with and without the query recommendation function. It was identified that, health associated web experience and health status are statistically significant, therefore, they were used for further analysis between the two groups.

The authors observed that 85.2\% with query recommendations were satisfied with the search tasks and 80.6\% without query recommendations were satisfied with the search tasks. The satisfaction rate increased by 79\% for the participants with recommendations. However, it was identified that the association between the groups with and without the query recommendation function and user satisfaction are not statistically significant. The percentage of successful query submissions (a query which retrieves one or more relevant search results) increased by 79\% for the participants with recommendations. Therefore, the association between two groups and query success rate was identified as statistically significant. Participants who used suggested queries exhibited a higher success rate compared to the participants who typed-in queries. When considering the score of pre-defined tasks a higher normalized mean score was exhibited by the group without recommendations, but the difference in scores was not statistically significant.                     

Overall, the use of query recommendations, increased the number of successful queries. However, the impact of query recommendations on successfully completing a predefined health search task and the overall user satisfaction was not clearly identified, because all the participants did not make use of the query recommendations, the provided query recommendations were not useful for some of the participants with very low health literacy and the measurement of user satisfaction was very subjective. In general, this Health Information Query Assistant system was identified as beneficial for many users performing health related search tasks because the system was able to provide meaningful and user-centred query suggestions.

\textbf{Limitations}

Because of the diversity of the participants' population, the semantic distance measurements between concepts were less precise. It is not always feasible to map user query concepts to UMLS concepts, because UMLS concepts are presented in health care professionals’ language. Some of the rules which were used in this study are not equivalent with some of the ranking methods, such as using an algebraic mean. The query success rates were determined by the authors instead of the participants and therefore, there is a possibility that authors may make mistakes while interpreting participants' retrieval goals. Time spent for each task was not analysed assuming that there can be many reasons for spending various amounts of time on different tasks. It is hard for this system to recognize users' real information needs when a very short query is issued. The aspects, such as the quality and the credibility of the material used were not considered when developing the Health Information Query Assistant system.

\section{Ryan, A., \& Wilson, S. (2008). Internet healthcare: do self-diagnosis sites do more harm than good?. Expert opinion on drug safety, 7(3), 227-229.}

Self-diagnosis sites usually consist of information, such as diagnosed conditions, possible diagnoses for different symptoms and advices to aid consumers in deciding whether to self-treat themselves or consult a doctor. These self-diagnosis websites suggest different diagnoses options for different symptoms reported by consumers. Usually these diagnoses options are provided with the help of private companies and recognised healthcare providers. Self-diagnosis websites were identified as different compared to the traditional text books because  accuracy and quality of content available in such websites are comparatively low, anyone can edit the information available on a website, there is no guarantee that the information available on a website is well edited and checked, some websites might financially exploit its users by trying to sell products and the risks associated with the home treatments and diagnostic acids suggested by these websites. 

There are several advantages of using well constructed quality information about symptoms provided by self-diagnosis websites. 1. The ability of the websites to provide immediate information compared to meeting a doctor by appointment which might take several days, 2. reduce the number of visits to the doctor by making use of home remedies, 3. gain additional knowledge regarding when it is necessary to visit a doctor, 4. gain precise knowledge about the symptoms prior meeting a doctor and clarify treatment options suggested by doctors with the use of that prior knowledge and 5. other subsidiary benefits, such as the availability of tips to stay healthy. 

There are several disadvantages or harms of using self-diagnosis websites. 1. Users of self-diagnosis websites can experience anxiety after reading inaccurate or false diagnoses information about serious conditions, 2. unreliability of information available on self-diagnosis websites because the information might not have been reviewed by a health professional, 3. the possibility of making users buy prescribed drugs from Internet pharmacies and 4. possibility of ignoring professional assistance and diagnosis because of the false reassurance provided by the self-diagnosis websites.        

At present consumers heavily use self-diagnosis websites to search and diagnose certain symptoms that they face and to identify different options available to treat those symptoms. Therefore, it is important to make sure that information available on self-diagnosis websites to be well structured, edited and evidence-based. Reliable self-diagnosis websites can be emphasized by providing them with a ‘quality mark’ which makes it easier for the users to choose high quality self-diagnosis websites. In addition, it is convenient for the users if healthcare, diagnostic and treatment services are provided in lay language because it is much easier for the less educated people to understand.       

\section{Alpay, L., Verhoef, J., Xie, B., Te'eni, D., \& Zwetsloot-Schonk, J. H. M. (2009). Current challenge in consumer health informatics: Bridging the gap between access to information and information understanding. Biomedical informatics insights, 2, BII-S2223.}

This study investigated how tailoring of information (contextualized) impact on the access and understanding of information. The term 'Contextualization' in health information stands for providing supportive information so as to thoroughly explain a health message. Hence, contextualization helps to improve users' understanding of different health concepts by reducing communication complexity. A website known as 'SeniorGezond Website' was used for the evaluation. The study consisted of three main parts, such as a pre-test, tasks involved with the website and a post-test. Forty people participated in the study. Participants were randomly assigned either to exposure or no exposure to contextualization with the SeniorGezond website. This website consisted of structured information about 'fall' incidences with four levels of information. Level 1 consisted of 'causes of falls', level 2 consisted of 'solutions', level 3 consisted of 'products and services' to aid people who experience this issue and level 4 consisted of 'supportive facts', such as information about insurance and addresses of the places where products and services which were mentioned in level 3 are available. The information consisted in level 2 was removed for the participants with no contextualisation because level 2 acted as a buffer between causes(level 1), and products and services (level 3).

The authors observed that the use of contextualization significantly increased the lay users' understanding of health information by raising users' health literacy. Participants' cognitive style (the way they organize, filter, transform and process information) was also identified as a significant factor on thoroughly understanding health content. Participants also used their own contexts (social context and psychological context) as a support to understand health information well. A gap between the informational contexts proposed to users and users' own personal contexts was identified.  

According to these observations the importance of embedding contextualization techniques to health-related websites or tailoring of health information was identified. 

\section{Ye, Z., Gwizdka, J., Lopes, C. T., \& Zhang, Y. (2017). Towards understanding consumers' quality evaluation of online health information: A case study. Proceedings of the Association for Information Science and Technology, 54(1), 838-839.}

This case study investigated how consumers evaluate quality of online health information. The approach of this study has been to use 'eye-tracking' and conducting 'retrospective interviews' in order to investigate different interface elements which are being used by consumers' to evaluate quality and also to investigate whether differences in individuals, such as eHealth literacy, personality, demographic facts and familiarity with health topics have any influence on consumers' behaviour. 12 lay people participated in the study; they were presented with five predefined tasks related to health information search; for each task they were shown 3 preselected web pages and asked if they would recommend those to their family or friends. 

The authors hand-picked results for 2 participants (because of opposite patterns shown in eye glaze and fixation). They observed that, the first participant had viewed the web page content as a F-shaped pattern with fixations focused on the main text content. In contrast the second participant had viewed it in an atypical pattern with scattered fixations. The choice made on this page, time on task, time on page, count of links clicked on and background information (eHEALS and TIPI scores) were compared to analyse the performance of the two participants. In conclusion, the authors noted that the participant with lower eHEALS score is tend to spend a longer time on the task and on the web page than the other participant with a higher eHEALS score. Eye-movement patterns were dependent on facts, such as familiarity with the health topic, health literacy and demographic factors.               

The authors also have formulated a few hypotheses for future studies:

1. Consumers’ with comparably lower eHealth literacy are tend to: 

H1a. ‘spend more time on quality evaluation’
H1b: ‘rely more on relevance than on quality indicators’  

2. Regarding health-related web page quality:

H2a: ‘Consumers with considerably lower eHealth literacy tend to rely more on the main content of a webpage than on quality indicators’
H2b: ‘Consumers with considerably higher eHealth literacy are able to take advantage of quality indicators’ 

\textbf{Limitations}

Their observations are limited by the fact that just 2 participants were studied. Also, in their analysis the authors did not account for the possibility that a participant with a non-focused eye glazing was distracted and not interested in the task.

\section{Lopes, C. T., \& Fernandes, T. A. (2016, September). Health Suggestions: A Chrome Extension to Help Laypersons Search for Health Information. In International Conference of the Cross-Language Evaluation Forum for European Languages (pp. 241-246). Springer, Cham.}

This study investigated how providing health query suggestions to consumers will be useful in successful health information retrieval. The study provided query suggestions in both Portuguese and English based on the initial query's language. Health suggestions were provided in three search engines which are Google, Bing and Yahoo. The ultimate aim was to provide consumers with a mechanism which aids consumers in retrieving high-quality health information which will also fit with their health expertise. The approach of the study was to implement health query suggestions in Google Chrome as a ‘Google Chrome extension’ and then to reach the three search engines. A health suggestion query panel was presented when matches were identified between health suggestions and user queries. Users were able to search for suggested queries (a new search was performed), switch search engines (same search was performed in the new search engine), minimise/maximise the panel, close the panel and navigate to the options page (turn the extension on/off, (dis)allow logging, opt for a local or remote database, specify queries’ language or ask the extension to automatically detect it). Two modules were used in the system. 'Suggestion engine' generated suggestions based on Consumer Health Vocabulary (CHV) which contains links between everyday health-related terms and technical terms. 'Login engine' which consisted of the extension and a server, studied user health search behaviour by tracking different actions, such as time spent on Health Suggestions’ panel, visited web pages, time on web page, number of scroll events, clicks on the extension’s panel, copy/cut events etc. performed by users while searching. 

36 students participated in the study. First group of students received assistance via Health Suggestions. Second group was unassisted. Health topics for the task were selected randomly from the topics mentioned by 20 laypeople (no medical expertise). Participants performed four tasks. For each task they formulated three queries. They were asked to save the most relevant documents from the top 10 results of each query. At the end of each task assisted participants explained how they used Health Suggestions (clicked on them, used terms from one of the suggestions, used terms from several suggestions), why the suggestions were considered useful and assessed the utility of the Health Suggestions. The number of relevant documents retrieved and users' opinions towards the task were used to determine the success of the search. The user satisfaction was scaled from 1 to 5. 4 participants (11\%) classified satisfaction as 3; 26 participants (72\%) classified satisfaction as 4; 6 participants (17\%) classified satisfaction as 5. The authors also evaluated health suggestions based on four research questions:

\textbf{(1) How are suggestions used?}: According to the assisted participants' group, query suggestions were provided for 71\% of the issued queries. In 27\% of these cases users clicked on suggestion; in 15\% of the cases users extracted terms from the suggestions and included them in their next query; in 4\% of the cases users extracted terms from several suggestions.           

\textbf{(2) Why are suggestions used?}: Five main reasons were observed. In 35\% of the cases because they presented synonyms; in 37\% of the cases because they presented alternatives in medico-scientific terminology; in 24\% of the cases because they suggested English terms; in 3\% of the cases because of the lay terminology. 

\textbf{(3) How do users assess the utility of the suggestions provided by the system?}: A scale in a range of 1-3 was used. 35\% of the queries were considered useful; 33\% partially useful; 29\% were useless.   

\textbf{(4) Do the suggestions lead to a more successful search?}: The average number of relevant documents saved were observed. The average was 16.3 for the assisted group and 14.1 for the unassisted group. The overall task success was scaled from 1 to 5. The mean value was 4 for the assisted group and 5 for the unassisted group. However, this difference was not considered as significant.   

In conclusion, health query suggestions from the tool were well accepted by the users, both lay terms and medico-scientific term suggestions in Health Suggestions lead to more successful searches; the utility of a multilingual and multi-terminology approach is important and useful to retrieve a huge number of relevant documents.     

\section{Silva, R., \& Lopes, C. (2016). The Effectiveness of Query Expansion when searching for Health related Content: InfoLab at CLEF eHealth 2016. In CLEF (Working Notes) (pp. 130-142).}

This study investigated how query expansion (supplementing the original query with additional terms) will aid laypeople in improving initial queries and then the overall retrieval performance. The impact of query expansion can be determined by evaluating the set of ranked list of documents retrieved from a recommended test collection. Two sub tasks, ad-hoc search (treating each query individually) and query variation (treating a group of query variations as one query) were performed. ClueWeb12 B13 Dataset was used as the document collection. Query generators created initial queries based on health consumer posts. ‘Terrier’ was used as the indexing technique. A set of different sources and methods were used to pick terms to be added to the initial queries.        

\textbf{Baseline}: 'BM25 term weighting model' was used to score and rank medical documents. Ranking was performed based on documents' relevance to the issued search query. 

\textbf{Pseudo Relevance feedback}: This method modified queries based on the top documents retrieved by the baseline approach. Existing query terms were re-weighted to pick useful terms and to drop useless terms. Two weighting models were used. The Bose-Einstein and the Kullback-Leibler Divergence. Bose-Einstein is based on the frequency of a query term in the top ranked documents. Kullback-Leibler Divergence is based on the probability of a query term in the top ranked documents. The most suitable model was determined by performing two runs.             

\textbf{Query expansion using the Medical Text Indexer}: This method linked query text to the Medical Subject Headings (MeSH) vocabulary. This provided additional related concepts. All these concepts were appended to the initial query.  

\textbf{Query expansion using Wikipedia}: Wikipedia was chosen as a source because it contains health information and medical terms in lay language. Two methods were used to obtain expansion terms from Wikipedia. Term frequency was used to extract the most frequent terms related to the MTI concepts, from the Wikipedia articles. The 5, 10 and 15 most frequent terms of each article were chosen. Link analysis was used to identify similar articles to extract expansion terms from the titles of the articles. Jaccard similarity coefficient was used to filer out all the irrelevant articles. Titles of the articles with Jaccard similarity coefficient greater than 0.25, 0.50 and 0.75 were chosen.              

\textbf{Query expansion using MedlinePlus}: Information from the infoboxes of the Wikipedia pages were used to select related MedlinePlus pages. The sections, such as Causes, Symptoms, Treatment, Possible Complications and Alternative Names of the MedlinePlus pages were used for the query expansion process. The 5, 10 and 15 most frequent terms from each relevant section were chosen.       

\textbf{Query expansion using the ICD-10}: Information from the infoboxes of the Wikipedia pages were used to select related ICD-10 pages. Information regarding diseases or symptoms which are related to initial query concepts were used for query expansion process. The 5, 10 and 15 most frequent terms of an ICD-10 page were chosen to be appended with the original query.      

\textbf{Query expansion using Latent Dirichlet Allocation over Wikipedia}: This method generated different latent topics represented by the text in Wikipedia articles with different MTI concepts. A combination of 3 topics with 1, 5, 10 words and 1, 5, 10 topics with 5 words were chosen.     

\textbf{Query expansion using Unified Medical Language System}: This method extracted terms from the UMLS definitions which were related to the MTI concepts. The top 5, 10 and 15 most frequent terms related to each MTI concept were chosen from the UMLS definitions for query expansion.  

\textbf{Readability}: Three readability metrices, such as SMOG, FOG and Flesch-Kinciad representing the ‘educational grade level’ needed to understand a document were used. SMOG was calculated based on the 'number of polysyllable words in 30 sentences'. FOG was calculated based on the 'average sentence length' and the 'percentage of hard words'. Flesch-Kinciad was calculated based on the 'average sentence length' and the 'average number of syllables per word'. Three formulas with the combination of readability metrics and relevance scores were used to re-rank the documents retrieved using expanded queries. 

Three runs were submitted. Firstly the Baseline approach; secondly the Wikipedia Link Analysis with a Jaccard similarity coefficient above 0.50; finally the Latent Dirichlet Allocation with 3 topics and 5 words. SMOG readability metric and one combination formula were used for re-ranking. However, because the unavailability of the relevance and readability assessments for the test collection at the time of reporting this study, it was impossible for the authors to compare results with the baseline for each sub task and to make any conclusions regarding the results which have been obtained by following different approaches.            

\textbf{Limitations}

One major limitation was the unavailability of the relevance and readability assessments for the test collection at the time of reporting the study. Another limitation was that Medical Text Indexer results were machine generated, so there was a possibility of generating irrelevant concepts.     

\section{Graham, L., Tse, T., \& Keselman, A. (2006). Exploring user navigation during online health information seeking. In AMIA Annual Symposium Proceedings (Vol. 2006, p. 299). American Medical Informatics Association.}

This study investigated user navigation behaviour while seeking for online health-related information. ClinicalTrials.gov web site's log data were obtained over a three month period. Data was extracted from log files using Transaction Log Analysis (TLA). Investigation focused on online search behaviour, query failures, navigation, browsing strategies, clicking on links, initiating queries, and task-oriented actions (logging into a system). Log data were converted to XML data structures and these structures included a client (a unique IP address), a session (a set of sequential actions performed during the search) and a request (a user action and its web server response). Web pages were aggregated into functional categories, such as search, browse, view results, view study etc. for the analysis. Page reviews, referral frequencies, page transition frequencies (presented as a single transitions table by computing the frequencies), navigation path frequencies (determined using algorithms), click stream data, participant comments and usage statistics were assessed. Then a pilot user study was conducted. 7 lay consumers were presented with two hypothetical scenarios; ‘sleep apnea’ and ‘Parkinson’s disease’. They were assigned to either one of the scenarios to perform the search. Any online resource could be used as their choice. 

By analysing ClinicalTrials.gov log data the authors observed that the ‘View Study' pages were the most frequently requested (40\%) and the most common entry and exit points to ClinicalTrials.gov. 'View Results' pages were the second most frequently requested (25\%). For 69\% of the sessions the initiation point was an external web site, such as search engines and government sites. The top five referring web sites were Google (41\%), NIH.gov domain (18\%), Yahoo, MSN and AIDSinfo which accounted for 66\% of all referrals. View Study (39\%), View Results and Opening Screen (homepage)  (24\%) pages served as web site entry and exit points. Viewing two studies in a row or viewing a study and exiting (40\%), and clicking on a specific study in the results (9\%) list were the most frequent moves between pages indicating that users were moving among a limited set of pages (only a few user online navigation activities). 8 common user navigation patterns were identified. These patterns revealed that users were tend to access ClinicalTrials.gov directly via ‘View Study’ pages. Search and browse features were directly used. Time was spent on exploring the site and refining search queries. Other available site features ‘Search within results’ and ‘Resources’ were not used.

The analysis of the pilot study revealed that 5 out of 7 consumers viewed at least one page at ClinicalTrials.gov during their session. Two of them referred the web site directly from search engines, another two from MedlinePlus and one from a non-profit health Web site. The most frequent enter and exit point was ' View Results' page in contrast to TLA data with most frequent entry/exit point as 'View Study' page. For search queries with general terms users navigated to 'View results' page and for search queries with specific terms users navigated to 'View Study' page. The name recognition, “dot-gov” domains and keywords were used by participants to judge the relevance of the web sites. All the participants' were satisfied about the overall search task. Pilot user study also had a similar structure of entry and exit points as the log data. Most moves were occurred within the pages opening screen, view results and view study. External web sites, such as search engines directed users to results pages and individual studies. 

The majority of the users are tend to refer to low-level pages (View Study) from external web sites, such as search engines. Therefore, it is worth placing links to background information and other search features on lower-level pages which are being directly accessed by users. Visible local maps retain users on web sites.

\section{Thenmozhi, D., Mirunalini, P., \& Aravindan, C. (2016). Decision Tree Approach for Consumer Health Information Search. In FIRE (Working Notes) (pp. 221-225).}

This study investigated how to determine sentences of a document is relevant, given a Consumer Health Information Search query and a document associated with the query. The approach of this study was to use a machine learning technique to categorize retrieved health-related information as relevant or irrelevant based on the issued query. A decision tree was created by following four steps. 1. Preprocessing the given text by removing punctuations and adding annotations for parts of speech, such as nouns, verbs and adjectives. Nouns and adjectives were chosen as the features from the training data. 2. Feature selection or extracting features for training data. Two variations were used. In the approach without feature selection, the linguistic features were used without explicit feature selection. The features were lemmatized to bring the features to their root form. In the approach with χ 2 feature selection, a chi-square value was computed to pick the important features from the linguistic features. A maximum χ 2 statistic value was calculated to select the features with a strong dependency on the categories.The occurrence or non-occurrence of each feature in relevant and irrelevant instances were determined.The expected frequency for each feature was also calculated using the observed frequency. 3. Using the selected features of training data to build a model using a classifier. For the approach without feature selection the features vector of training data was used to extract features from the test data with unknown labels. For the approach with feature selection, the features with  χ 2 statistic value greater than 3.841 were considered as significant features and were used to build the model. 4. Predict the class label either as “relevant” or “irrelevant” for the instance using the built decision tree model. 

The data set consisted of five queries, and training and test data for each query. The number of nodes for the χ 2 feature selection tree were considerably lower than that of tree without feature selection. The reduction in size of the tree for χ 2 feature selection was statistically significant. The authors observed that the approach without feature selection had higher cross-validation accuracy for each query than for the approach with χ2 feature selection. However, χ2 feature selection method had a 2.23\% higher test data accuracy compared to the method without feature selection. The improvement in performance for the χ2 feature selection method was statistically significant. It was shown that the feature selection approach is able to significantly reduce the size of the model(decision tree) without compromising the performance. In conclusion, since χ2 feature selection model can categorize health-related documents as relevant and irrelevant successfully, it can be used for filtering out irrelevant health related documents prior presenting the results to consumers/ laypeople. This would help consumers in retrieving more relevant health related documents according to their issued query.

\section{Broussard, R., \& Zhang, Y. (2013). Seeking treatment options: consumers' search behaviors and cognitive activities. Proceedings of the Association for Information Science and Technology, 50(1), 1-10.} 

This study investigated consumers’ exploration of treatment options in both behavioural and cognitive perspectives. Two research questions were addressed; 1. the behaviours consumers exhibit when searching for treatment options online; 2. the cognitive activities involved in the search: finding, selecting and evaluating information. The approach of the study had two parts; 1. participant observation; 2. post-session interview. 40 people participated in the study. Two interfaces were used; 1. a classic Web search engine interface with a basic search box and the results were presented as a ranked list; 2. a Scatter/Gather-enabled search interface with a basic search box, but the results were grouped into a number of clusters. These clusters were ranked based on their size. 20 participants were assigned to each interface and were asked to search treatment options for migraines’. Four types of data were collected; 1. participants’ demographics and experience with health information search; 2. transaction logs about session length, queries submitted, sites visited, participants’ ratings on relevance and usefulness of the sites visited; 3. participants’ ratings on the mental efforts required to complete the task and their satisfaction with performance; 4. information gathered via playback interviews (playback of the search process): selection of keywords, query reformulation and examination, selection of particular search results, and evaluation of search results. Two different interface groups were pooled together for further analysis because they had no difference in information, such as session length, number of queries submitted, and sites visited.                       

The authors observed that the mean values for web experience, health search experience and health search frequency were 13.2 (years), 3.7 (years) and 2.8 (times per month) respectively. Average ratings for the perceptions of the task were; familiar: 2.3 (somewhat familiar); easy: 2.5 (somewhat difficult); effortful: 2.7 (medium amount of effort) and satisfied (with results): 4.2 (generally satisfied). Three aspects were analysed for search behaviour. 1. Session length: average session length was 10.25 minutes; 2. Basic query behaviour: average number of queries issued was 4.2 and average query length was 3.8 terms; 3. Web sites visited (60\% of all visits): frequently visited medical-specific websites: WebMD (26 visits), livestrong (22 visits), mayoclinic (17 visits), migraines.org (19 visits) and migraines.com (10 visits); frequently visited general-purpose sites:  ehow (22 visits), Wikipedia (18 visits) and buzzle.com (5 visits). Sites with checkout behaviour (73.2\% of all the visits): medical-specific, evidence-based and general-purpose sites. Sites visited twice: 50\% general purpose sites and 50\% medical-related sites. 44 unique websites were visited only once. Four aspects were analysed for cognitive activities. 1.Query formulation and reformulation (89 times): specification (22), generalization (9), parallel move (25), new concept (27) and rephrase (6). 2. Examination of search results: trial and error approach was taken to decide whether to select a search result or not. The rank of a result, familiarity of websites and the validity of information were important when selecting a search result. 3. Judgement of sources: the design (clarity, simplicity, clean and well-structured text with images), readability (text without jargon), completeness (pros and cons, side effects and expensiveness of the treatments), and credibility (depending on participants' experience and the content of websites) were used to judge sources. 4. Search timeline -cognitive development: one pattern was to start with a general search and gradually move towards more specific information; another pattern was to start from confirming information and gradually move towards novel information and validating that novel information (double check a fact).

In conclusion, the behaviours consumers' exhibit when searching for treatment options online:  submitting short queries with misspelled keywords and selecting results exclusively from the first page of search results. Websites visited: participants relied on both medical-specific and general web sites; web sites with its name containing the medical condition searched were visited more frequently (20\% of the 66 unique sites). Cognitive activities performed when searching for treatment options: start the search with general concepts and gradually move towards different treatments and the aspects of each treatment, and select search results based on familiarity. Search engine rankings and familiarity of the web sites were more important when selecting results when compared to trustworthiness, quality or usefulness of the information contained in the websites. Therefore, search engines need to be designed in a way that they will support different consumer behaviours.  

\textbf{Limitations}

The task was a simulated task and did not reflect the participants’ real needs. Due to the exploratory nature of the study, only one treatment option task was assigned to the participants, thus participants’ behaviour may be affected by the nature of the task. A lab setting was used for the study, but it is not a natural environment for treatment search.  

\section{Keselman, A., Browne, A. C., \& Kaufman, D. R. (2008). Consumer health information seeking as hypothesis testing. Journal of the American Medical Informatics Association, 15(4), 484-495.} 

This study investigated the most common patterns consumers follow when searching health information, depending on their initial theories, search strategies and comprehension. These patterns were then categorized as successful and failed. A framework was used for the study. This framework had two perspectives: hypothesis testing perspective and Human Computer Interaction (HCI) perspective to obtain additional insight regarding consumers’ difficulties in locating and interpreting health information. 4 states were considered: 1. Beginning state which constitutes of the background knowledge (theory) and initial hypotheses (perceived information need); 2. Search goal; 3. Shaping the search goal by search action steps; 4. Evaluation of retrieved information. 20 lay individuals with different levels of education participated in the study. They were presented with a hypothetical scenario which described symptoms of ‘stable angina’. They were asked to do two tasks: 1. Semi-structured interviews where possible causes of the symptoms were discussed and 2. Search MedlinePlus to seek information on the disease. Information gathered via semi-structured interviews was compared with a reference model of stable angina. The searching processes were categorized as action-related and competency processes. Action-related represented actions performed or steps of information seeking path. Competency level included: Domain Knowledge (e.g., of heart disease),General Search Strategies (e.g., query expansion), Resource Knowledge (of MedlinePlus), Metaknowledge (of desirable site characteristics) and Language (spelling and vocabulary knowledge). Each action step was assigned with one of these competencies. Each searching process was integrated with the corresponding verbal transcripts obtained during semi-structured interviews for further analysis (examine search strategies and visualize trends across participants’ behaviours). 

The authors observed that overall participants’ understanding of the scenario was incorrect or imprecise. Participants' understanding was different to the reference model in three aspects. Firstly, the key concepts were analysed. Coronary artery disease (CAD) and angina were in the reference model, but no participant mentioned them. The third concept was atherosclerosis in the reference model, but only 3 participants mentioned a lexical form of this term. For the majority of the participants 'heart attack' was the primary hypothesis. ‘Blockage of blood vessels’ was the main cause for such cardiac problems. Tear to the heart muscles, irregular heart beat and “electrical problem with the heart” were the other mentioned causes. Non-cardiac problems, such as stroke, arthritis, asthma and diabetes were also mentioned by the participants. Secondly, symptoms’ grouping was analysed. Reference model indicated that all symptoms were related to one condition. In contrast participants mentioned that nausea and dizziness are unrelated to a cardiac problem. Finally, symptoms’ characteristics were analysed. The short duration of pain, its relation to exertion and response to rest were not noted by the participants. In contrast the reference model identified the significance of these factors. 

Participants were categorized into three main clusters; Verification-First, Problem Area Search-First and Bottom-Up for the analysis of Information-Seeking Processes. Firstly, the 'Verification-First' Cluster was analysed. It represented 8 participants (40\%). They started the search process by attempting to verify a specific illness related to heart attack. The only strategy used was ‘verification’. One participant correctly concluded the scenario as angina. Majority of them (7) incorrectly concluded the scenario as related to heart attack by only considering the similarities between descriptions of heart attacks and symptoms in the scenario. The behaviour of ignoring symptoms’ characteristics that are viewed as non-essential was identified as ‘selective perception bias’. The behaviour of stating that information confirmed their hypothesis depending on information of a site they visited, was identified as ‘confirmation bias’. The behaviour of stopping search just after reviewing only one content topic was identified as ‘premature search termination bias’. Secondly, the 'Problem Area Narrowing-First' Cluster was analysed. It represented 5 participants (25\%). They started the search process with problem area search. Participants had both ‘Area’ hypothesis (start searching with queries) and ‘Assorted’ hypothesis ( browse the site index tree). One participant had switched to a ‘bottom-up’ approach during this searching process. Sites with specific disease information were visited. Participants had left without a conclusion rather than providing incorrect conclusions. The behaviour of picking text from the scenario (pain episode only lasts 2–3 minutes) and making conclusions (very minor heart attacks) based on that was identified as ‘selective perception bias'. Finally, the 'Bottom-Up First' Cluster was analysed. It represented 7 participants (35\%). Some started the search process without a specific hypothesis. The attempts to search for a general-purpose diagnostic tool were unsuccessful. Some participants had  switched to other ‘hypothesis-driven strategies’, but made incorrect heart attack conclusion. Neither of the participants in this cluster had made the exact correct conclusion as angina. The behaviour of trying to match description of a heart attack with the facts mentioned in the scenario was identified as ‘selective perception bias’.                      

Other factors, such as domain knowledge was important for setting goals and information evaluation. Domain understanding was important for determining the direction of the search and for the interpretation of the results. Resource knowledge, strategies and metaknowledge were important for navigational actions. Participants were able to understand the symptoms described in the scenario disregard of their education level. Participants with higher levels of education were likely to be familiar with MedlinePlus, had used efficient search strategies and made meta-level comments, such as judging the authoritativeness of a source.

In conclusion, incomplete and inaccurate domain knowledge (setting information goals and evaluating retrieved information), imprecise search queries entered and the inconvenient configuration of the web resources were identified as main causes of the problems faced by participants when searching for information and selecting correct results. An incorrect hypothesis can cause the search of irrelevant resources. Prior hypothesis and background knowledge influences hypothesis generation, evidence interpretation and evaluation of retrieved information. Two aspects of MedlinePlus interface influenced participants’ search process. 1. Lack of explicitness in relating lay and professional terms in the index and 2. The order and organization of query results lists. Searching processes with a specific preconception ended with incorrect conclusions. Searching processes without a specific preconception ended without any conclusion.The importance of identifying the difficulties faced by consumers’ is that health information website(consumer health sites) designers can try eliminate these difficulties by providing support in places where consumers were tend to behave erroneously. In information portals like MedlinePlus it is worth providing query suggestions, presenting information in an organized manner and suggesting consultation with a health professional. In individual websites it is worth addressing needs of targeted users with consumer-friendly terminology. In education tools it is worth teaching consumers how to formulate specific queries, evaluate qualifiers in the information and not to terminate their searches prematurely.

\textbf{Limitations}

The hypothetical nature of the scenario (possibility of affecting users’ motivation), the complexity of the diagnostic task and the similarity between angina and heart attack symptoms (ambiguity in the description of the scenario) were identified as limitations and potential reasons which led this study to result in a very low success rate. 

\section{Zhang, Y., Wang, P., Heaton, A., \& Winkler, H. (2012, January). Health information searching behavior in MedlinePlus and the impact of tasks. In Proceedings of the 2nd ACM SIGHIT International Health Informatics Symposium (pp. 641-650). ACM.} 

This study investigated consumer health information searching behaviour (interaction behaviour) in MedlinePlus which was a browsing oriented system with a simple search engine, and the impact of the number of concepts involved on the search behaviours. Three research  questions were examined. 1. How do users search MedlinePlus?; 2. How do users browse MedlinePlus?; 3. how do search tasks influence interaction strategies?. The difficulties faced by consumers when searching and how they handled these problems were also examined. 20 undergraduate students who have not used MedlinePlus before participated in the study. Three search tasks were given to each participant. 1. Find arguments for and against the use of marijuana for medical purposes; 2. Find the relation between Type I and Type II diabetes, and hypertension; 3. Find the functions of liver and kidney, what is the role of insulin in the liver and kidney, why insulin would be needed? and whether insulin is related to liver and kidney diseases?. Prior to performing the tasks information regarding participants' spatial ability and demographic factors were collected. Participants' searching and browsing behaviours were examined during the search tasks. After each task participants' opinions on difficulty of the task, required mental effort and satisfaction of the search performance in MedlinePlus website were collected.                

The authors observed that participants' spatial ability score ranged between 6.8 to 17.6. Internet experiences ranged between 6 to 13 years. Online health information searching frequency was on a yearly or monthly basis. Session lengths varied between 11.63 to 29.01 minutes. The time spent, difficulty and the required mental effort for task 3 were significantly higher than the others. Satisfaction of task 3 was the lowest among three tasks. Four aspects were considered for the analysis of search behaviour. 1. Query features (number of queries and terms): task 3 had the maximum number of queries per person which was statistically significant and task 2 had the maximum average query terms which was not statistically significant. 2. Search terms: three types of search terms were identified; (1) keywords with semantic meanings (meaningful terms); (2) stop words (no semantic meaning); (3) search operators (AND, OR). 3. Query reformulation: two factors were considered. (1) Executed actions which modified the initial queries. Three actions were identified. Concept related changes: add (28.9\%), delete, repeat/re-execution, replace (30\%) and change to a new concept/ switching topic (20.7\%). Task 3 had the maximum number of query reformulations. Form of terms: change the form of term and correct misspelled terms. Conceptual relationships (boolean operators): changing boolean operators and altering the order of the words. (2) Subsequent conceptual changes to the queries. Four categories were identified; specification (the inquiry becomes more specific); generalization (the inquiry becomes more general); parallel movement (the reformulated query has a partial overlap with the previous query/ two aspects of one query); replacement with synonym (replace the current terms with words with similar meaning). 4. Accessing and evaluation of results: two options, such as directly access the results (results from Health Topics section and Encyclopedia) and filter out results. Majority directly accessed the results rather than filtering the results. A few also used the options to filter out the results. The results were evaluated by scanning, checking the authors, checking the source of information and the organization of the content in the resources. Two aspects were considered for the analysis of browse behaviour. 1. Accessing different resources; participants accessed 'Drugs \& Supplements' and 'Encyclopedia' from the alphabetical lists, Dictionary, Health Topics section ('Alphabetical list', 'Body Location/System', 'Diagnosis and Therapy' and 'Health and Wellness'), news section,  Directories, Go Local and Multiple Languages sections. 2. Accessing related topics; in-text and related topics list hyperlinks available on MedlinePlus health topic pages were used for task 3. 

In conclusion, session length of a search process increases with the complexity of the task. Most query reformations were related to conceptual changes (85\% ) while making queries specific,  followed by making queries more general, switching to a new topic and making parallel movements. Query re-executions were performed the most as query iterations. Participants with higher understanding of the scenario explained in the search task also searched for concepts associated with the concepts mentioned in the scenarios. The use of stop words demonstrated participants' preference of using natural language during search. Generally searching and browsing strategies were used as a combination mostly when performing task 3 (relatively higher complexity). Only one strategy was used mostly when performing task 1 and task 2 (relatively lower complexity). The searching process patterns of task 3 were more diverse, complex, exploratory, and iterative when compared to other two tasks. This indicated that task complexity influences interaction strategies and patterns. Searching strategies were preferred the most because it was easier to perform than browsing strategies. Encyclopedia, Health Topics, dictionary and links to related topics were used to gain preliminary understanding of the scenarios in complex tasks, such as task 3. Therefore, the need for functions to detect misspellings and provide query suggestions with automatic synonym expansions, a hierarchical terminology structure to assist participants in selecting query terms for query construction, suggestions for alternative moves when reformulating queries, the visibility of information architecture to users, advanced search function to support exploring relationships between multiple health concepts, backtrack function to access search histories, mechanisms to connect lay terms with medical terms and encouragement for evaluating search results were identified when designing consumer health-related information websites, so as to aid consumers when performing search tasks with varying complexity.         

\textbf{Limitations}

This study recruited undergraduate students and they tended to have more versed skills in web searching but less often on health-related topics. Predefined tasks were used for the study rather than focusing on users’ real needs. It is important to perform the same tasks in different consumer health information websites to understand how different information architectures or structures and interfaces influence interaction behaviour.

\section{Toms, E. G., \& Latter, C. (2007). How consumers search for health information. Health informatics journal, 13(3), 223-235.}

This study investigated consumer health information searching behaviour online. Three aspects were considered. 1. How people specify their information requests; 2. how people select from search result lists; 3. How people examine the page(s) they declare relevant to the information search problem. 48 young, educated adults who have used the web before participated in the study. Search tasks were performed in Google. A set of Open Directory categories were added to Google so as to provide participants with an alternative option which is a scan capability. Four tasks were given to participants. Two tasks were fully specified; 1. Find three categories of people who should or should not get a flu shot and why; 2. Find a website likely to contain reliable information on the effect of second-hand smoke. The other two tasks could be personalized; 3. List two of the generally recommended treatments for (fill in the blank with a health-related matter that interests you); 4. Identify two pros or cons of taking large doses of (fill in the blank with a drug or treatment or remedy that interests you). Information regarding participants' demographic factors and web search experience were collected. Each participant was assigned to one of the four tasks. Information, such as participants' familiarity and expertise with the search topic, actions performed while searching for the topic (issuing queries, selecting categories, examining pages and selecting results), and participants' perception on completing the task was collected. Judges assessed the ‘aboutness’ of the pages declared as relevant by the participants. Completeness of the tasks were assessed by examining the pages which were recommended as useful by the participants.

The authors observed that participants were ‘somewhat familiar’ to ‘very familiar’ (66\%) with the topics of the search tasks. Average number of queries created for each task: 1.3 queries. Average number of keywords in each query: 4.3 keywords. 63\% used only the search box to enter a query. 6\% used the categories. The remainder used both queries and categories. On average 4.5-9 minutes were spent per task. Time was mostly spent on interpreting results page to comprehend the information presented on the web page. Formulation of queries and selection of categories were analysed. To perform the first two tasks participants issued various queries and selected various categories. For task 1, 12 participants issued 23 entries. However, for the task 2 less variability in query content was observed. Query formulation was a quick process. The use of keyword search either on a search engine or on a website of interest was the most popular query formulation. Keyword search was also used as a ‘trial and error’ method. With the use of keyword search participants were able to control the search process and results. They made the search more robust ( adding specific words while searching), restricted the search by specific sources of information and the reliability of the sources of information, preceded the search terms with a plus sign to ensure results fell within a common page and encapsulated search terms in quotation marks to ensure that all terms are found together in sites identified in the search results. Therefore keyword search method was preferred over the use of categories because categories were seen as a less direct route (included more steps) to locate information, information contained in categories was too general, increased exposure to advertisements and increased time taken to find relevant information.

How participants selected from the results list was analysed. On average 5.4 pages of results lists were examined. Average rank on a results list page was 4 therefore, all items on the results list page had an equal likelihood of being selected. Summaries/ description, URLs (determine the nature of the site:an information-based site or an e-commerce site) and titles were the mostly used to select the most appropriate links to explore. In addition, dates, size of the site and type of file (e.g. PDF) were least used when selecting appropriate links. The retrieval of a large number of results was indicated as the search needed to be refined. URLs were used to  assess the potential credibility (university, government, scientific, pharmaceutical research information and associations for medical professionals were considered reliable) of the information found. Some participants looked at links on several pages of the results list and some chose the first link on the results list. How participants identified appropriate websites was analysed. Average number of web pages selected was 2.6 web pages. Average web page ‘aboutness’ was 4.5 out of 5. An average score of 4.5 out of 5 was assigned to the pages which were identified as relevant to the tasks by the participants. Therefore, the pages examined were clearly on the topic of the search and the task completion was between 80-100\% of each task. 44\% of the relevant pages did not had any advertisements and 40\% were created by a government agency. The pages were classified as informational,journal articles, fact sheets and newspaper articles. The use or rejection of a site was performed based on three key considerations; 1. Information expectations: does the site provide any or all of the information being sought?; 2. Information quality: author of the site and the information, the purpose of the site (e.g. to provide information, to sell products, to promote unsubstantiated opinions), the way information was presented (e.g. opinion-based articles, discussion group, academic research, formal medical findings); 3. Information presentation and accessibility: ease of finding the desired information on the site, whether the information is easy to understand. 

Factors, such as what participants were looking for, their knowledge of the area, their biases in terms of credibility and their personal understanding of the topic influenced keyword searches. The system limited the improvements that could be performed while constructing queries and the apparent system improvements were not realized by the participants. Not knowing what a category contains and having an endless set of category levels were identified as barriers when using categories. The information design of the pages had a significant barrier to the online health consumer because time taken to select from the results pages was high. Erroneous decisions were made by considering the web page appearances. Credibility, reliability and trustworthiness were important when selecting pages from results lists. The major challenges for a successful search were formulating good queries and having a results list with appropriate design and standard (easy scanning and efficient recognition).  

In conclusion, consumer health information search still remains as a challenging task for the average person because they have different comprehension levels, searching abilities and levels of information needs. Both information design and search engine technology were identified as important to build good consumer health information systems. Therefore, the need for a system which provides assistance to query development, which will evaluate the information that is being provided, which will have a flexible and responsive terminological infrastructure to support the search process and which will consist of summaries on the results list indicating the  reliability, credibility, type and information content of the documents that they represent was identified.   

\section{Jimmy, Zuccon, G., \& Koopman, B. (2018, March). Choices in Knowledge-Base Retrieval for Consumer Health Search. In European Conference on Information Retrieval (pp. 72-85). Springer, Cham.}  

This paper investigated how to overcome consumer health search problems by expanding or reformulating health queries with more effective terms. Specialised health knowledge bases (MeSH and UMLS) and general knowledge bases (Wikipedia) were used to expand consumer health queries. The 'Entity Query Feature Expansion' model was used for retrieval on both the Wikipedia and UMLS knowledge bases in this empirical evaluation. In Wikipedia KB page title, categories, links, aliases, and body were the useful features for a retrieval scenario. In UMLS KB concept unique identifier, aliases, body, parent concepts and related concepts were the useful features for a retrieval scenario. The impact of different choices in knowledge based retrieval on query expansion and retrieval effectiveness of consumer health search was examined in five different aspects. Firstly, knowledge based construction was examined. For Wikipedia three choices were made to collect health related pages; 1. pages with Medicine infobox type; 2. pages with Medicine infobox type and with links to medical terminologies, such as Mesh and UMLS; 3. pages with title matching an UMLS entity. For UMLS two choices were made; 1. all entities; 2. entities related to four key aspects of medical decision criteria, such as symptoms, diagnostic test, diagnoses and treatments.  Secondly, entity mention extraction was examined. Text from the query which can be mapped to entities were identified. Thirdly, entity mapping was examined. Exact matches between mentions and entities were identified to map a mention to an entity. Fourthly, source of expansion was examined. The sources from the knowledge bases were selected to use for drawing candidate terms for query expansion. Finally, the use of relevance feedback was examined. Both explicit relevance feedback and Pseudo Relevance Feedback were considered. A term was considered as health related if it exactly matched with a title or an alias of an entity (a single Wikipedia page or the most frequently used terms for a single concept unique identifier) in the target knowledge base (Wikipedia or UMLS). The preference to use a specialised knowledge base over a general knowledge base was also investigated. 

300 query topics were considered for this study. A baseline was used for the evaluation, and the title field and the body field were considered for the evaluation. Wikipedia knowledge base consisted of a set of candidate pages and they were indexed using field-based indexing. Therefore, the fields were used as a source of query expansion terms. UMLS knowledge base consisted of English terms indexed with fields, such as title, aliases, body, parent and related. The average number of terms added in the expanded query and the number of expanded queries were recorded.

1. Knowledge Base Construction: for Wikipedia knowledge base, infobox type and links to medical terminologies had the highest effectiveness in information retrieval; for UMLS knowledge base, all entities had the highest effectiveness in information retrieval; baseline method performed considerably better than any of the knowledge base retrieval methods because  knowledge base retrieval methods ranked many unjudged documents amongst the top 10 results for a large number of queries in contrast to the lower number of unjudged documents ranked by the baseline method.

2. Entity Mentions Extraction: for Wikipedia knowledge base, the entity mentions which included only those uni-, bi- and tri-grams of the query that
matched entities (Wikipedia pages) in the Consumer Health Vocabulary had the highest effectiveness in information retrieval; for UMLS knowledge base, the entity mentions using all uni-, bi-, and tri-grams of the queries terms had the highest retrieval effectiveness.

3. Entity Mapping: for both Wikipedia knowledge base and UMLS knowledge base, mapping entities to Aliases had the highest retrieval effectiveness. 

4. Source of Expansion: for both knowledge bases the selection of titles as the source of expansion had the highest retrieval effectiveness. 

5. Relevance Feedback: for Wikipedia knowledge base the addition of feedback had produced mixed results; for UMLS knowledge base explicit relevance feedback had produced the best performance; however the baseline method had performed worse when compared to the knowledge base methods because only explicit relevance information was able to improve its retrieval performance.  

In conclusion, pseudo relevance feedback did not improve results, independently of the knowledge base. In contrast, relevance feedback provided improved effectiveness in both queries that were expanded and the extent of expansion. UMLS knowledge base generally provided better improved effectiveness when compared to Wikipedia knowledge base. UMLS knowledge base expanded more queries than the Wikipedia knowledge base because the Wikipedia knowledge base was comparatively incomplete by only having pages with health infobox and links to medical terms. The two methods had provided different query expansions with an average of only 8.9\% of common expansion terms. The two methods retrieved different sets of documents with an average overlap of 61\% of the top 1,000 documents.  The effective choices for knowledge base query expansion lead to produce the lowest number of expansion terms and expanded the smallest number of queries. Relevance feedback had added a significant number of expansion terms and expanded a large number of queries.  In total by both the knowledge bases, 183 queries were expanded. 16 of these expansions had no change in effectiveness when compared to the baseline. 92 showed improvements and 75 showed losses. Overall, UMLS knowledge base was more effective when compared to Wikipedia knowledge base. UMLS knowledge base also performed better than the baseline. Therefore, it was confirmed that the use of a knowledge-base retrieval approach has the ability to translate well into the challenging consumer health search domain.                 

\textbf{Limitations}

The number of unjudged documents retrieved using the expanded queries was the major limitation of this study, which made it challenging to fairly evaluate the methods. In spite of that, this investigation of choices in knowledge base retrieval for consumer health search has been able to highlight both pitfalls and payoffs.

\section{Sillence, E., Briggs, P., Harris, P. R., \& Fishwick, L. (2007). How do patients evaluate and make use of online health information?. Social science \& medicine, 64(9), 1853-1862.} 

This study investigated how menopausal women search the Internet for information and advice on hormone replacement therapy (HRT) and how they determine which sites/information to trust. A staged model of trust development was used for evaluation. This model represents the way users firstly assess sites by screening them (heuristic stage/ initial searching process), then move on to in-depth evaluation of available information (analytic stage) and finally how they develop a long-term trusting relationship with a few particular sites. 15 menopausal women who were interested in assessing the costs and benefits of taking HRT participated in the study. The authors aimed at evaluating how the Internet advice influenced these consumers' decision making and communication with physicians. The research was divided into 3 phases:

Phase 1: Online search behaviours were evaluated. 4 weekly sessions were conducted: during each session, consumers searched the Internet for information and advice regarding menopause and had a discussion with a facilitator. Two searches were freely performed on the Web and the other two were performed within specific websites. These specific websites represented a range of information providers, different content and design features to evaluate the impact of them on trust of the sources. The authors observed that participants efficiently recognized and rejected web pages that acted as starting points or gateways to other sites, websites unrelated to menopause, websites which were not menopause specific, websites with broken links and sales sites. According to the analysis of participants interviews, design factors such as complex layout, too much text, corporate look and feel etc. and content factors such as irrelevant material and inappropriate material were the main reasons to reject or mistrust a website quickly. Content factors such as informative content, unbiased information, relevant illustrations, wide variety of topics covered were more important than design factors, such as clear layout, good navigation aids and interactive features when accepting and trusting a website. Content factors, such as source expertise and credibility (information from reputable organizations and authors were accepted, and information from pharmaceutical companies or websites explicitly selling products were rejected), accessibility and consistency (the information was expected to provide a good match and answers to the queries issued, and easily accessible), and social identification and personalization (sites written by people similar to the participants, aimed at people like the participants, with familiar language and highly relevant content) were the most important in terms of influencing the trust of a website. The risk information contained in websites including figures were found to be confusing and misleading. 

Phase 2: Internet usage and integration of information across different resources over a long-term period was evaluated. At the end of Phase 1, participants logged their ongoing health information and advice searches on both online and offline, in diaries over the subsequent 6 month period. The authors observed that only two-three sites in addition to a few new sites were visited for seeking advices during this time period; therefore, no factors which lead participants to stick to certain sites were identified using this diary information. However, facts related to the integration of material from different sources was evident from the diary information.   

Phase 3: Consumers' reflections on the value of online menopause health information on decision making and communication with physicians were evaluated; a telephone interview was conducted one month after completing phase 2. The authors observed that only a few  participants had repeated visits to the sites for further information and to obtain information to share with their friends, family and physicians. Participants reported that Internet was an important source of information in the early stages of decision making; the Internet was also used for investigating specific concerns, comparing sites and assessing risks. Online information was found to affect almost all participants' thinking and decision making behaviours.                     

In conclusion, according to these findings it is important to note that, if the relevant content to a person's information seeking is buried deeply in a website or if the website has a poor design, the information seeker will reject these resources earlier on in their search. This work also found that consumers are interested in seeking information and advice which will support their own viewpoints, because this will help them to build confidence when making decisions. They do so by accessing online information; however, they tended to integrate online and offline information and advice when making decisions. It was also found that online information is used to commence and improve discussions with friends and physicians. Note that only few consumers reported developing long-term trusting relationships with a particular site.     

\textbf{Limitations}

The sample size was relatively small (15 participants), and only a very few participants completed the diary information: therefore, this information was not enough to allow for conclusions.  Asking consumers to use specific websites may have influenced their decision making.


\section{Efthimiadis, E. N. (2009, January). How students search for consumer health information on the web. In System Sciences, 2009. HICSS'09. 42nd Hawaii International Conference on (pp. 1-8). IEEE.}

This study investigated how undergraduate and graduate students search the web for consumer health information. 32 students participated in the study. They were asked to find answers to four health related questions: 1. related to flu shot, 2. related to second-hand smoking, 3. related to major depression and 4. related to taking large doses of aspirin. Data related to students' demographic information, educational background, search experience on the Web, the searching process and behaviours while performing the tasks, decision making steps, including search strategies, website evaluation and question interpretations which were explained verbally, satisfaction with each task, and the reliability of the information sources were collected. the facts, such as the starting point of the search (default starting point for all participants was the Internet Explorer (IE) browser), query formulation and refinement, evaluation of the results, evaluation of the websites, the types of sources consulted and how the familiarity of the subject impacts on search experience and the satisfaction of the results were analysed and evaluated.

The authors analysed the results in different perspectives. Firstly, participants' characteristics were analysed. Undergraduates' age ranged from 19-27 and majority of them (87.5\%) conducted a search daily. Graduates' age ranged from 21-52 and 72.2\% of them conducted a search daily as well. 100\% of both groups were very familiar with web search engines. Graduates were more familiar with scientific databases compared to undergraduates. Both groups tended to search information for themselves and others. Secondly, searcher's first destination was analysed. All participants started the search task from the default Internet Explorer (IE) web browser and navigated to a web resource to commence their search (first search strategy destination choice). Four main categories of first search strategy destinations, such as commercial, government, not-for-profit organizations and University of Washington resources were identified. Majority from both groups started the search task using a search engine. The second most used starting point was directory-based search including both commercial and government directories for the graduates group, but the undergraduates did not use them much. The other first search strategy destinations included resources from the local University of Washington, not-for-profit professional organizations and government sites. Thirdly, search length and website visitations were analysed. Authors identified that the search length increased with the complexity of the question. Therefore, for the questions related to flu and smoke the average time taken was quite similar (5.4 and 4.4 mins), the time was twice as long for the question on depression (8.1 mins) and had a threefold increase for the question on aspirin (13.5 mins). Overall the number of websites and web pages visited increased with the complexity of the question. Therefore, searches on flu and smoke had the minimum number of website visitations (3.9 and 3.8) followed by searches on depression (6.1 mins) and aspirin (7.8 mins). Undergraduates visited more web pages, such as 50\% and 100\% for depression and aspirin related questions respectively indicating that the higher the complexity of the question, the more difficult it is for them to answer the question. Also, the number of multiple windows opened increased with the complexity of the question. Fourthly, query formulation and search terms were analysed. The average number of query formulations and search terms increased with the difficulty of the questions. Finally, familiarity with the topics and searcher satisfaction were analysed. Both groups were somewhat familiar with the topics flu and smoke. However, graduates had more knowledge on the topics depression and aspirin. Search planning was more easier for the topics flu and smoke, but was more challenging for the topics depression and aspirin for both groups. For both groups the first two topics were 'extremely' easy to search and the last two topics were not that easy. Undergraduates were 'extremely' satisfied with the search results and the majority of the graduates were satisfied, but were more conservative compared to undergraduates. Both groups were satisfied with the time it took to complete first two topics, but were concerned about the time it took to complete the last two topics. All undergraduates mentioned they found answers to all the questions, but 12.5\% of graduates mentioned they did not find answers to the last two questions. 25\% of the undergraduates mentioned they were not sure about the answers to last two questions and certain percentages of graduates mentioned they were not sure about the answers to all four questions ranging from 5.9\% for smoke to 25\% for aspirin. Undergraduates mentioned  the websites used for the first two questions as reliable and certain percentages of graduates were not sure about the reliability of the websites across all four questions.         

In conclusion, searching the web for consumer health information still remains as a challenging process, because many consumers have problems in query formulations. 

\section{Wallnofer, R., Rammer, T., Schabetsberger, T., Pfeiffer, K. P., \& Gobel, G. (2007). Use of the Vector Space Model for Expansion of Medical Queries. In Medinfo 2007: Proceedings of the 12th World Congress on Health (Medical) Informatics; Building Sustainable Health Systems (p. 2277). IOS Press.}

This study investigated the application of a vector space model for expansion of medical queries; the vector space model was used to do individual weighting of terms; the semantic relation between each term was extracted with the Main Headings from the medical thesaurus 'Medical Subject Headings' (MeSH); an acyclic graph was generated to represent all related terms from the thesaurus, for an issued search term; the distance to each extracted term was calculated; then the vector space model was used to calculate the similarity between the search term and the existing documents; the tool was developed to perform weighted query expansion based on MeSH terms; the authors observed that with this approach the users are able to retrieve documents which are most similar and related to the issued search terms, from a defined set of documents, by calculating the similarity of the documents with the use of Main Headings from the MeSH thesaurus on the basis of the vector space model.    

\section{PUSPITASARI, I., LEGASPI, R., \& NUMAO, M. (2013). Characterizing the Effect of Consumer Familiarity with Health Topics on Health Information Seeking Behavior. 人工知能学会全国大会論文集, 27, 1-5.}


This study investigated the effects of consumers' familiarity with health topic, on their health information seeking behaviour. 10 volunteers, including undergraduates, graduates and postdoctoral researcher of Osaka University participated in the study; all participants had prior experience in searching for medical information online. They reported that when searching online for health information, they mostly searched for information related to diseases and health problems.

Participants were asked to perform four health search tasks: 
\begin{itemize}
	\item task 1: an exploratory task where participants had to search for the reasons for 'experiencing intense painful swelling and redness in their right big toe';
	
	\item task 2: a specified task where participants had to search for the 'relation between rheumatoid arthritis and osteoporosis, how they affect each other, and which treatment of rheumatoid arthritis that would not lead to osteoporosis or at least decreases significantly the possibility of osteoporosis' in favour of their friend;
	
	\item task 3: a specified task where participants had to search for 'details about  hydrochlorothiazide and why it should not be taken together with decongestant drugs' as per their doctor's advice;
	
	\item task 4: a personalized task where participants had to search for 'two of generally recommended treatments for a health-related matter, disease or symptom that (1) interest them, or (2) become their concern'.
\end{itemize}

Participants were allowed to use any search engine, access any relevant website and to search at their own speed. Rated information regarding search performance, the cognitive effort required to complete the task, the difficulty and the familiarity of the task, several characteristics of search behaviour, such as query keywords, query reformulation patterns and explanations about search strategies used, and impressions on the search processes were collected. 


The authors observed that according to the ratings of the participants, task 2 and task 3 were significantly more difficult than task 1 or task 4, hence they have spent averagely 70\% of their time on completing task 2 and task 3. 
Participants were more significantly unfamiliar with health terminologies on task 2 and task 3. 
Self reported search performance for task 2 and task 3 were comparatively much lower (somewhat disappointed) than for task 1 and task 4 (satisfied). The cognitive effort needed for completing task 2 and task 3 was much higher than the effort needed to complete task 1 and task 4. 
Task 3 and task 2 had the most submitted queries; the formulation of the queries and the selection of the relevant results for both these tasks were difficult because of the unfamiliarity with the health terminologies on these two tasks. 
Task 2, task 1 and task 3 had the longest average query length; task 1's query length was longer than task 3 because participants used more stop-words in task 1's queries. 

All participants used the Google search engine as their starting point. The most number of results/ web pages were accessed for task 4 followed by task 2, task 1 and task 3. The issued query keywords were classified as general consumer terms and medical specified terminology; task 1 contained common consumer terms, and task 2 and task 3 contained medical specified terms. Query keywords issued  by the participants for task 1 contained common consumer terms whereas query keywords issued for task 2 and task 3 contained medical specified terms, because participants used keywords from the task descriptions. For task 4 participants used both common consumer terms and medical specified terms because the search was performed either based on their interest or concern. 
A medical vocabulary thesaurus, Medical Subject Headings (MeSH) was used to perform keywords classification; 24 out of 29 queries contained phrases from MeSH. 
Both syntactic and semantic changes were identified when performing query reformulations; different reformulation patterns, such as generalized (generalising the meaning of previous query), specified (specifying the meaning of previous query), parallel (modifying queries from one aspect of a concept to another), building block (combing concepts from the previous queries) and dynamic (performing inconsistent reformulation patterns from one pattern to another pattern) were identified. For task 1 only three participants reformulated their queries and the reformulation pattern being 'specified'. For the other tasks all participants reformulated their queries; the majority performed dynamic reformulation when completing task 2 and task 3; the majority performed parallel reformulation when completing task 4. 

In conclusion, the familiarity with the health topics affects consumers' searching behaviour. Participants used more specific and more varied vocabulary (query keywords selection) when performing more familiar tasks. Participants tended to perform different query reformulation patterns depending on their different familiarity level (different patterns for familiar and non-familiar tasks). For unfamiliar tasks they performed dynamic reformulation patterns (task 2 and task 3) by relying only on the task descriptions for formulating queries. Selection of the results was also more difficult to perform in unfamiliar tasks (task 2 and task 3). For familiar tasks they performed the parallel reformulation pattern (task 4).     


The importance of estimating consumers' familiarity with the health topics is because this information can be used by a health information retrieval system to provide better assistance and retrieve more understandable results. Therefore, a health information retrieval system can classify query keywords as common consumer terms and medical specified terms using Consumer Health Vocabulary and Medical Subject Headings, detect query reformulation patterns, including semantic changes using Unified Medical Language System (UMLS) Semantic Network and track the site selection in order to estimate consumers' familiarity with the health-related search tasks. Once familiarity is estimated, the system can provide query suggestions and site recommendations to lay people and personalize the search results. 	
	
\section{Inthiran, A., Alhashmi, S. M., \& Ahmed, P. K. (2016). Describing health querying behavior. In Proceedings of the 2nd SIGIR workshop on Medical Information Retrieval (MedIR).} 

This study investigated the querying behaviour of laypeople when searching for health information. The querying process of a whole search process is important because it represents the searchers' initial understanding of the health aspect they search for. The improvement of this understanding can also be understood by evaluating the query reformulation patterns. Therefore, the authors aimed at evaluating qualitative querying and reformulation patterns when performing health tasks with different levels of difficulty. 20 laypeople (undergraduate students, postgraduate students and university staff members) participated in the study. Searching was performed on MedlinePlus; two simulated situations were given to each participant; search session time was unlimited. Information, such as socio-demographic details, general search experience, health search experience were gathered via a pre-experiment interview. Clinical scenarios were used because laypeople tend to perform such health searches. Simulated situation 1 was related to searching reasons for experiencing kidney enlargement followed by urine retention and whether there are any alternative treatments or is surgery the only option available to treat this condition. Simulated situation 2 was related to searching treatment options for a swollen neck on the left side which cannot be moved left or right and the reasons for experiencing this condition. At the end of each task, participants were asked to rate their perception of the task difficulty via a post-experiment interview. Querying patterns were classified as informational directed (related to knowing about a particular topic) and informational undirected (related to knowing everything/ anything about a topic). Query reformulation patterns were analysed using semantic analysis methods.

For simulated situation 1, 5 participants rated it as easy, another 5 rated as neutral and 10 rated as difficult. For simulated situation 2, 15 rated it as easy, another 5 rated as neutral and 5 rated as difficult. When performing easy tasks, participants started search process with informational directed queries (searching answers for specific questions) and then moved to informational undirected queries (searching for broader aspects). When performing neutral tasks, participants started search process with informational undirected queries and moved to informational directed queries (from broader topics to more narrower aspects). When performing difficult tasks, participants started search process with informational directed queries, then moved to informational undirected queries and ended the search session with informational directed queries (no much focus). This behaviour was called as participants performing 'unsystematic movements'. The authors used a model based on a triangle to illustrate querying patterns. According to this model, broadening the query was represented by moving towards the base of triangle and narrowing the query was represented by moving towards the tip of the triangle. Two query reformulation patterns were demonstrated by the participants when performing easy tasks. One pattern was to start the reformulation strategy with switching topic and end it with specialization. The other pattern was to start the reformulation strategy with specialization and end it with parallel movement (previous and the new queries having partial overlaps). When performing neutral tasks, again two query reformulation patterns were demonstrated by the participants. One was to start with parallel movement and end with specialization. The other was using switching topics. When performing difficult tasks also, participants tended to use switching topics. Therefore, participants tended to use varying query reformulation strategies when performing much easier tasks and did not change the reformulation strategies (switching topic only) when performing more difficult tasks. 

In conclusion, based on these results, two recommendations were proposed, so as to aid laypeople to perform well in querying and reformulating (high querying efficacy). One recommendation is to 'develop an algorithm to detect and classify query and reformulation patterns' and the other is to 'provide personalized query recommendations'. As participants performed different querying patterns for tasks with different levels of difficulty, querying patterns are useful to detect task difficulty. Therefore, such an algorithm will be able to detect task difficulty based upon query patterns. In addition, when an algorithm detects a query or a reformulation pattern related to a difficult task, health domains can be used to provide relevant assisting features. Hence, if a switching topic pattern was detected, query suggestions could be provided to laypeople so as to help them in keeping the focus on the search goal. With the use of such query suggestions, laypeople will be able to issue efficient and relevant queries. Health domains can also be used to personalize assisting features according to the health search task performed.      

\section{Hu, R., Lu, K., \& Joo, S. (2013). Effects of topic familiarity and search skills on query reformulation behavior. Proceedings of the Association for Information Science and Technology, 50(1), 1-9.}

This study investigated the effect of topic familiarity and search skills on query reformulation when searching for health information. Four research questions were considered in this study; 1. 'Does users' topic familiarity influence their selection of query reformulation types?'; 2. 'Does users' search skills influence their selection of query reformulation types?'; 3. 'Does users' topic familiarity influence the frequencies of query reformulations in a session?'; 4. 'Is there a significant difference in the time spent on each type of query reformulation?'. An experimental IR system of health information was designed with a Google-like search interface. Six search topics were selected from the medical information database MEDLINE to be used in the user studies. 45 graduate students participated in the study. Users were instructed to perform two pre-experiment search tasks. First one was to 'provide a definition of a key biomedical term selected from each search topic'. The second one was to 'find the relationships between biomedical concepts and to provide an answer for each search topic'. The use of search system was not allowed while answering questions. Participants' demographic information, familiarity with the six topics and information about their major was collected. Each participant performed three tasks using health information search system. Data, such as user queries, timestamps of search submissions, document IDs viewed and the search task were collected. 135 search sessions were collected. These sessions were categorized into two groups by considering users' familiarity with the topic. Therefore, the two categories were expert (topic familiarity greater than 2) sessions (34 out of 135 sessions) and novice (topic familiarity less than or equal to 2) sessions (101 out of 135 sessions). 45 participants were divided into two groups based on their major. Therefore, the two groups were Library and Information Science (LIS) major (with more skills in searching information and included 28 out of 45 participants) and other major (17 out of 45 participants). Query reformulations were categorized into six types. 4 main facets were considered for the analysis. 

\begin{enumerate}
	\item Content changed: included 4 sub-facets; (i) specification: adding or using more specific terms to specify a query; (ii) generalization: replacing specific terms with general terms to generalize a query; (iii) parallel movement: shifting query terms to another aspect of the topic.
	\item Content unchanged: included 1 sub-fact; (i) synonym: replacing terms with more common terms which have the same meaning of the previous terms.
	\item  Format: format changes were performed, such as term variations and search operators (abbreviation, preposition, Boolean operators etc.).
	\item  Error: correcting typos and wrong formats in previous terms. 
\end{enumerate}


The authors observed that the average topic familiarity before performing searches was approximately 2 (from a nine-level Likert scale). More than 65\% of the participants occasionally search health information online and about 11\% of them never searched health information online. 112 out of 135 sessions had at least one query reformulation action. A total of 334 query reformulation actions were performed. The reformulation types 'Specification', 'Generalization' and 'Parallel movement' were performed in 78\% of the query reformulations. 'Format', 'Error' and 'Synonym' were performed in 16.8\%, 3.9\% and 0.3\% of the query reformulations respectively. Specification was the most frequently used query reformulation type and synonym was the least frequently used query reformulation type. During expert sessions the most frequently applied query reformulation types were 'Generalization' (22.4\%) and 'Format' (22.4\%). Users with higher topic familiarity tended to start the search process with specific terms and extended search results with the use of 'Generalization' reformulation type. During novice sessions the most frequently applied query reformulation types were 'Error' (4.1\%), 'Parallel movement' (25.6\%) and 'Specification' (35.3\%). This revealed that users with low topic knowledge are more prone to make mistakes (misspelling and typo) because of the lack of health topic related knowledge. They also started their search process with general terms and gradually had to move to more specific terms in order to obtain precise and relevant information. However, according to the statistical analysis there was no significant association between topic familiarity (domain knowledge) and the selection of query reformulation types. LIS major group more frequently applied 'Format'(17.7\%), 'Generalization' (21.7\%) and 'Specification' (35.9\%) during their search process. LIS major users tended to change the format of the queries very frequently so as to retrieve more precise search results and made comparatively less errors. The other major group more frequently applied 'Error' (7.4\%) and 'Parallel movement' (27.4\%) during their search process. Hence, this group was unable to issue efficient queries to obtain relevant results and was not good at moving from one aspect to another while searching. However, according to the statistical analysis there was no significant association between search skills and the selection of query reformulation types.  The average query reformulations per session was lower for the 'expert sessions' group (less than 2 query reformulations) compared to the 'novice sessions' group (approximately 2.7 query reformulations). Therefore, users with higher topic knowledge were able to complete search tasks with lesser effort in query reformulation. According to the statistical analysis a significant association between topic familiarity and the frequencies of query reformulations in a session was identified (users with higher topic familiarity applied fewer query reformulations). In terms of time spent on each type of query reformulation, users spent comparatively longer time when correcting queries. However, according to the statistical analysis there was no significant difference in the time spent on each type of query reformulation.   

In conclusion, topic familiarity has the ability to influence the frequencies of query reformulations and the time spent on each type of query reformulation. Users with higher topic familiarity made less spelling errors and preferred applying general terms to modify their queries. Therefore, they tended to issue more correct search statements, and initiated their search processes with more specific terms and gradually moved on to issuing more general terms to obtain broader search results (improve search results with higher recall rate). In addition, these users performed less number of query reformulations to complete their search tasks (more efficient). In contrast, users with lower topic familiarity were less likely to apply 'Format' and 'Generalization' query reformulation types when modifying queries. They initiated their search processes with more broader terms and gradually moved on to issuing more specific terms to obtain more precise search results. Because these users obtained irrelevant search results with their inadequate initial queries, they applied a 'Parallel reformulation' to move from one search aspect to another aspect. Therefore, such users had more difficulty in selecting proper search strategies when initiating a search process. As a result, they performed more number of query reformulations to complete their search tasks (less efficient). Users with higher search skills (LIS major) made less errors and applied 'Format', 'Generalization' or 'Specification' query reformulation types. 'Error' and 'Parallel movement' query reformulation types were not much used by these users. In contrast, users from other major group did not very frequently apply any of the 'Format', 'Generalization' or 'Specification' query reformulation types and made more mistakes when issuing search terms to the system. Therefore, users with higher search skills applied more various query reformulation types. Overall, users spent more time when conducting 'Error' query reformulation type because they had to correct terms from previous queries. Therefore, it is said that query expansion techniques, query suggestion techniques, controlled vocabularies and auto filling techniques can be used in IR systems to aid users with lower topic familiarity.                   

\textbf{Limitations}

Authors reported that 135 search sessions from 45 participants were not enough to generalize the findings. Topic familiarity was measured on the basis of user perception, so the information might not be reliable. Information about user search skills were not objectively measured for the study (although LIS major participants were assumed to have advanced search skills compared to other participants).    


	
\end{document}

