
\documentclass[]{article}
\usepackage{graphicx}

%opening
\title{A literature review on Consumer Health Search}
\author{Kaushi Perera}

\begin{document}
	
\maketitle
	
\textbf{Unit: IFN 701}
\textbf{Credit Points: 24 credit points}	
\textbf{Student ID:n9789511}	
\textbf{Student Name: D. F. Kaushalya S. Perera}	
\textbf{Project Supervisor: Dr. Guido Zuccon}	
\textbf{Project Coordinator: }		
	
\title{Abstract}

purpose, 
scope, 
research method for the Literature review, 
key outcomes and 
recommendations from the literature review completed

\title{Introduction}    

\textbf{background and/or context}

At present online resources are increasingly used by lay people to search different types of information. A significant portion of these searches are performed to retrieve health information. Hersh (2015) state that, the use of Information Retrieval systems by consumers to retrieve health information has become ubiquitous. For example, as Zuccon, Koopman and Palotti (2015) have highlighted in their paper, an analysis conducted using three web search engines, has been able to reveal that, approximately 10\% of user queries issued to these search engines are health related.  Hersh (2015) also mentions that, at present, most scientific papers are published electronically. These profound changes in publishing information is also identified as a main reason for this escalated use of online resources by consumers to seek health information.  

Consumers search health information online to obtain various types of information, such as medical information, details about health professionals, self-diagnosis information and to decide the efficacy of medical treatments (Pogacar et al., 2017). However, when lay people search for health information, there is a higher possibility of them retrieving inaccurate, irrelevant and unreliable information, because they are more likely to use lay terms to describe a medical condition. For instance, a lay person might issue a query, such as ‘my head is pounding’ except typing the proper medical term (cephalalgia) when searching for medical information. This issue is known as the ‘circumlocutory in medical queries’ (Stanton, Ieong \& Mishra, 2014). In addition, consumers also tend to express health conditions using different words which might make it hard to retrieve information containing words that matches those consumers’ words. As Croft, Metzler and Strohman (2010) mention, this is known as the ‘vocabulary mismatch problem’. In other words, even though the purpose of an information retrieval system is to retrieve information which satisfies users’ information needs, the information that is being retrieved highly depend on issued user queries. Consumers are highly likely to have language and knowledge gaps, such as not knowing adequate vocabulary related to health domain and might also be unfamiliar with most of the proper medical terms (Soldaini et al., 2016). 
   
Therefore, consumers’ increased use of online resources to retrieve health-related information can also increase the potential problems associated with the retrieval process, such as retrieving inaccurate, unreliable and useless information. For example, as Pogacar et al., (2017) state, users are highly likely to be influenced by inaccurate and unreliable information, when searching for the efficacy of medical treatments, and might end up having harmful impact on their own lives. Hence, it is crucial for health information search systems to satisfy users’ health information needs by retrieving authoritative, accurate and useful information.  
   
Because of the importance of this field, researchers have focused more on to improving health information retrieval systems as a means of making sure that these retrieval systems are able to satisfy consumers’ information needs. As a result, researches have been conducted to cover various aspects, such as user search behaviours and retrieval models of consumer health information searches (Toms \& Latter, 2007 and Jimmy, Zuccon \& Koopman, 2018).    

\textbf{aims, objectives and the anticipated significance of this literature review project}

The aim of this literature analysis is to contribute to the various aspects of consumer health information searches, investigated in recent researches, by synthesizing this information and presenting it as a literature analysis. Therefore, this literature analysis aims to cover various aspects, such as user search behaviours, consumer health information retrieval models, strategies to improve user queries and different problems faced by consumers when searching for health information online. This synthesized knowledge and information then will be reported as an analysis of different aspects covered in each research paper, methodologies which are used for each study and their corresponding results in a well-structured literature review paper. Prior preparing the final literature analysis, this project also prepares an annotated bibliography of all the research papers covered and a table of topics summarising the claims of each research paper. Therefore, this analysis is really useful for health information search system designers because they can find heaps of useful information and tips which can be embedded when designing such systems. In addition, this information is also important for researchers who are interested in conducting researches on ‘consumer health information search’, because this literature analysis highlights any uncovered aspects of consumer health information searches. In other words, this literature analysis highlights areas which are pertinent for future studies.   
    

\textbf{a brief overview of the method(s) have applied }

The methodology of this literature analysis consists of a few main steps.

Step 1: Defining a protocol for the literature analysis. The main purpose of defining a protocol is to retrieve more relevant and useful research papers to conduct this literature analysis 

Step 2: Reading and analysing the chosen research papers to gather knowledge and information related to consumer health information searches including:

(i)	User search behaviours
(ii) Retrieval models
(iii) Strategies used to improve user queries
(iv) Problems faced by consumers when searching for health information online

Step 3: Preparing an annotated bibliography based on all the covered research papers

Step 4: Preparing a table of topics summarising the claims in each research paper

Step 5: Preparing the literature analysis by highlighting different topics (aspects), methods and findings covered in each research paper. In addition, any uncovered aspects of consumer health information searches will also be highlighted in this literature analysis

More details on the methodology which has been applied to conduct this literature analysis is presenter under the section 'Literature Review Methodology'. 

\textbf{Recap the scope of the project}

This literature review only focuses on recent research papers which were published 2005 onwards. The knowledge and information in these research papers also need to contribute well to at least one of the main aspects of consumer health information searches as listed below.
1.	User search behaviours
2.	Consumer health retrieval models
3.	Strategies to improve user queries 
4.	Problems faced by consumers when searching for health information  

\textbf{Briefly summarise the key deliverables (results and outcomes}

This literature analysis consists of a few main deliverables.

1.	An annotated bibliography: This will include an annotation for each of the chosen research paper. Each annotation will highlight key factors, such as the aim of the study, the methodology which has been followed, key findings and the importance of those findings as presented in each research paper.  

2.	A table of topics and claims: This table will summarise different topics and claims covered in each of the chosen research paper. For example, an excel spreadsheet will be used to note down different aspects, such as user search behaviours and retrieval models covered in each research paper with their claims/ key findings. Then these findings will be analysed to identify patterns within each aspect.  

3.	A presentation highlighting the key findings of the analysis: The final presentation which was prepared and presented in week 12, included information, such as different aspects of consumer health information search covered in each research paper, the key findings of each research paper and the importance of each of the finding.  

4.	A literature review paper which includes a thorough analysis of the chosen research papers: The final literature review paper consists of an analysis which includes, different aspects of consumer health information searches covered in each chosen research paper and their corresponding findings. Hence, this literature analysis will be a synthesis of the key findings of each research paper depending on which aspect of consumer health information search it covers. In addition, this literature analysis will also highlight any uncovered aspects of consumer health information search by the chosen research papers.

\title{Literature Review Methodology}  

The methodology of this literature analysis consists of a few main steps.

Step 1: Defining a protocol for the literature analysis

The main target of this step is to define a protocol (Knopf, 2006) for this literature analysis, which will be used to search and determine the relevance and usefulness of research papers. The relevance and usefulness of research papers will be determined based on a few main factors which have been described below. The researchers paper which are identified as highly relevant and useful will be chosen to conduct this literature analysis. 

(1)	The keywords/phrases and other techniques used to search for relevant and useful research papers 

a. Consumer health search
b. Consumer health information searching behaviour
c. Retrieval models for consumer health search
d. Strategies for improving user queries 
e. Problems for consumers when searching health information online

The use of proper keywords or phrases for searching relevant and useful research papers is important because the retrieval of research papers via search engines is highly impacted by those keywords and phrases. Hence, all the keywords and phrases should be carefully chosen to search for relevant and useful research papers. In addition to executing these queries, the links presented in Google Scholar, such as ‘Related articles’ and ‘Cited by’ will also be used to retrieve relevant and useful research papers.   

(2)	Inclusion and exclusion criteria 

A few factors are taken into account when determining the usefulness and relevance of a research paper. The major area which will be covered by this literature analysis is ‘Consumer Health Search’. Hence, this is a broader topic, this literature analysis specifically aims at reviewing research papers which cover at least one of the main aspects, such as user search behaviours, retrieval models, strategies for improving user queries or problems for consumers when searching health information online of consumer health information search. Hence, all the research papers which are chosen for this literature analysis must be directly relevant to at least one of the previously mentioned aspects. In addition, all the chosen research papers for this literature analysis must be recent research papers. Therefore, this literature review will only analyse research papers which were published 2005 onwards. Moreover, all the chosen research papers must also be academic and peer-reviewed research papers. 

(3)	Search services used

Mainly two services which are the Google Scholar and QUT library databases will be used in this literature review project to retrieve relevant and useful research papers. The main reasons for choosing these search services to search for useful and relevant research papers are because these services are able to retrieve information sources which contain highly reliable (academic and peer-reviewed) and complete information on prior research work.   

Step 2: Searching for relevant research papers by issuing the queries and analysing those research papers  

The main goal of this step is to issue queries as defined in step 1 or use links, such as ‘Related articles’ and ‘Cited by’ on Google Scholar website to retrieve recent research papers that contain information on at least of the aspects, such as user search behaviours, retrieval models, strategies for improving use queries and problems for consumers when searching for health information online. The next step is to read and analyse the chosen research papers to gather knowledge and information they have contributed to different aspects of consumer health information search as previously mentioned. It is crucial to analyse and understand existing research work related to consumer health search, because the foundation of this literature analysis is formed based on prior related research work. In addition, literature analysis also aims at covering approximately five to ten related research papers per week, depending on the length of each research paper.             

Step 3: Preparing an annotated bibliography based on all the covered research papers 

In order to analyse all the chosen research papers an annotated bibliography (The Writing Center, 2018) will be prepared and used as the analysis technique for this literature review. The tool TeXstudio which is a LaTeX editor will be used to prepare this annotated bibliography. An annotation will be written for each of the research paper covered in this literature analysis and then will be included in the final annotated bibliography. Each annotation will include factors, such as the aim of the study, the methodology which has been used, key findings of the study and the importance of those findings as presented in each research paper. 

Step 4: Preparing a table of topics summarising the claims in each research paper

At this step, this project aims at preparing an excel spreadsheet as a table of topics using the annotations written for each of the research paper in step 3. Hence, this table of topics will include all the aspects (user search behaviours and retrieval models) covered in each of the chosen research paper, as a categorization of the themes and summaries of their claims. Most importantly these findings then will be analysed to identify any patterns within each aspect. Therefore, this categorization and the analysis of patterns will be the basis of the literature review conducted as the final step of this project.  Hence, this table of topics which summarises the claims of each research paper is really important because the content of final literature review will depend on the knowledge and information gained in this analysis. 

Step 5: Preparing the final literature review by synthesizing all the covered topics (aspects), methods, findings in each research paper, and by highlighting any uncovered aspects

This is the final and the main step of this literature review project. In this step a literature review will be prepared with the use of knowledge and information gained in the thorough analysis which was conducted in the previous step. Therefore, the final literature analysis will most importantly present any identified patterns within each covered aspect by highlighting the corresponding findings from each research paper. Hence, this literature analysis can also be seen as a synthesis of the key findings of each research paper which are categorized according to their common features. In addition, this literature review will also emphasize any uncovered aspects of consumer health information search by the chosen research papers. The final literature analysis will be prepared and presented by follow the literature review guidelines presented in the paper ‘Writing Integrative Literature Reviews: Guidelines and Examples’ (Torraco, 2005). Moreover, the tool TeXstudio which is a LaTeX editor will also be used to prepare the literature review and the final report.

Project Management Methodology 

This literature review project will use Dynamic Systems Development Method (DSDM) as its project management methodology. The main reason for using this project management approach is because DSDM is an agile project management approach (The DSDM Agile Project Framework, 2014). With the use of this project management approach, the main deliverables of the project, such as the annotated bibliography, the table of topics and the final literature analysis can be implemented as increments. Most importantly, this methodology provides. room to obtain continuous feedback from the supervisor and to identify any potential issues or risks associated with the project before it is too late. Therefore, it is possible to fix any identified issue as soon as they have been identified and then continue with the rest of the project. In addition, in a situation where there is a possibility of timeline slippage, it is possible to adjust the number of research papers read in each week, so as to align project tasks with their deadlines. For example, if time does not permit to read at least five papers in one week, it is possible to read 4 papers in that week and read an extra paper in the following week. Furthermore, another importance of adhering to DSDM as an agile project management approach is its assurance to deliver the project with all the essential requirements in it. Hence, the use of an agile project management methodology, such as DSDM will guarantee a high-level of quality for the final deliverables and will also assist in delivering them on time. 




  



\end{document}

