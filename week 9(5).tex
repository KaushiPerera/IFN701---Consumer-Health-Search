
\documentclass[]{article}
\usepackage{graphicx}

%opening
\title{Analysis 5}
\author{Kaushi Perera}

\begin{document}
	
\maketitle
	
\textbf{The use of different methods to evaluate the relevance of health-related search results}

Retrieved results for a health search task are evaluated in order to determine whether those results are relevant to what a user has searched for. Researchers have used different methods in their studies to determine the relevance of retrieved search results. (A) recruited human assessors to evaluate the relevance of retrieved search results. According these assessors, documents which contained information focused on the symptoms searched by users, and descriptions and causes of those symptoms aided by photographic material were considered as ‘Highly relevant’. Therefore, they have observed that, if users seek for highly relevant information, only 3 out of top 10 results on average are highly useful for self-diagnosis purposes. In addition, only about 4 to 5 results out of the top 10 results, which contained little information that was helpful to people, were considered as relevant (provide useful information to people in order to self-diagnose themselves). Furthermore, documents which contained information not only about relevant symptoms but also about other symptoms were considered as ‘somewhat relevant’. Moreover, ‘on topic but unreliable’ documents were considered as irrelevant in this evaluation. (B) have used a Decision Tree approach in their study to categorize retrieved health search results as relevant or irrelevant based on users' queries. According to their observations, the method with feature selection (selected the most important features which were strongly related to relevance) performs well and has a higher accuracy when determining the relevance of the search results compared to the method without feature selection. Therefore, they concluded that, this feature selection model can be used to filter out irrelevant documents prior presenting search results to consumers, so as to aid them in retrieving more relevant health related documents.     

   


\end{document}








