\documentclass[]{article}
\usepackage{graphicx}

%opening
\title{Annotated Bibliography}
\author{Kaushi Perera}

\begin{document}
	
\maketitle
	
\section{Sillence, E., Briggs, P., Harris, P. R., \& Fishwick, L. (2007). How do patients evaluate and make use of online health information?. Social science \& medicine, 64(9), 1853-1862.} 

This study investigated how menopausal women search the Internet for information and advice on hormone replacement therapy (HRT) and how they determine which sites/information to trust; a staged model of trust development was used for the evaluation; this model represents the way users firstly, assess the sites by screening them (heuristic stage/ initial searching process), then move on to in-depth evaluation of available information (analytic stage) and finally how they develop a long-term trusting relationship with a few particular sites. 15 menopausal women who were interested in assessing the costs and benefits of taking hormone replacement therapy (HRT), participated in the study; how the Internet advice influenced these consumers' decision making and communication with physicians were evaluated. Three phases of research involved and their corresponding results are as follows;  

Phase 1: Online search behaviours were evaluated; 4 weekly sessions were conducted; during each session, consumers searched the Internet for information and advice regarding menopause and had a discussion with a facilitator; 2 searches were freely performed on the Web and the other two were performed in specific websites; these specific websites represented a range of information providers, different content and design features to evaluate the impact of them on trust of the sources. The authors observed that, participants efficiently recognized and rejected web pages that acted as starting points or gateways to other sites, websites unrelated to menopause, websites which were not menopause specific, websites with broken links and sales sites. According to the discussions, design factors, such as complex layout, too much text, corporate look and feel etc. and content factors, such as irrelevant material and inappropriate material were the main reasons to reject or mistrust a website quickly; content factors, such as informative content, unbiased information, relevant illustrations, wide variety of topics covered were more important than design factors, such as clear layout, good navigation aids and interactive features when accepting and trusting a website; content factors, such as source expertise and credibility (information from reputable organizations and authors were accepted, and information from pharmaceutical companies or websites explicitly selling products were rejected), accessibility and consistency (the information was expected to provide a good match and answers to the queries issued, and easily accessible), and social identification and personalization (sites written by people similar to the participants, aimed at people like the participants, with familiar language and highly relevant content) were the most important in terms of influencing the trust of a website; the risk information contained in websites including figures were found to be confusing and misleading. 

Phase 2: Internet usage and integration of information across different resources over a long-term period was evaluated; at the end of Phase 1, participants logged their ongoing health information and advice searches on both online and offline, in diaries over the subsequent 6 month period. The authors observed that only a two-three sites with a few new sites were visited for seeking advices during this time period; therefore, no much factors which lead participants to stick to certain sites were identified using this diary information; however, facts related to the integration of material from different sources was evident from the diary information.   

Phase 3: Consumers' reflections on the value of online menopause health information on decision making and communication with physicians were evaluated; a telephone interview was conducted one month after completing phase 2. The authors observed that only a few of the participants had repeated visits to the sites for further information and to obtain information to share with their friends, family and physicians; the Internet was said to be really important in the early stages of decision making; the Internet was also used for investigating specific concerns, comparing sites and assessing risks; online information was able to affect almost all participants' thinking and decision making behaviours.                     

In conclusion, according to these findings it is important to note that, if related content of a topic is buried deeply in a website or if the website has a poor design, such sites will be rejected earlier by the consumers;  consumers are interested in seeking information and advice which will support their own viewpoints, because it will help them to build confidence when making decisions; online information is used to commence and improve discussions with friends and physicians; the Internet is used as a source of evidence by the consumers to support their choices; only a very few consumers are tended to develop long-term trusting relationships with particular sites; however, consumers are tended to integrate online and offline information and advice when making decisions.     

\textbf{Limitations}

The sample size was relatively small; asking consumers to use specific websites may have influenced their decision making; relatively only a very few participants completed the diary information and therefore, this information was not enough to make conclusions.  


\section{Efthimiadis, E. N. (2009, January). How students search for consumer health information on the web. In System Sciences, 2009. HICSS'09. 42nd Hawaii International Conference on (pp. 1-8). IEEE.}

This study investigated how undergraduate and graduate students search the web for consumer health information; 32 students participated in the study; they were asked to find answers to four health related questions; 1. related to flu shot; 2. related to second-hand smoking; 3. related to major depression; 4. related to taking large doses of aspirin; data related to students' demographic information, educational background, search experience on the Web, the searching process and behaviours while performing the tasks, decision making steps, including search strategies, website evaluation and question interpretations which were explained verbally, satisfaction with each task, and the reliability of the information sources were collected; the facts such as the starting point of the search (default starting point for all participants was the Internet Explorer (IE) browser), query formulation and refinement, evaluation of the results, evaluation of the websites, the types of sources consulted and how the familiarity of the subject impacts on search experience and the satisfaction of the results were analysed and evaluated.

The authors analysed the results in different perspectives; 1. participants' characteristics: undergraduates' age ranged from 19-27 and majority of them (87.5\%) conducted a search daily; graduates' age ranged from 21-52 and 72.2\% of them conducted a search daily as well; 100\% of both groups were very familiar with web search engines; graduates were more familiar with scientific databases compared to undergraduates; both groups were tend to search information for themselves and others; 2. searcher's first destination: all participants started the search task from the default Internet Explorer (IE) web browser and navigated to a web resource to commence their search (first search strategy destination choice); four main categories of first search strategy destinations, such as commercial, government, not-for-profit organizations and University of Washington resources were identified; majority from both groups started the search task using a search engine; the second most used starting point was directory-based search including both commercial and government directories for the graduates group, but the undergraduates have not used them much; the other first search strategy destinations included resources from the local University of Washington, not-for-profit professional organizations and government sites; 3. search length and website visitations: authors identified that the search length increased with the complexity of the question; therefore, for the questions related to flu and smoke the average time taken was quite similar (5.4 and 4.4 mins), the time was twice as long for the question on depression (8.1 mins) and had a threefold increase for the question on aspirin (13.5 mins); overall the number of websites and web pages visited increased with the complexity of the question; therefore, searches on flu and smoke had the minimum number of website visitations (3.9 and 3.8) followed by searches on depression (6.1 mins) and aspirin (7.8 mins); undergraduates visited more web pages, such as 50\% and 100\% for depression and aspirin related questions respectively indicating that the higher the complexity of the question gets the more difficult it is for them to answer the question; also the number of multiple windows opened increased with the complexity of the question; 4. query formulation and search terms: the average number of query formulations and search terms increased with the difficulty of the questions; 5. familiarity with the topics and searcher satisfaction: both groups were somewhat familiar with the topics flu and smoke; however, graduates had more knowledge on the topics depression and aspirin; search planning was more easier for the topics flu and smoke, but was more challenging for the topics depression and aspirin for both groups; for both groups the first two topics were 'extremely' easy to search and the last two topics were not that easy; undergraduates were 'extremely' satisfied with the search results and the majority of the graduates were satisfied, but were more conservative compared to undergraduates; both groups were satisfied with the time it took to complete first two topics, but were concerned about the time it took to complete the last two topics; all undergraduates mentioned they found answers to all the questions, but 12.5\% of graduates mentioned they did not find answers to the last two questions; 25\% of the undergraduates mentioned they were not sure about the answers to last two questions and certain percentages of graduates mentioned they were not sure about the answers to all four questions ranging from 5.9\% for smoke to 25\% for aspirin; undergraduates mentioned  the websites used for the first two questions as reliable and certain percentages of graduates were not sure about the reliability of the websites across all four questions.         

In conclusion, searching the web for consumer health information still remains as a challenging process, because many consumers have problems in query formulations. 

\section{Wallnofer, R., Rammer, T., Schabetsberger, T., Pfeiffer, K. P., \& Gobel, G. (2007). Use of the Vector Space Model for Expansion of Medical Queries. In Medinfo 2007: Proceedings of the 12th World Congress on Health (Medical) Informatics; Building Sustainable Health Systems (p. 2277). IOS Press.}

This study investigated the application of a vector space model for expansion of medical queries; the vector space model was used to do individual weighting of terms; the semantic relation between each term was extracted with the Main Headings from the medical thesaurus 'Medical Subject Headings' (MeSH); an acyclic graph was generated to represent all related terms from the thesaurus, for an issued search term; the distance to each extracted term was calculated; then the vector space model was used to calculate the similarity between the search term and the existing documents; the tool was developed to perform weighted query expansion based on MeSH terms; the authors observed that with this approach the users are able to retrieve documents which are most similar and related to the issued search terms, from a defined set of documents, by calculating the similarity of the documents with the use of Main Headings from the MeSH thesaurus on the basis of the vector space model.    

\section{PUSPITASARI, I., LEGASPI, R., \& NUMAO, M. (2013). Characterizing the Effect of Consumer Familiarity with Health Topics on Health Information Seeking Behavior. 人工知能学会全国大会論文集, 27, 1-5.}


This study investigated the effects of consumers' familiarity with health topics, on their health information seeking behaviour; 10 volunteers, including undergraduates, graduates and postdoctoral researcher of Osaka University participated in the study; all participants had prior experience in searching for medical information online; the most searched information was related to diseases and health problems; participants were asked to perform four health search tasks; task 1: an exploratory task where participants had to search for the reasons for 'experiencing intense painful swelling and redness in their right big toe'; task 2: a specified task where participants had to search for the 'relation between rheumatoid arthritis and osteoporosis, how they affect each other, and which treatment of rheumatoid arthritis that would not lead to osteoporosis or at least decreases significantly the possibility of osteoporosis' in favour of their friend; task 3: a specified task where participants had to search for 'details about  hydrochlorothiazide and why it should not be taken together with decongestant drugs' as per their doctor's advice; task 4: a personalized task where participants had to search for 'two of generally recommended treatments for a health-related matter, disease or symptom that (1) interest them, or (2) become their concern'; participants were allowed to use any search engine, access any relevant websites and to search at their own speed; rated information regarding searching performance, the cognitive effort required to complete the task, the difficulty and the familiarity of the task, several characteristics of search behaviour, such as query keywords, query reformulation patterns and explanations about search strategies used, and impressions on the search processes were collected. 

The authors observed that according to the ratings of the participants, task 2 and task 3 were significantly more difficult than task 1 or task 4, hence they have spent averagely 70\% of their time on completing task 2 and task 3; participants were more significantly unfamiliar with health terminologies on task 2 and task 3; searching performance for task 2 and task 3 were comparatively much lower (somewhat disappointed) than for task 1 and task 4 (satisfied); cognitive effort needed for completing task 2 and task 3 were much higher than the effort needed to complete task 1 and task 4; task 3 and task 2 had the most submitted queries respectively; the formulation of the queries and the selection of the relevant results for both these tasks were difficult because of the unfamiliarity with the health terminologies on these two tasks; task 2, task 1 and task 3 had the longest average query length respectively; the task 1's query length was longer than task 3 because participants used more stop-words in task 1's queries; all participants used Google search engine as the starting point; the most number of results/ web pages were accessed for task 4 followed by task 2, task 1 and task 3 respectively; the issued query keywords were classified as general consumer terms and medical specified terminology; the task 1 of this study contained common consumer terms, and task 2 and task 3 contained medical specified terms; query keywords issued  by the participants for task 1 contained common consumer terms whereas query keywords issued for task 2 and task 3 contained medical specified terms, because participants used keywords from the task descriptions; for task 4 participants used both common consumer terms and medical specified terms because the search was performed either based on their interest or concern; a medical vocabulary thesaurus, Medical Subject Headings (MeSH) was used to perform keywords classification; 24 out of 29 queries, contained phrases from MeSH database; both syntactic and semantic changes were identified when performing query reformulations; different reformulation patterns, such as generalized (generalising the meaning of previous query), specified (specifying the meaning of previous query), parallel (modifying queries from one aspect of a concept to another), building block (combing concepts from the previous queries) and dynamic (performing inconsistent reformulation patterns from one pattern to another pattern) were identified; for task 1 only three participants reformulated their queries and the reformulation pattern being 'specified'; for other tasks all participants reformulated their queries; the majority performed dynamic reformulation when completing task 2 and task 3; the majority performed parallel reformulation when completing task 4. 

In conclusion, the familiarity with the health topics affects consumers' searching behaviour; participants used more specific and more varied vocabulary (query keywords selection) when performing more familiar tasks; the participants were tended to perform different query reformulation patterns depending on their different familiarity levels of the tasks (different patterns for familiar and non-familiar tasks); for unfamiliar tasks they performed dynamic reformulation patterns (task 2 and task 3) by relying only on the task descriptions for formulating queries; selection of the results were also more difficult perform in unfamiliar tasks (task 2 and task 3); for familiar tasks they performed parallel reformulation pattern (task 4).     


The importance of estimating consumers' familiarity with the health topics is that, this information can be used by a health information retrieval system to provide better assistance and retrieve more understandable results; therefore, a health information retrieval system can classify query keywords as common consumer terms and medical specified terms using Consumer Health Vocabulary and Medical Subject Headings, detect query reformulation patterns, including semantic changes using Unified Medical Language System (UMLS) Semantic Network and track the site selection in order to estimate consumers' familiarity with the health-related search tasks; once the familiarity is estimated, the system can provide query suggestions and site recommendations to lay people and personalize the search results.        



\end{document} 