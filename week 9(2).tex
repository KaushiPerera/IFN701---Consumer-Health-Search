
\documentclass[]{article}
\usepackage{graphicx}

%opening
\title{Analysis 2}
\author{Kaushi Perera}

\begin{document}
	
\maketitle
	
\textbf{Query improvements}


How query recommendations, clarifications, suggestions and expansions impact health information searches


The first aspect of query improvements is how query recommendations/ clarifications impact health information searches. The different ways in which query recommendations and clarifications impact health searches have been identified based on four different studies. According to (A), the use of query recommendations, increases the number of successful queries and query suggestions provided by different systems are beneficial for consumers when searching health related information. This fact has been supported by (C) mentioning that, query suggestions lead to more successful searches.  However, (A) also state that, the impact of query recommendations on successful completion of a predefined health search task and the overall user satisfaction is not significant. The importance of query clarifications when performing health information searches was highlighted by (B) stating that, the number of correctly answered medical questions increases with the trustworthiness of the web pages used to answer the questions and with the use of clarified queries, consumers are able to retrieve more trustworthy resources as results. According to (B), implicit query clarifications which are performed on websites without the involvement of users, are also highly useful and it does not require users to be aware of the correct medical terminologies added to user queries while clarifying them. In addition, (C) argue that, multilingual and multi-terminology suggestions are also useful to retrieve more relevant documents. (D) investigated how query expansion aids laypeople in improving initial queries and the overall retrieval performance. However, they have not been able to make clear conclusions because of the lack of information about relevance and readability assessment of the test collection they have used to evaluate retrieved documents. 

The impact of different synonym mappings on query clarification

The second aspect of query improvements is the impact of different synonym mappings on query clarification. The observations are presented based on one study which utilized three main synonym mappings for the analysis. According to (B), Behavioural synonym mappings which mapped user expressions to medical symptoms is the best performing synonym mapping compared to MedSyn mappings (mappings focused on diseases and symptoms, and removed terms which were not related to UMLS semantic types) and DBpedia  mappings (mapping lay terms to expert terminology based on Wikipedia redirect pages), because users are more likely to correctly answer questions with the use of results retrieved using queries clarified with Behavioural mapping and Behavioural mapping is the least preferred mapping for incorrectly answered questions. In addition, (B) also state that, 'simple multinomial logistic regression classifier' (an ensemble of three query clarification methods and no query clarification method) is the best performing query clarification method compared to all individual synonym mappings and any unclarified queries. Furthermore, according to (B)'s observations, the clarification process can be further improved by selecting the most appropriate query clarification method (synonym mapping) to clarify each query.



Consumers' preference for query recommendations, clarifications and suggestions when performing health information search tasks 

The final aspect of query improvements is consumers' preference for query recommendations, clarifications and suggestions when performing health information search tasks. Consumers' different preferences are presented based on two different investigations. According to (B), comparatively lay people prefer query clarifications more than experts because lay people are able to retrieve  more useful information to successfully complete a health-related search. (C) identified consumers' higher preference for query suggestions and they have mentioned this as good acceptance of health query suggestions from the tool (Google Chrome extension) was observed. In addition, (B) have noted that there is a strong correlation of the success rate of each health information search task between lay people and experts because some tasks were more difficult for both groups when compared to other health information search tasks, no matter which query clarification method has been used. However, (B) also have identified that, the difference in success rates between lay people and experts is not significant. 


\end{document} 


