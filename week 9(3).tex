
\documentclass[]{article}
\usepackage{graphicx}

%opening
\title{Analysis 3}
\author{Kaushi Perera}

\begin{document}
	
\maketitle
	
\textbf{The use of different methods to enhance consumers' understandability of health information and retrieval models}


The importance of methods which enhance consumers' understandability of health information 


Studies have been conducted in order to investigate the importance of different methods which enhance consumers' understandability of health information. The observations are presented based on both system side approaches and user side approaches which have been used to enhance consumers' understandability of health information. According to (A), the use of supportive information to thoroughly explain a health concept (contextualization of health information), significantly increases lay users' understanding of health information by raising users' health literacy and reducing communication complexity. In addition, (A) mention that, users' cognitive style, including the way they organize, filter, transform and process information is also a significant factor in thoroughly understanding health content. Apart of that, as (A) mention, consumers also use their own contexts, such as social context and psychological context as a support for understanding health information well.        


The impact of retrieval models on consumer health information search 


Studies have also been conducted in order to investigate the impact of retrieval models on consumer health information search. The observations are presented based on a study which was conducted to investigate the performance of two different knowledge bases. According to B's observations, UMLS Knowledge Base has the ability to effectively expand more queries and retrieve a better set of documents compared to Wikipedia Knowledge Base. In addition, (B) also state that, relevance feedback has the ability to improve the effectiveness of these query expansions.        

   

\end{document}





