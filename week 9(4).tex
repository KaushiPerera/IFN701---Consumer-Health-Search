
\documentclass[]{article}
\usepackage{graphicx}

%opening
\title{Analysis 4}
\author{Kaushi Perera}

\begin{document}
	
\maketitle
	
\textbf{Current web search engines and problems when searching health information online}

Problems faced by consumers when searching for self-diagnosis information, symptoms and efficacy of health treatments


Researchers have conducted studies in order to understand different problems faced by lay people when searching for health related information, as a means of improving health information search systems. The problems faced by consumers when searching for self-diagnosis information, symptoms and treatments are presented in this section based on four studies. According to (A)'s observations, consumers face different problems when using self-diagnosis websites' information to diagnose symptoms. They state these problems as experiencing anxiety after reading inaccurate or false diagnosis information about serious conditions, unreliability of available information because there is no involvement of a health professional, the possibility of making users buy prescribed drugs from Internet pharmacies, and the potential to ignore professional assistance and diagnosis by believing false reassurance provided by self-diagnosis websites. This has also been confirmed by (C) stating that, there is a higher possibility of users retrieving misleading advice and irrelevant information when searching for self-diagnosis information which will then lead to erroneous self-diagnosis decisions and ultimately cause harm. One of the main reasons for users obtaining irrelevant health information has also been highlighted by (C) mentioning that, current retrieval techniques are not able to retrieve relevant health information when queries which describe symptoms in a circumlocutory manner are issued. Therefore, (A) have concluded that information available on self-diagnosis websites should be well structured, reliable, evidence-based and presented in lay language. In addition to problems faced by consumers when searching for self-diagnosis information and symptoms, according to (D), consumers also face similar problems when searching and determining the efficacy of health treatments. They have mainly observed that users' decisions are highly dependent upon search results' bias. Therefore, when search results are bias towards correct information, users were able to accurately determine the efficacy of health treatments. However, when search results were bias towards incorrect information users’ made more inaccurate and harmful decisions which were even worse than the decisions made without any search results. Other than that, users who were less confident with the efficacy of the treatments labelled them as inconclusive. (D) also claim that the impact of search results bias on the fraction of correct decisions and harmful decisions is statistically significant. A few of the main reasons for users making inaccurate decisions regarding the efficacy of health treatments were identified. They are, users less likely behaviour to label a health treatment as unhelpful because they are willing to see positive information in search results, users' prior health knowledge which make them think that most of the medical treatments are useful, users' bias towards clicking on the top ranked (rank 1) search result and the lower number of search results interacted by users. Therefore, they have concluded that, search results significantly affect people’s decisions about the efficacy of medical treatments and have a great potential to both help and harm users. Furthermore, as (B) have observed, providing supportive information (contextualization) has the ability to reduce communication complexity and increase users' understanding of different health concepts. However, they have also identified that there is a gap between this additional/ supportive information, which is known as 'informational context', provided to users by health information websites and users' own personal contexts, such as social context and psychological context which are used as a support to understand health information well.     



(TABLE FOR PROBLEMS FACED FOR SELF-DIAGNOSIS INFO AND EFFICACY OF HEALTH TREATMENTS)



\end{document}







