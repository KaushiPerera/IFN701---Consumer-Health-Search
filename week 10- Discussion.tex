
\documentclass[]{article}
\usepackage{graphicx}

%opening
\title{Discussion}
\author{Kaushi Perera}

\begin{document}
	
\maketitle

This section critically analyses the significance of this literature review, key findings, limitations associated with this literature reivew, and key recommendations and steps that could be taken to further develop this work. The anticipated significance of this literature analysis was to contribute to the existing various aspects of consumer health search, investigated in recent studies, by synthesizing this existing information and presenting it as a literature analysis. The different consumer health search aspects which were considered in this literature analysis, include user search behaviours, strategies used to improve user queries, methods used to enhance consumer understandability of health information, retrieval models, problems faced by consumers when searching health information online, methods used to evaluate the relevance of health search results and techniques used to map user queries to medical symptoms. In addition to different aspects of consumer health search, this analysis has also highlighted several issues which are uncovered by the considered set of studies.   


key findings:

The key findings of this literature analysis is presented with respect to the seven main aspects of consumer health search which were identified by analysing ........ recent research papers. 

The first key aspect of consumer health search identified was user search behaviours. According to the findings, the complexity or difficulty of a consumer health search influences the overall health search process, including session length, number of web pages visited, interaction strategies, search patterns, query reformulation patterns and search planning. When considering the purposes of consumer health searches, it was identified that, in addition to searching information about diseases, symptoms, treatments etc., users also search for confirming and novel information, and some of this information is also used by consumers to improve communications with physicians. In addition, formulation of effective queries and the use of most appropriate query reformulation types lead consumers to successful health information searches. In relation to the starting point of a search, it was identified that, at the beginning of a search process consumers search for more general concepts via search engines. Moreover, consumers tend to evaluate the usefulness of search results based on factors, such as trustworthiness, quality, features of websites, reliability, ranking, familiarity, personalized content and prior knowledge. The organization of the search results lists is important, because consumers use it to assess the efficiency of health information systems. Furthermore, domain knowledge, topic familiarity and search skills have a higher impact on health information searches performed by consumers. Therefore, consumers who have higher levels of domain knowledge, topic familiarity and more search skills are able to perform successful health searches. Hence, health information search still remains as a challenging task to most of the consumers because consumers have problems with query formulations and selection of useful search results. As a solution, information portals, individual websites and education tools are suggested to be used as a support for consumers when searching for health information. 

The second key aspect of consumer health search identified was strategies used to improve user queries. According to the findings of this aspect, query recommendations, clarifications and suggestions, all are able to increase the number of successful queries and therefore, positively impact consumer health searches. The use of different synonym mappings, such as Behavioural, MedSyn and DMpedia has the ability to improve query clarification processes with varying levels of performance. In addition, query clarification process can be further improved by using a logistic regression classifier which ensemble all three synonym mapping methods and a method with no query clarification. Furthermore, lay people prefer to have query recommendations, clarifications and suggestions while searching for health information, because they are able to retrieve more useful information with the assistance of these techniques. 

The third key aspect of consumer health search identified was methods used to enhance consumers' understandability of health information. The findings of this aspect relieved that, both user side approaches, such as users’ cognitive style and the use of their own contexts (social and psychological), and system side approaches, such as contextualisation of health information (providing supportive information to thoroughly explain a health concept) which are used to enhance consumers’ understandability of health information have a positive influence on overall consumer health information searches, because they assist to increase consumers’ understandability of health information.

The fourth key aspect of consumer health search identified was retrieval models. According to the findings of retrieval models, the use of most suitable retrieval model is able to effectively expand more queries and retrieve highly useful information. In addition, relevance feedback can be used to further improve the effectiveness of these query expansions.

The fifth key aspect of consumer health search identified was problems faced by consumers when searching health information online. The findings of this aspect revealed that, consumers face various problems when searching for different types of health information. When they search for self-diagnosis information they can face many problems, such as experiencing anxiety after reading inaccurate or false diagnosis information, unreliability of available information, making users buy drugs from Internet pharmacies, ignore professional assistance by believing everything on self-diagnosis websites, and making erroneous and harmful self-diagnosis decisions. In addition, they also face problems when determining the efficacy of health treatments, because their decisions are highly dependent on search results' bias. In other words, if search results are bias towards inaccurate information consumers tend to make erroneous and harmful decisions about the efficacy of health treatments. Furthermore, there is a gap between the supportive information provided to users by health information websites and users' own personal contexts (social and psychological) which they use to support their understanding of health information.  

The sixth key aspect of consumer health search identified was methods used to evaluate the relevance of health search results. The findings of the methods used to evaluate the relevance of health search results revealed that, search results which contain information directly relevant to what user has searched are considered as highly relevant by human assessors. In addition, when using an approach, such as Decision Trees to determine the relevance of health search results it is important to use a method with feature selection, because it improves the accuracy of decisions made on the relevance of search results.

The final key aspect of consumer health search identified was techniques used to map user queries to medical symptoms. According to the findings of techniques used to map user queries to medical symptoms, it is understood that, users are able to correctly describe and search for a symptom, when they actually experience it. In addition, the use of features, such as Encyclopedias, anatomy (body parts), synonyms, medical dictionaries etc. is highly useful because these features improve the performance of health information search systems, when matching user queries with symptom names.      

------------------------------------------------------------------------------

uncovered issues: 

There are a few uncovered issues which were noticed while analysing recent research papers. One of the main issues is that, studies have not been conducted by performing the same consumer health information search task in several different consumer health information websites. This is important in order to understand how different architectures, structures and interfaces of websites influence, consumer health information searches. In another study, which aimed at investigating how query expansions impact on improving initial user queries and the overall retrieval performance, authors were not able to provide clear conclusions, because of the unavailability of some other important information. For instance, relevance assessments for the test collection used for the study.   
 

limitations of this lit analysis:

only a few papers covered for some aspects




key recommendations arising from my work:

next steps that must be taken to move the project forward/ develop the work further (propose a research agenda base on this literature analysis):

Based on this literature analysis, it can be concluded that, it is really important for consumer health information system designers to consider above mentioned findings as much as possible when designing these systems. It will be really helpful for consumers with low levels of domain knowledge to retrieve more useful health information as required by formulating effective queries and selecting the most useful health search results.   

\end{document}




