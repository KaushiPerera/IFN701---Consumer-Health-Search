\documentclass[]{article}
\usepackage{graphicx}

%opening
\title{Annotated Bibliography}
\author{Kaushi Perera}

\begin{document}
	
\maketitle

\section{Broussard, R., \& Zhang, Y. (2013). Seeking treatment options: consumers' search behaviors and cognitive activities. Proceedings of the Association for Information Science and Technology, 50(1), 1-10.} 

This study investigated consumers’ exploration of treatment options in both behavioural and cognitive perspectives. Two research questions were addressed; 1. the behaviours consumers exhibit when searching for treatment options online; 2. the cognitive activities involved in the search: finding, selecting and evaluating information. The approach of the study had two parts; 1. participant observation; 2. post-session interview. 40 people participated in the study. Two interfaces were used; 1. a classic Web search engine interface with a basic search box and the results were presented as a ranked list; 2. a Scatter/Gather-enabled search interface with a basic search box, but the results were grouped into a number of clusters. These clusters were ranked based on their size. 20 participants were assigned to each interface and were asked to search treatment options for migraines’. Four types of data were collected; 1. participants’ demographics and experience with health information search; 2. transaction logs about session length, queries submitted, sites visited, participants’ ratings on relevance and usefulness of the sites visited; 3. participants’ ratings on the mental efforts required to complete the task and their satisfaction with performance; 4. information gathered via playback interviews (playback of the search process): selection of keywords, query reformulation and examination, selection of particular search results, and evaluation of search results. Two different interface groups were pooled together for further analysis because they had no difference in information, such as session length, number of queries submitted, and sites visited.                       

The authors observed that the mean values for web experience, health search experience and health search frequency were 13.2 (years), 3.7 (years) and 2.8 (times per month) respectively. Average ratings for the perceptions of the task were; familiar: 2.3 (somewhat familiar); easy: 2.5 (somewhat difficult); effortful: 2.7 (medium amount of effort) and satisfied (with results): 4.2 (generally satisfied). Three aspects were analysed for search behaviour; 1. session length: average session length was 10.25 minutes; 2. basic query behaviour: average number of queries issued was 4.2 and average query length was 3.8 terms; 3. web sites visited (60\% of all visits): frequently visited medical-specific websites: WebMD (26 visits), livestrong (22 visits), mayoclinic (17 visits), migraines.org (19 visits) and migraines.com (10 visits); frequently visited general-purpose sites:  ehow (22 visits), Wikipedia (18 visits) and buzzle.com (5 visits). Sites with checkout behaviour (73.2\% of all the visits): medical-specific, evidence-based and general-purpose sites; sites visited twice: 50\% general purpose sites and 50\% medical-related sites; 44 unique websites were visited only once. Four aspects were analysed for cognitive activities; 1.query formulation and reformulation (89 times): specification (22), generalization (9), parallel move (25), new concept (27) and rephrase (6); 2. examination of search results: trial and error approach was taken to decide whether to select a search result or not; the rank of a result, familiarity of websites and the validity of information were important when selecting a search result; 3. judgement of sources: the design (clarity, simplicity, clean and well-structured text with images), readability (text without jargon), completeness (pros and cons, side effects and expensiveness of the treatments), and credibility (depending on participants' experience and the content of websites) were used to judge sources; 4. search timeline -cognitive development: one pattern was to start with a general search and gradually move towards more specific information; another pattern was to start from confirming information and gradually move towards novel information and validating that novel information (double check a fact).
      
In conclusion, the behaviours consumers' exhibit when searching for treatment options online:  submitting short queries with misspelled keywords and selecting results exclusively from the first page of search results. Websites visited: participants relied on both medical-specific and general web sites; web sites with its name containing the medical condition searched were visited more frequently (20\% of the 66 unique sites). Cognitive activities performed when searching for treatment options: start the search with general concepts and gradually move towards different treatments and the aspects of each treatment, and select search results based on familiarity. Search engine rankings and familiarity of the web sites were more important when selecting results when compared to trustworthiness, quality or usefulness of the information contained in the websites. Therefore, search engines need to be designed in a way that they will support different consumer behaviours.  

\textbf{Limitations}

The task was a simulated task and did not reflect the participants’ real needs. Due to the exploratory nature of the study, only one treatment option task was assigned to the participants, thus participants’ behaviour may be affected by the nature of the task. A lab setting was used for the study, but it is not a natural environment for treatment search.  

\section{Keselman, A., Browne, A. C., \& Kaufman, D. R. (2008). Consumer health information seeking as hypothesis testing. Journal of the American Medical Informatics Association, 15(4), 484-495.} 

This study investigated the most common patterns consumers follow when searching health information, depending on their initial theories, search strategies and comprehension. These patterns were then categorized as successful and failed. A framework was used for the study; this framework had two perspectives: hypothesis testing perspective and Human Computer Interaction (HCI) perspective to obtain additional insight regarding consumers’ difficulties in locating and interpreting health information; 4 states were considered: 1. beginning state which constitutes of the background knowledge (theory) and initial hypotheses (perceived information need); 2. search goal; 3. shaping the search goal by search action steps; 4.evaluation of retrieved information. 20 lay individuals with different levels of education participated in the study; they were presented with a hypothetical scenario which described symptoms of ‘stable angina’; they were asked to do two tasks: 1. semi-structured interviews where possible causes of the symptoms were discussed; 2. search MedlinePlus to seek information on the disease. Information gathered via semi-structured interviews was compared with a reference model of stable angina. The searching processes were categorized as action-related and competency processes; action-related represented actions performed or steps of information seeking path; competency level included: Domain Knowledge (e.g., of heart disease),General Search Strategies (e.g., query expansion), Resource Knowledge (of MedlinePlus), Metaknowledge (of desirable site characteristics) and Language (spelling and vocabulary knowledge); each action step was assigned with one of these competencies; each searching process was integrated with the corresponding verbal transcripts obtained during semi-structured interviews for further analysis (examine search strategies and visualize trends across participants’ behaviours). 

The authors observed that overall participants’ understanding of the scenario was incorrect or imprecise; participants' understanding was different to the reference model in three aspects; 1. key concepts: coronary artery disease (CAD) and angina were in the reference model, but no participant mentioned them, and the third concept was atherosclerosis in the reference model, but only 3 participants mentioned a lexical form of this term; for the majority of the participants 'heart attack' was the primary hypothesis; ‘blockage of blood vessels’ was the main cause for such cardiac problems; tear to the heart muscles, irregular heart beat and “electrical problem with the heart” were the other mentioned causes;  non-cardiac problems, such as stroke, arthritis, asthma and diabetes were also mentioned by the participants; 2. symptoms’ grouping: reference model indicated that all symptoms were related one condition; in contrast participants mentioned that nausea and dizziness are unrelated to a cardiac problem; 3.symptoms’ characteristics: the short duration of pain, its relation to exertion and response to rest were not noted by the participants; in contrast the reference model identified the significance of these factors. 

Participants were categorized into three main clusters: Verification-First, Problem Area Search-First and Bottom-Up for the analysis of Information-Seeking Processes.1. Verification-First Cluster: represented 8 participants (40\%); they started the search process by attempting to verify a specific illness related to heart attack; the only strategy used was ‘verification’; one participant correctly concluded the scenario as angina; majority of them (7) incorrectly concluded the scenario as related to heart attack by only considering the similarities between descriptions of heart attacks and symptoms in the scenario; the behaviour of ignoring symptoms’ characteristics that are viewed as non-essential was identified as ‘selective perception bias’; the behaviour of stating that information confirmed their hypothesis depending on information of a site they visited, was identified as ‘confirmation bias’; the behaviour of stopping search just after reviewing only one content topic was identified as ‘premature search termination bias’; 2. Problem Area Narrowing-First Cluster: represented 5 participants (25\%); they started the search process with problem area search; participants had both ‘Area’ hypothesis (start searching with queries) and ‘Assorted’ hypothesis ( browse the site index tree); one participant had switched to a ‘bottom-up’ approach during this searching process; sites with specific disease information were visited; participants had left without a conclusion rather than providing incorrect conclusions; the behaviour of picking text from the scenario (pain episode only lasts 2–3 minutes) and making conclusions (very minor heart attacks) based on that was identified as ‘selective perception bias'; 3.Bottom-Up First Cluster: represented 7 participants (35\%); some started the search process without a specific hypothesis; the attempts to search for a general-purpose diagnostic tool were unsuccessful; some participants had  switched to other ‘hypothesis-driven strategies’, but made incorrect heart attack conclusion; neither of the participants in this cluster had made the exact correct conclusion as angina; the behaviour of trying to match description of a heart attack with the facts mentioned in the scenario was identified as ‘selective perception bias’                      
     
Other factors: domain knowledge was important for setting goals and information evaluation; domain understanding was important for determining the direction of the search and for the interpretation of the results; resource knowledge, strategies and metaknowledge were important for navigational actions. Participants were able to understand the symptoms described in the scenario disregard of their education level; participants with higher levels of education were likely to be familiar with MedlinePlus, had used efficient search strategies and made meta-level comments, such as judging the authoritativeness of a source

In conclusion, incomplete and inaccurate domain knowledge (setting information goals and evaluating retrieved information), imprecise search queries entered and the inconvenient configuration of the web resources were identified as main causes of the problems faced by participants when searching for information and selecting correct results. An incorrect hypothesis can cause the search of irrelevant resources;  prior hypothesis and background knowledge influences hypothesis generation, evidence interpretation and evaluation of retrieved information; two aspects of MedlinePlus interface influenced participants’ search process: 1. lack of explicitness in relating lay and professional terms in the index; 2.the order and organization of query results lists. Searching processes with a specific preconception ended with incorrect conclusions; searching processes without a specific preconception ended without any conclusion.The importance of identifying the difficulties faced by consumers’ is that health information website(consumer health sites) designers can try eliminate these difficulties by providing support in places where consumers were tend to behave erroneously; in information portals like MedlinePlus: provide query suggestions, present information in an organized manner and suggest consultation with a health professional; in individual websites: address needs of targeted users with consumer-friendly terminology; education tools: teach consumers how to formulate specific queries,evaluate qualifiers in the information and not to terminate their searches prematurely.

\textbf{Limitations}

The hypothetical nature of the scenario (possibility of affecting the users’ motivation), the complexity of the diagnostic task and the similarity between angina and heart attack symptoms (ambiguity in the description of the scenario) were identified as limitations and potential reasons which led this study to result in a very low success rate. 

\section{Zhang, Y., Wang, P., Heaton, A., \& Winkler, H. (2012, January). Health information searching behavior in MedlinePlus and the impact of tasks. In Proceedings of the 2nd ACM SIGHIT International Health Informatics Symposium (pp. 641-650). ACM.} 

This study investigated consumer health information searching behaviour (interaction behaviour) in MedlinePlus which was a browsing oriented system with a simple search engine, and the impact of the number of concepts involved on the search behaviours; three research  questions were examined: 1. how do users search MedlinePlus?; 2. how do users browse MedlinePlus?; 3. how do search tasks influence interaction strategies?; the difficulties faced by consumers when searching and how they handled these problems were also examined. 20 undergraduate students who have not used MedlinePlus before participated in the study; three search tasks were given to each participant; 1. find arguments for and against the use of marijuana for medical purposes; 2. find the relation between Type I and Type II diabetes, and hypertension; 3. find the functions of liver and kidney, what is the role of insulin in the liver and kidney, why insulin would be needed? and whether insulin is related to liver and kidney diseases?; prior to performing the tasks information regarding participants' spatial ability and demographic factors were collected; participants' searching and browsing behaviours were examined during the search tasks; after each task participants' opinions on difficulty of the task, required mental effort and satisfaction of the search performance in MedlinePlus website were collected.                

The authors observed that participants' spatial ability score ranged between: 6.8 to 17.6; Internet experiences ranged between: 6 to 13 years; online health information searching frequency: on a yearly or monthly basis; session lengths: 11.63 to 29.01 minutes; the time spent, difficulty and the required mental effort for task 3 were significantly higher than the others; satisfaction of task 3 was the lowest among three tasks. Four aspects were considered for the analysis of search behaviour; 1. query features (number of queries and terms): task 3 had the maximum number of queries per person which was statistically significant and task 2 had the maximum average query terms which was not statistically significant; 2. search terms: three types of search terms were identified; (1) keywords with semantic meanings (meaningful terms); (2) stop words (no semantic meaning); (3) search operators (AND, OR); 3.query reformulation: two factors were considered; (1) executed actions which modified the initial queries; three actions were identified; concept related changes: add (28.9\%), delete, repeat/re-execution, replace (30\%) and change to a new concept/ switching topic (20.7\%); task 3 had the maximum number of query reformulations; form of terms: change the form of term and correct misspelled terms; conceptual relationships (boolean operators): changing boolean operators and altering the order of the words; (2) subsequent conceptual changes to the queries; four categories were identified; specification (the inquiry becomes more specific); generalization (the inquiry becomes more general); parallel movement (the reformulated query has a partial overlap with the previous query; two aspects of one query); replacement with synonym (replace the current terms with words with similar meaning); 4. accessing and evaluation of results: two options, such as directly access the results (results from Health Topics section and Encyclopedia) and filter out results; majority directly accessed the results rather than filtering the results; a few also used the options to filter out the results; the results were evaluated by scanning, checking the authors, checking the source of information and the organization of the content in the resources. Two aspects were considered for the analysis of browse behaviour; 1. accessing different resources: participants accessed 'Drugs \& Supplements' and 'Encyclopedia' from the alphabetical lists, Dictionary, Health Topics section ('Alphabetical list', 'Body Location/System', 'Diagnosis and Therapy' and 'Health and Wellness'), news section,  Directories, Go Local and Multiple Languages sections; 2. accessing related topics: in-text and related topics list hyperlinks available on MedlinePlus health topic pages were used for task 3. 

In conclusion, session length of a searching process increases with the complexity of the task. Most query reformations were related to conceptual changes (85\% ) while making queries specific,  followed by making queries more general, switching to a new topic and making parallel movements. Query re-executions were performed the most as query iterations. Participants with higher understanding of the scenario explained in the search task also searched for concepts associated with the concepts mentioned in the scenarios; the use of stop words demonstrated participants' preference of using natural language during search. Generally searching and browsing strategies were used as a combination mostly when performing task 3 (relatively higher complexity). Only one strategy was used mostly when performing task 1 and task 2 (relatively lower complexity). The searching process patterns of task 3 were more diverse, complex, exploratory, and iterative when compared to other two tasks. This indicated that task complexity influences interaction strategies and patterns. Searching strategies were preferred the most because it was easier to perform than browsing strategies. Encyclopedia, Health Topics, dictionary and links to related topics were used to gain preliminary understanding of the scenarios in complex tasks, such as task 3. Therefore, the need for functions to detect misspellings and provide query suggestions with automatic synonym expansions, a hierarchical terminology structure to assist participants in selecting query terms for query construction, suggestions for alternative moves when reformulating queries, the visibility of information architecture to users, advanced search function to support exploring relationships between multiple health concepts, backtrack function to access search histories, mechanisms to connect lay terms with medical terms and encouragement for evaluating search results were identified when designing consumer health-related information websites, so as to aid search tasks with varying complexity.         

\textbf{Limitations}

This study recruited undergraduate students and they tended to have more versed skills in web searching but less often on health-related topics. Predefined tasks were used for the study rather than focusing on users’ real needs. It is important to perform the same tasks in a different consumer health information websites to understand how different information architectures or structures and interfaces influence interaction behaviour.

\section{Toms, E. G., \& Latter, C. (2007). How consumers search for health information. Health informatics journal, 13(3), 223-235.}

This study investigated consumer health information searching behaviour online; three aspects were considered; 1. how people specify their information requests; 2. how people select from search result lists; 3. how people examine the page(s) they declare relevant to the information search problem. 48 young, educated adults who have used the web before participated in the study; search tasks were performed in Google; a set of Open Directory categories were added to Google so as to provide participants with an alternative option which is a scan capability; 4 tasks were given to participants; 2 tasks were fully specified; 1. find three categories of people who should or should not get a flu shot and why; 2. find a website likely to contain reliable information on the effect of second-hand smoke; the other two tasks could be personalized; 3. list two of the generally recommended treatments for (fill in the blank with a health-related matter that interests you); 4. identify two pros or cons of taking large doses of (fill in the blank with a drug or treatment or remedy that interests you); information regarding participants' demographic factors and web search experience were collected; each participant was assigned to one of the four tasks; information, such as participants' familiarity and expertise with the search topic, actions performed while searching for the topic (issuing queries, selecting categories, examining pages and selecting results), and participants' perception on completing the task was collected; judges assessed the ‘aboutness’ of the pages declared as relevant by the participants; completeness of the tasks were assessed by examining the pages which were recommended as useful by the participants.

The authors observed that participants were ‘somewhat familiar’ to ‘very familiar’ (66\%) with the topics of the search tasks; average number of queries created for each task: 1.3 queries; average number of keywords in each query: 4.3 keywords; 63\% used only the search box to enter a query; 6\% used the categories; the remainder used both queries and categories; on average 4.5-9 minutes were spent per task; time was mostly spent on interpreting results page to comprehend the information presented on the web page. Formulating queries and selecting categories: to perform the first two tasks participants issued various queries and selected various categories; for task 1, 12 participants issued 23 entries; however, for the task 2 less variability in query content was observed; query formulation was a quick process; the use of keyword search either on a search engine or on a website of interest was the most popular query formulation; keyword search was also used as a ‘trial and error’ method; with the use of keyword search participants were able to control the search process and results; they made the search more robust ( adding specific words while searching), restricted the search by specific sources of information and the reliability of the sources of information,  preceded the search terms with a plus sign to ensure results fell within a common page and encapsulated search terms in quotation marks to ensure that all terms are found together in sites identified in the search results; therefore keyword search method was preferred over the use of categories because categories were seen as a less direct route (included more steps) to locate information, information contained in categories was too general, increased exposure to advertisements and increased time taken to find relevant information.
    
Selecting from the results list: on average 5.4 pages of results lists were examined; average rank on a results list page was 4 therefore, all items on the results list page had an equal likelihood of being selected; summaries/ description, URLs (determine the nature of the site:an information-based site or an e-commerce site) and titles were the mostly used to select the most appropriate links to explore; in addition, dates, size of the site and type of file (e.g. PDF) were least used when selecting appropriate links; the retrieval of a large number of results was indicated as the search needed to be refined; URLs were used to  assess the potential credibility (university, government, scientific, pharmaceutical research information and associations for medical professionals were considered reliable) of the information found;some participants looked at links on several pages of the results list and some chose the first link on the results list. Identifying appropriate websites: average number of web pages selected was 2.6 web pages; average web page ‘aboutness’ was 4.5 out of 5; an average score of 4.5 out of 5 was assigned to the pages which were identified as relevant to the tasks by the participants; therefore, the pages examined were clearly on the topic of the search and the task completion was between 80-100\% of each task; 44\% of the relevant pages did not had any advertisements and 40\% were created by a government agency; the pages were classified as informational,journal articles, fact sheets and newspaper articles; the use or rejection of a site based on three key considerations; 1. information expectations: does the site provide any or all of the information being sought?; 2. information quality: author of the site and the information? the purpose of the site (e.g. to provide information, to sell products, to promote unsubstantiated opinions)? the way information was presented (e.g. opinion-based articles, discussion group, academic research, formal medical findings)?; 3. information presentation and accessibility: ease of finding the desired information on the site? is the information easy to understand?
             
Factors, such as what participants were looking for, their knowledge of the area, their biases in terms of credibility and their personal understanding of the topic influenced keyword searches; the system limited the improvements that could be performed while constructing queries and the apparent system improvements were not realized by the participants; not knowing what a category contains and having an endless set of category levels were identified as barriers when using categories; the information design of the pages had a significant barrier to the online health consumer because time taken to select from the results pages was high; erroneous decisions were made by considering the web page appearances; credibility, reliability and trustworthiness were important when selecting pages from results lists; the major challenges for a successful search were formulating good queries and having a results list with appropriate design and standard (easy scanning and efficient recognition).  

In conclusion, consumer health information search still remains as a challenging task for the average person because they have different comprehension levels, searching abilities and levels of information needs; both information design and search engine technology were identified as important to build good consumer health information systems; therefore, the need for a system which provides assistance to query development, which will evaluate the information that is being provided, which will have a flexible and responsive terminological infrastructure to support the search process and which will consist of summaries on the results list indicating the  reliability, credibility, type and information content of the documents that they represent was identified.   

\section{Jimmy, Zuccon, G., \& Koopman, B. (2018, March). Choices in Knowledge-Base Retrieval for Consumer Health Search. In European Conference on Information Retrieval (pp. 72-85). Springer, Cham.}  

This paper investigated how to overcome consumer health search problems by expanding or reformulating health queries with more effective terms; specialised health knowledge bases (MeSH and UMLS) and general knowledge bases (Wikipedia) were used to expand consumer health queries;the 'Entity Query Feature Expansion' model was used for retrieval on both the Wikipedia and UMLS knowledge bases in this empirical evaluation;  in Wikipedia KB page title, categories, links, aliases, and body were the useful features for a retrieval scenario; in UMLS KB concept unique identifier, aliases, body, parent concepts and related concepts were the useful features for a retrieval scenario; the impact of different choices in knowledge based retrieval on query expansion and retrieval effectiveness of consumer health search was examined in five different aspects; (i) knowledge based construction: for Wikipedia three choices were made to collect health related pages; 1. pages with Medicine infobox type; 2. pages with Medicine infobox type and with links to medical terminologies, such as Mesh and UMLS; 3. pages with title matching an UMLS entity; for UMLS two choices were made; 1. all entities; 2. entities related to four key aspects of medical decision criteria, such as symptoms, diagnostic test, diagnoses and treatments; (ii) entity mention extraction: text from the query which can be mapped to entities were identified; (iii) entity mapping: exact matches between mentions and entities were identified to map a mention to an entity; (iv) source of expansion: the sources from the knowledge bases were selected to use for drawing candidate terms for query expansion; (v) use of relevance feedback: both explicit relevance feedback and Pseudo Relevance Feedback were considered; a term was considered as health related if it exactly matched with a title or an alias of an entity (a single Wikipedia page or the most frequently used terms for a single concept unique identifier) in the target knowledge base (Wikipedia or UMLS); the preference to use a specialised knowledge base over a general knowledge base was also investigated.

300 query topics were considered for this study; a baseline was used for the evaluation;  the title field and the body field were considered for the evaluation; Wikipedia knowledge base consisted of a set of candidate pages and they were indexed using field-based indexing; therefore, the fields were used as a source of query expansion terms; UMLS knowledge base consisted of English terms indexed with fields, such as title, aliases, body, parent and related;  the average number of terms added in the expanded query and the number of expanded queries were recorded; 1. Knowledge Base Construction: for Wikipedia knowledge base, infobox type and links to medical terminologies had the highest effectiveness in information retrieval; for UMLS knowledge base, all entities had the highest effectiveness in information retrieval; baseline method performed considerably better than any of the knowledge base retrieval methods because  knowledge base retrieval methods ranked many unjudged documents amongst the top 10 results for a large number of queries in contrast to the lower number of unjudged documents ranked by the baseline method; 2. Entity Mentions Extraction:for Wikipedia knowledge base, the entity mentions which included only those uni-, bi- and tri-grams of the query that
matched entities (Wikipedia pages) in the Consumer Health Vocabulary had the highest effectiveness in information retrieval; for UMLS knowledge base, the entity mentions using all uni-, bi-, and tri-grams of the queries terms had the highest retrieval effectiveness; 3. Entity Mapping: for both Wikipedia knowledge base and UMLS knowledge base, mapping entities to Aliases had the highest retrieval effectiveness; 4. Source of Expansion: for both knowledge bases the selection of titles as the source of expansion had the highest retrieval effectiveness; 5. Relevance Feedback: for Wikipedia knowledge base the addition of feedback had produced mixed results; for UMLS knowledge base explicit relevance feedback had produced the best performance; however the baseline method had performed worse when compared to the knowledge base methods because only explicit relevance information was able to improve its retrieval performance.  

In conclusion, pseudo relevance feedback did not improve results, independently of the knowledge base; in contrast, relevance feedback provided improved effectiveness in both queries that were expanded and the extent of expansion; UMLS knowledge base generally provided better improved effectiveness when compared to Wikipedia knowledge base; UMLS knowledge base expanded more queries than the Wikipedia knowledge base because the Wikipedia knowledge base was comparatively incomplete by only having pages with health infobox and links to medical terms; the two methods had provided different query expansions with an average of only 8.9\% of common expansion terms;  the two methods retrieved different sets of documents with an average overlap of 61\% of the top 1,000 documents; the effective choices for knowledge base query expansion lead to produce the lowest number of expansion terms and expanded the smallest number of queries; relevance feedback had added a significant number of expansion terms and expanded a large number of queries; in total by both the knowledge bases, 183 queries were expanded; 16 of these expansions had no change in effectiveness when compared to the baseline; 92 showed improvements and 75 showed losses. Overall, UMLS knowledge base was more effective when compared to Wikipedia knowledge base; UMLS knowledge base also performed better than the baseline; therefore, it was confirmed that the use of a knowledge-base retrieval approach has the ability to translate well into the challenging consumer health search domain.                 

\textbf{Limitations}

The number of unjudged documents retrieved using the expanded queries was the major limitation of this study, which made it challenging to fairly evaluate the methods. In spite of that, this investigation of choices in knowledge base retrieval for consumer health search has been able to highlight both pitfalls and payoffs.
 
\end{document} 