


\documentclass[]{article}
\usepackage{graphicx}
\usepackage[svgnames]{xcolor}
\usepackage{ragged2e}

\usepackage{hyperref}
\usepackage{longtable}
\usepackage[super]{nth}
\usepackage{textcomp}



\hypersetup{
	colorlinks,
	citecolor=black,
	filecolor=black,
	linkcolor=black,
	urlcolor=black
}

%opening

\begin{document}

%titlepage
\thispagestyle{empty}
\begin{center}
	\begin{minipage}{0.75\linewidth}
		\centering
	\centering

	\Large \textbf{IFN 701 Project}\par
	\vspace{3cm}
		%Thesis title
		{\uppercase{\Large \textbf{A literature review on Consumer Health Search}\par}}
		\vspace{6cm}
		%Author's name
		{\Large 
			
			\par}\begin{tabular}{c}
			Student ID: n9789511 \\
			Student Name: D. F. Kaushalya S. Perera\\			
			Project Supervisor: Dr. Guido Zuccon\\
			Project Coordinator: Dr. Xiaoyong Xu\\		
		\end{tabular}
	
			
		\vspace{3cm}
		%Date
		{\Large \nth{03} June 2018}
	\end{minipage}
\end{center}
\clearpage



\textbf{Abstract}\\

At present search engines are increasingly being used by consumers to particularly retrieve health information. However, they also face many difficulties when they try to retrieve reliable and useful health information, as a result of their knowledge and language gap. Unfortunately this has lead to the retrieval of unreliable and inaccurate health information which also can ultimately cause them harm. Therefore, researches on information retrieval have mainly focused on consumer health information retrieval to understand and improve consumers' health information retrieval process. The purpose of this literature analysis was to contribute to the existing scattered knowledge of consumer health search presented in recent research papers by summarising the key findings and categorising them into different aspects of consumer health search. The scope of this project was to analyse research papers containing information about different aspects of consumer health information retrieval and they needed to be published 2005 onwards. The methodology of this literature review project consisted of four main steps, such as defining a protocol, including the keywords/phrases, inclusion/exclusion criteria and search services, retrieving relevant research papers, preparing the annotated bibliography and the table of topics, and finally preparing the literature review.

This literature analysis has covered 22 research papers and have categorised the findings of each research paper into different aspects of consumer health information search while preparing the table of topics. Therefore, the key findings of these 22 research papers have been categorised into 7 sub aspects of consumer health information retrieval, such as (1) User search behaviours, (2) Strategies to improve user queries, (3) Methods to enhance consumers' understandability of health information, (4) Retrieval models, (5) Problems faced by consumers when searching health information online, (6) Methods to evaluate the relevance of health search results and (7) Techniques used to map user queries to medical symptoms. Then each aspect of consumer health information search has been analysed separately to compare findings from different research papers in order to identify various patterns among them.  

There are two main recommendations from this literature review completed. One is for the designers of consumer health information systems, to embed these findings as much as they can while designing Consumer Health Information Systems in favour of consumers, so that consumers can formulate effective queries and retrieve more reliable health-related information. The other recommendation is for researchers who are keen to conduct research on consumer health information system to conduct more research in this area, because still there are plenty of gaps in knowledge remaining in this area.

\pagebreak

\tableofcontents

\pagebreak	

\section{Introduction}    
	
\subsection{Background and Context}
	
At present online resources are heavily used by lay people to search different types of information. A significant portion of these searches are performed to retrieve health information. Hersh \cite{hersh2015information} has stated that, the use of Information Retrieval systems by consumers to retrieve health information has become ubiquitous. For example, as Zuccon et al. \cite{zuccon2015diagnose} have highlighted in their paper, an analysis conducted using three web search engines, has been able to reveal that, approximately 10\% of user queries issued to these search engines are health related.  Hersh \cite{hersh2015information} also mentions that, at present, most scientific papers are published electronically. These profound changes in publishing information is also identified as a main reason for this escalated use of online resources by consumers to seek health information.
	
Consumers search health information online to obtain various types of information, such as medical information, details about health professionals, self-diagnosis information and to decide the effectiveness of medical treatments 
\cite{pogacar2017positive}. However, when lay people perform health information searches, there is a higher possibility of them retrieving inaccurate, irrelevant and unreliable information, because they are more likely to use lay terms to describe a medical condition. For instance, a lay person might issue a query, such as ‘my head is pounding’ except typing the proper medical term (cephalalgia) when searching for medical information. This issue is known as the ‘circumlocutory in medical queries’ \cite{stanton2014circumlocution}. In addition, consumers also tend to express health conditions using different words which might make it hard to retrieve information containing words that matches those consumers\textquotesingle  words. As Croft et al. \cite{croft2010search} mention, this issue has been identified as the ‘vocabulary mismatch problem’. In other words, this basically means that, although the main purpose of information retrieval models is to fetch (retrieve) information which satisfies users’ information needs, the information that is being retrieved highly depends on the issued user queries. Consumers are highly likely to have language and knowledge gaps, such as not knowing adequate vocabulary related to health domain and might also be unfamiliar with most of the proper medical terms \cite{soldaini2016enhancing}. 
	
Therefore, consumers\textquotesingle  increased use of online resources to retrieve health-related information can also increase the potential problems associated with the retrieval process, such as retrieving inaccurate, unreliable and useless information. For example, as Pogacar et al. \cite{pogacar2017positive} have stated, users are highly likely to be influenced by inaccurate and unreliable information, when searching for the efficacy of medical treatments, and might end up having harmful impact on their own lives. Hence, it is crucial for health information search systems to satisfy users’ health information needs by retrieving authoritative, accurate and useful information.  
	
Because of the importance of this field, researchers have focused more on to understanding and improving health information retrieval processes as a means of making sure that health information retrieval systems are able to satisfy consumers\textquotesingle’ information needs. As a result, researches have been conducted to cover various aspects, such as user search behaviours and retrieval models of consumer health information searches \cite{toms2007consumers,zuccon2018choices}.    
	
\subsection{Aims, Objectives and the anticipated Significance of this Literature Review Project}
	
The aim of this literature analysis was to contribute to the various aspects of consumer health information searches, investigated in recent researches, by synthesizing this information and presenting it as a literature analysis. Therefore, this literature analysis aimed to cover various aspects, such as user search behaviours, consumer health information retrieval models, strategies to improve user queries and different problems faced by consumers when searching for health information online. This synthesized knowledge and information then has been reported as an analysis of different aspects covered in each research paper, methodologies which have been used in each study and their corresponding results in a well-structured literature review paper. Prior preparing the final literature analysis, this project also prepared an annotated bibliography by including annotations for all the research papers covered and a table of topics summarising the claims of each research paper. Therefore, this analysis is really useful for health information search system designers because they can find heaps of useful information which can be embedded when designing such systems. In addition, this information is also important for researchers who are interested in conducting researches on ‘consumer health information search’, because this literature analysis also highlights some of the uncovered aspects of consumer health information searches. In other words, this literature analysis highlights areas which are pertinent for future studies.	
	
\subsection{A brief Overview of the Project Methodology}
	
The methodology of this literature analysis consisted of a few main steps.\\
	
\textbf{Step 1:} Defining a protocol for the literature analysis. The main purpose of defining a protocol was to retrieve more relevant and useful research papers to conduct this literature analysis 
	
\textbf{Step 2:} Reading and analysing the chosen research papers to gather knowledge and information related to consumer health information searches including:

\begin{enumerate}
	\item User search behaviours
	\item Retrieval models
	\item Strategies used to improve user queries
	\item Problems faced by consumers when searching for health information online	
\end{enumerate}	 
	
\textbf{Step 3:} Preparing an annotated bibliography based on all the covered research papers
	
\textbf{Step 4:} Preparing a table of topics summarising the claims in each research paper
	
\textbf{Step 5:} Preparing the literature analysis by highlighting different topics (aspects), methods and findings covered in each research paper. In addition, a few uncovered aspects of consumer health information searches were also highlighted in this literature analysis
	
More details on the methodology which has been applied to conduct this literature analysis are presented under the section 'Literature Review Methodology'. \\ 
	
\subsection{Project Scope}
	
This literature review only focused on recent research papers which were published 2005 onwards. The knowledge and information in these research papers also needed to contribute well to at least one of the main aspects of consumer health information searches as listed below.

\begin{enumerate}
	\item User search behaviours
	\item Consumer health information retrieval models
	\item Strategies to improve user queries
	\item Problems faced by consumers when searching for health information	\\
\end{enumerate}	
	
\subsection{A brief Summary of the Key Deliverables}
	
This literature analysis consists of a few main deliverables.\\
	
\textbf{1. An annotated bibliography:} This includes annotations for each of the chosen research papers. Each annotation highlighted key factors, such as the aim of the study, the methodology which has been followed, key findings and the importance of those findings as it was presented in each research paper.  
	
\textbf{2. A table of topics and claims:} This table includes summaries of different topics and claims covered in each of the chosen research paper. For example, an excel spreadsheet was used to note down different aspects, such as user search behaviours and retrieval models covered in each research paper with their claims/ key findings. Then these findings were analysed to identify patterns within each aspect.  
	
\textbf{3. A presentation highlighting the key findings of the analysis:} The final presentation which was prepared and presented in week 12, included information, such as different aspects of consumer health information search covered in each research paper, the key findings of each research paper and the importance of each of the finding.  
	
\textbf{4. A literature review paper which includes a thorough analysis of the chosen research papers:} The final literature review paper consists of an analysis which includes, different aspects of consumer health information searches covered in each chosen research paper and their corresponding findings. Hence, this literature analysis is a synthesis of the key findings of each research paper depending on which aspect of consumer health information search it covers. In addition, this literature analysis has also highlighted a few uncovered aspects of consumer health information search by the chosen research papers.
	
\section{Literature Review Methodology}

\subsection{Project Methodology}    
	
As mentioned above the methodology of this literature analysis consisted of a few main steps.\\
	
\textbf{Step 1: Defining a protocol for the literature analysis}\\
	
The main target of this step was to define a protocol \cite{knopf2006doing} for this literature analysis, which was then used to search and determine the relevance and usefulness of research papers. The relevance and usefulness of research papers were determined based on a few main factors which have been described below. The research papers which were identified as highly relevant and useful were chosen to conduct this literature analysis.\\ 
	
\textbf{(1)	The keywords/phrases and other techniques used to search for relevant and useful research papers }


\begin{enumerate}	
	\item Consumer health search
	\item Consumer health information searching behaviour
	\item Retrieval models for consumer health search
	\item Strategies for improving user queries
	\item Problems for consumers when searching health information online	
\end{enumerate}	

	
The use of proper keywords or phrases for searching relevant and useful research papers was important because the retrieval of research papers via search engines was highly impacted by those keywords and phrases. Hence, all the keywords and phrases were carefully chosen to search for relevant and useful research papers. In addition to executing these queries, the links presented in Google Scholar, such as ‘Related articles’ and ‘Cited by’ were also used to retrieve relevant and useful research papers.\\
	
\textbf{(2)	Inclusion and exclusion criteria}\\ 
	
A few factors were taken into account when determining the usefulness and relevance of a research paper. The major area which was covered by this literature analysis was ‘Consumer Health Search’. Hence, this was a broader topic, this literature analysis specifically aimed at reviewing research papers which cover at least one of the main aspects, such as user search behaviours, retrieval models, strategies for improving user queries or problems for consumers when searching health information online of consumer health information search. Hence, all the research papers which were chosen for this literature analysis had to be directly relevant to at least one of the previously mentioned aspects. In addition, all the chosen research papers for this literature analysis had to be recent research papers. Therefore, this literature review only analysed research papers which were published 2005 onwards. Moreover, all the chosen research papers also had to be academic and peer-reviewed research papers.\\ 
	
\textbf{(3)	Search services used}\\
	
Mainly two services which are the Google Scholar and QUT library databases were used in this literature review project to retrieve relevant and useful research papers. The main reasons for choosing these search services to search for useful and relevant research papers were because these services were able to retrieve information sources which contained highly reliable (academic and peer-reviewed) and complete information on prior research work.\\   
	
\textbf{Step 2: Searching for relevant research papers by issuing the queries and analysing those research papers} \\ 
	
The main goal of this step was to issue queries as defined in step 1 or use links, such as ‘Related articles’ and ‘Cited by’ on Google Scholar website to retrieve recent research papers that contained information on at least one of the aspects, such as user search behaviours, retrieval models, strategies for improving use queries and problems for consumers when searching for health information online. The next step was to read and analyse the chosen research papers to gather knowledge and information they have contributed to different aspects of consumer health information search as previously mentioned. It was crucial to analyse and understand existing research work related to consumer health search, because the foundation of this literature analysis was formed based on prior related research work. In addition, this literature analysis also aimed at covering approximately five to ten related research papers per week, depending on the length of each research paper. \\            
	
\textbf{Step 3: Preparing an annotated bibliography based on all the covered research papers}\\ 
	
In order to analyse all the chosen research papers an annotated bibliography \cite{unc} was prepared and used as the analysis technique for this literature review. The tool TeXstudio which is a LaTeX editor was used to prepare this annotated bibliography. Annotations were written for each of the research paper covered in this literature analysis and then all the annotations were included in the final annotated bibliography. Each annotation consists of factors, such as the aim of the study, the methodology which has been used, key findings of the study and the importance of those findings as presented in each research paper. \\
	
\textbf{Step 4: Preparing a table of topics summarising the claims in each research paper}\\
	
At this step, this project aimed at preparing an excel spreadsheet as a table of topics using the annotations written for each of the research paper in step 3. Hence, this table of topics included all the aspects (user search behaviours and retrieval models) covered in each of the chosen research papers, as a categorization of the themes and summaries of their claims. Most importantly these findings then were analysed to identify any patterns within each aspect. Therefore, this categorization and the analysis of patterns was the basis of the literature review conducted as the final step of this project.  Hence, this table of topics which includes the summarises of the claims of each research paper was really important because the content of the final literature review depended on the knowledge and information gained in this analysis.\\ 
	
\textbf{Step 5: Preparing the final literature review by synthesizing all the covered topics (aspects), methods, findings in each research paper, and by highlighting any uncovered aspects}\\
	
This was the final and the main step of this literature review project. In this step a literature review was prepared with the use of knowledge and information which was gained by conducting a thorough analysis in the previous step. Therefore, the final literature analysis most importantly presented all the identified patterns within each covered aspect by highlighting the corresponding findings from each research paper. Hence, this literature analysis can also be seen as a synthesis of the key findings of each research paper which were categorized according to their common features. In addition, this literature review has also emphasized a few uncovered aspects of consumer health information search by the chosen research papers. The final literature analysis was prepared and presented by following the literature review guidelines presented in the paper ‘Writing Integrative Literature Reviews: Guidelines and Examples' \cite{torraco2005writing}. Moreover, the tool TeXstudio which is a LaTeX editor was also used to prepare the literature review and the final report. \\
	
\subsection{Project Management Approach} 
	
The Dynamic Systems Development Method (DSDM) was the project management methodology which was used to complete this literature review project. The main reason for using this project management approach was because DSDM is an agile project management approach \cite{agile}. With the use of this project management approach, the main deliverables of the project, such as the annotated bibliography, the table of topics and the final literature analysis were implemented as increments. Most importantly, this methodology provided room to obtain continuous feedback from the supervisor and to identify any potential issues or risks associated with the project before it was too late. Therefore, I was able to fix any identified issue as soon as they were identified and then I continued with the rest of the project. In addition, in situations where there was a possibility of timeline slippage, it was possible to adjust the number of research papers read in each week, so as to align project tasks with their deadlines. For example, in situations where time did not permit to read at least five papers in one week, it was possible to read 4 papers in that week and read an extra paper in the following week. Furthermore, another importance of adhering to DSDM as an agile project management approach was its assurance to deliver the project with all the essential requirements in it. Hence, the use of an agile project management methodology, such as DSDM guaranteed a high-level of quality for the final deliverables and it also assisted in delivering them on time. 

\section{Summaries of the Research Papers} 

In this section, summaries are presented for all the research papers which were covered in this literature review project in a chronological order. 

\vspace{0.5cm}

\begin{enumerate}
	\renewcommand\labelenumi{\bfseries\theenumi .}
	\item { \textbf{Laurel Graham, Tony Tse, and Alla Keselman. Exploring user navigation during online health information seeking. In \textit{AMIA Annual Symposium Proceedings,} volume 2006, page 299. American Medical Informatics Association, 2006.}}
	
	This study investigated user navigation behaviour while seeking for online health-related information. ClinicalTrials.gov web site\textquotesingle s log data was collected and used to extract information, such as client IP addresses, information about search sessions and web page requests made by those clients. In addition to that other information, such as page reviews, referral frequencies, page transition frequencies, navigation path frequencies, click stream data etc. was also used to assess user navigation patterns while searching for online health information. Parallel to the analysis of web log data, a pilot user study was conducted in which participants were presented with two hypothetical scenarios. They were assigned to either one of the scenarios to perform the search and were allowed to use any online resource as per their choice. The findings of this study revealed that, the majority of the users directly start by accessing low-level pages (View Study) via web search engines or consumer health sites and visible local maps on web sites are able to retain users for a longer time.
	
	\textbf{\item {Qing T Zeng, Jonathan Crowell, Robert M Plovnick, Eunjung Kim, Long Ngo, and Emily Dibble. Assisting consumer health information retrieval with query recommendations. \textit{Journal of the American Medical Informatics Association,} 13(1):80-90, 2006.}}
	
	This study investigated how the use of a novel system known as \textquotesingle the Health Information Query Assistant (HIQuA)\textquotesingle  impact on consumers\textquotesingle health information retrieval by providing supplementary query terms, which are related to consumers\textquotesingle initial queries, to construct more effective queries. Consumer usage patterns, medical vocabularies and concept co-occurrences in medical publications were used to identify related terms and then they were suggested to consumers. The \textquotesingle Unified Medical Language System (UMLS)\textquotesingle was used to map user queries with medical concepts and to identify other concepts which are associated with the initial concept. The relevance was calculated based on the frequency of concept co-occurrences in publications or frequency of relation occurrences in different medical vocabularies. The participants were instructed to perform two pre-defined health search tasks. 50\% of the participants were assigned with task 1 and the other 50\% were assigned with task 2. Then 50\% of the participants were provided with the query recommendation function and the rest of them were not provided with it. In addition, each participant was also instructed to perform one self-defined health search task and also to search for a health-related question according to their task definition. The findings of this study revealed that, the use of query recommendations, increases the number of successful queries, the impact of query recommendations on successfully completing a predefined health search task and the overall user satisfaction is not significant, and query suggestions provided by the Health Information Query Assistant system are beneficial for consumers when searching for health-related information.
	
	\textbf{\item {\textbf{Elaine G Toms and Celeste Latter. How consumers search for health information.\textit{Health informatics journal,} 13(3):223-235, 2007.}}}
	
	This study investigated consumer health information search behaviour online. The approach of this study was to provide participants with a set of directory categories by appending them to Google, so that there is an additional scan option available to them. Altogether four tasks were presented to participants. The first two tasks were specified for the participants and the participants were able to personalise the second two tasks. Each participant was assigned to one of the four tasks. Participants were asked to mark the web pages which they thought as relevant to the health search task they performed and Judges were recruited to assess those marked web pages. The findings of this study revealed that, users\textquotesingle domain knowledge, the way they assess credibility and familiarity with the health topic influence keyword searches, categories are not that useful when performing health information searches because consumers are not familiar with what a category contains and they do not like the fact that some categories having endless sets of category levels, the information design of the web pages influences the time taken to select from the results pages, users tend to make erroneous decisions by only considering the web page appearances, credibility, reliability and trustworthiness are important factors when selecting web pages from results lists, formulation of good queries and retrieval of a results list with an appropriate design and standards are both important for a successful health search, consumer health information search (query formulation and efficient selection of appropriate results) still is a challenging task to many of the consumers, and both information design and search engine technology are important when building better consumer health information systems.
	
	\textbf{\item {Elizabeth Sillence, Pam Briggs, Peter Richard Harris, and Lesley Fishwick. How do patients evaluate and make use of online health information? \textit{Social science \& medicine,} 64(9):1853-1862, 2007.}}
	
	This study investigated how menopausal women search the Internet for information and medical advice on hormone replacement therapy (HRT) and how they determine which sites/information to trust. A stepwise model which represented users\textquotesingle gradual development of trust was used for this investigation. The authors also evaluated how Internet advice influence decisions taken by consumers and their communication with physicians. The necessary data and information was collected by asking participants to accomplish tasks in three separate phases. In the first phase participants searched for relevant health information, in the second phase they demonstrated Internet usage and the utilization of information from various resources over a particular period of time and in the third phase their opinions of the usefulness of online menopause-related health information for decision making and communication with physicians were collected. The findings of this study revealed that, both design factors and content factors were considered when accepting or rejecting a website, credibility and personalized content were important when assessing the trustworthiness of a website, both online and offline information and advice are integrated when making decisions, online information is used to improve communications with physicians and information written by similar people as consumers is considered as trustworthy. 
	
	\textbf{\item{Alla Keselman, Allen C Browne, and David R Kaufman. Consumer health information seeking as hypothesis testing. \textit{Journal of the American Medical Informatics Association,} 15(4):484-495, 2008.}}
	
	This study investigated the most common patterns consumers follow when searching health information, depending on their initial theories, search strategies and comprehension. A framework with two perspectives, such as hypothesis testing perspective and Human Computer Interaction (HCI) perspective was used to obtain further insight regarding difficulties faced by consumers when searching and understanding health information. The participants of this study were presented with a hypothetical scenario which described symptoms of stable angina and were asked to seek information about this disease. Consumers\textquotesingle understandability of this disease was compared with a reference model built in association with \textquotesingle stable angina\textquotesingle. The findings of this study revealed that, despite their web experience, users with incorrect and insufficient domain knowledge tend to search health information on unrelated sites and are unsuccessful in evaluating retrieved information, consumers usually seek information to confirm their own incorrect initial assumptions, imprecise health search queries and poor web resource configurations are two main reasons for unsuccessful health information searches, users with better online search skills are able to perform efficient searches, resource knowledge is important for navigational actions, users with a higher education level tend to judge the authoritativeness of a source, features of website interfaces, such as the absence of relating lay terms and professional terms in the index and organization of search results lists influence users\textquotesingle search process, and laypeople with imprecise domain knowledge can be supported via information portals, individual websites and education tools. 
	
	\textbf{\item {Angela Ryan and Sue Wilson. Internet healthcare: do self-diagnosis sites do more harm than good? \textit{Expert opinion on drug safety,} 7(3):227-229, 2008.}}
	
	This paper consisted of information about an evaluation conducted to assess self-diagnosis information provided in various types of self-diagnosis websites which usually contain information, such as diagnosed conditions, possible diagnoses for different symptoms and advices to aid consumers in deciding whether to self-treat themselves or consult a doctor. The findings of this evaluation revealed that, consumers face several problems when using self-diagnosis websites\textquotesingle information to diagnose symptoms. They are experiencing anxiety after reading inaccurate or false diagnoses information about serious conditions, unreliability of available information with no involvement of a health professional, making users buy prescribed drugs from Internet pharmacies, and ignoring professional assistance and diagnosis by believing the false reassurance provided by the self-diagnosis websites. 
	
	\textbf{\item{Efthimis N Efthimiadis. How students search for consumer health information on the web. In \textit{System Sciences, 2009. HICSS09. 42nd Hawaii International Conference on,} pages 18. IEEE, 2009.}}
	
	This study investigated the search behaviours exhibited by students (undergraduates and graduates) while searching for consumer health information online. Participants were instructed to search and find answers for four health associated questions. Their search processes and behaviours (starting point of the searches, selection of search results, evaluation of selected search results etc.) while performing the health search tasks along with other background information, such as demographic information, educational information etc. were collected in order to conduct the investigation. The findings of this study revealed that, search engines were the most popular starting point of the searches, search duration and the count of webpages visited increases with the difficulty of the topic, undergraduates were more satisfied with the search results compared to the graduates, search planning is challenging for more complex tasks than for the easier tasks, and overall, consumers have problems in formulating queries. 	 
	
	\textbf{\item {Laurence Alpay, John Verhoef, Bo Xie, Dov Teeni, and JHM ZwetslootSchonk. Current challenge in consumer health informatics: Bridging the gap between access to information and information understanding. \textit{Biomedical informatics insights,} 2:BIIS2223, 2009.}} 
	
	This study investigated how tailoring of health information (contextualized) impact on the retrieval and understanding of that health information. In this study participants were randomly assigned to two groups in order to investigate their understandability of the health information. One group was presented with contextualized information and the other group did not receive any contextualized information. The findings of this study revealed that, the use of contextualization significantly increases lay users\textquotesingle understanding of health information by raising users\textquotesingle health literacy, contextualization helps to improve users\textquotesingle understanding of different health concepts by reducing communication complexity, users\textquotesingle cognitive style (the way they organize, filter, transform and process information) is also a significant factor in thoroughly understanding health content and consumers also use their own contexts (social and psychological) as a support to understand health information well. However, gaps between informational contexts provided to users by health information websites and users\textquotesingle personal contexts that they use to enhance their understandability of health information were identified. 	
	
	\textbf{\item {Yan Zhang, Peiling Wang, Amy Heaton, and Heidi Winkler. Health information searching behavior in medlineplus and the impact of tasks. In \textit{Proceedings of the 2nd ACM SIGHIT International Health Informatics Symposium,} pages 641-650. ACM, 2012.}}
	
	This study investigated health-related information search behaviour of consumers in MedlinePlus which consisted of a basic search engine, and the impact of the number of concepts involved in the search behaviours. Three health information search tasks were presented to each participant. These health information search tasks included searching information about marijuana, searching information about two types of diabetes and hypertension, and searching information about liver and kidney. Participants health information search behaviour and patterns were observed and data was collected based on these behaviours. The findings of this study revealed that, session length of a search process increases with the complexity of the health information search task, most of the query reformations are performed by conducting conceptual changes, query re-executions are performed as query iterations, users with higher familiarity with the search task, also search for other concepts associated with the main concept, users prefer to use natural language while searching, both searching and browsing strategies are used when performing more complex search tasks, task complexity also influences interaction strategies and patterns, and resources, such as Encyclopedia and dictionary are used to gain a preliminary understanding when performing more complex health search tasks.   
	
	\textbf{\item {Ira Puspitasari, Roberto Legaspi, and Masayuki Numao. Characterizing the effect of consumer familiarity with health topics on health information seeking behavior. In \textit{The 27th Annual Conference of the Japanese Society for Artificial Intelligence,} volume 27, pages 15, 2013.}} 
	
	This study investigated the impact of health topic familiarity, on consumers\textquotesingle health information seeking behaviour. Participants were instructed to perform four health-related search tasks. The first one was an exploratory task which was associated with searching reasons for having a swelled red big toe on the right leg. The second task was a specified task associated with searching information on rheumatoid arthritis and osteoporosis. The third task was a specified task associated with searching information on hydrochlorothiazide. The fourth task was a personalized task associated with searching information on two medical treatments for any particular health concern, such as a disease or a symptom. Rated information related to search performance, cognitive effort required for each search task, the complexity and the familiarity of the task, characteristics of search behaviour (query keywords and query reformulation patterns) etc. was collected. The findings of this study revealed that, familiarity with the health topics influence users\textquotesingle search behaviour. It was also found that consumers use more specific and more varied vocabulary/query keywords when performing more familiar search tasks and different query reformulation patterns, such as dynamic reformulation pattern for unfamiliar tasks and parallel reformulation pattern for familiar tasks are being used by the consumers depending on the familiarity levels of the tasks. 
	
	\textbf{\item {Ramona Broussard and Yan Zhang. Seeking treatment options: consumers search behaviors and cognitive activities. In \textit{Proceedings of the Association for Information Science and Technology,} volume 50, pages 1-10. Wiley Online Library, 2013.}}
	
	This study investigated consumers exploration of treatment options in both behavioural and cognitive perspectives. Two interfaces were used in this study. The first one was a simple web-based search engine interface with a basic search box and the results were presented as a ranked list. The second one was a \textquotesingle Scatter/Gather enabled\textquotesingle search interface with a basic search box, but the results were grouped into a number of clusters. 50\% of the participants were assigned to the first interface and the rest of the participants were assigned to the second interface. Then they were asked to search treatment options for migraines using these interfaces. The findings of this study revealed that, consumers submit short queries with inaccurately spelled keywords and select results from the very first page when searching for treatment options online, both medical-specific and general web sites are being used when performing the searches, users visit websites named with the medical condition searched, users start the search process with general concepts and move towards different treatments/ aspects of the treatments, search results are selected based on rankings and familiarity on top of trustworthiness, quality and usefulness of the information, and users also search the web to obtain confirming and novel information.
	
	\textbf{\item {Rong Hu, Kun Lu, and Soohyung Joo. Effects of topic familiarity and search skills on query reformulation behavior. In \textit{Proceedings of the Association for Information Science and Technology,} volume 50, pages 1-9. Wiley Online Library, 2013.}}
	
	This study investigated how query reformulation will be influenced by topic familiarity/ domain knowledge and search skills when searching for health-related information. A health information retrieval system with a search interface was used for this study. Six search topics were selected from the medical information database MEDLINE to be used in the user studies. Participants were instructed to perform two pre-experiment search tasks by finding the definitions of medical terms by picking one from each search topic and by seeking for relationships between medical concepts and to find answers to each search topic. The use of search system was not allowed while answering the questions. Each participant was also instructed to perform three tasks using the health information search system. The findings of this study revealed that, topic familiarity influences the number of query reformulations performed and the time taken to perform each query reformulation type, users with high levels of topic familiarity initiate their search processes with specific terms and move towards using more general terms, they also perform a lower number of query reformulations per session, users with lower topic familiarity tend to do more mistakes when issuing query terms, and start search process with more general terms and move towards more specific terms, comparatively these users perform a higher number of query reformulations per session, users with more search skills change query format more frequently to obtain relevant results and make less mistakes when issuing queries, users with less search skills are unable to issue efficient queries to obtain relevant results and make more mistakes when issuing queries, and overall, users spend comparatively longer time when performing \textquotesingle Error\textquotesingle (correcting queries) query reformulation type.  
	
	\textbf{\item {Isabelle Stanton, Samuel Ieong, and Nina Mishra. Circumlocution in diagnostic medical queries. In \textit{Proceedings of the 37th international ACM SIGIR conference on Research \& development in information retrieval,} pages 133-142. ACM, 2014.}}
	
	The objective of this study was to find the corresponding professional medical terms for medical signs and symptoms that are referred by user generated \textquotesingle colloquial health search queries\textquotesingle. The method authors used to obtain colloquial variants for a medical symptom was a reversed approach of crowd sourcing in conjunction with images and videos. User\textquotesingle s colloquial queries were identified with the use of sources, such as Encyclopedia, synonyms, paraphrases etc. Then each query and its associated symptom were converted to two vectors and the cosine similarity measure was obtained to compare the similarity between the user query and its associated symptom. The findings of this study revealed that, users are able to generate more successful queries to search for the name of a symptom, when they are presented with images and videos of a medical symptom, the features, such as Encyclopedia, synonyms and paraphrases are useful when attempting to match a user query with its corresponding symptom name and an approach such as this has the ability to improve the performance when matching user queries with symptom names compared to random guessing. 
	
	\textbf{\item {Guido Zuccon, Bevan Koopman, and Joao Palotti. Diagnose this if you can. In \textit{European on Information Retrieval,} pages 562-567. Springer, 2015.}}
	
	This study investigated the effectiveness of current web search engines in retrieving relevant and useful information for diagnostic medical queries written in a circumlocutory manner. The authors instructed participants to classify the relevance of search results (retrieved using the circumlocutory diagnostic medical queries) by asking them to assign labels, such as \textquotesingle not relevant\textquotesingle, \textquotesingle on topic but unreliable\textquotesingle, \textquotesingle somewhat relevant\textquotesingle and \textquotesingle highly relevant\textquotesingle to the retrieved search results based on the content of each search result. The findings of this study revealed that, current retrieval techniques are not able to retrieve relevant health information when queries which describe symptoms in a circumlocutory manner are issued and there is a higher possibility of users retrieving misleading advice and irrelevant information when searching for self-diagnosis information which will also lead to inaccurate self-diagnosis decisions and ultimately cause them harm. It was also observed that on average only about 4 to 5 results out of the first 10 results are considered as containing relevant information that was helpful to people, if users seek highly relevant information, only 3 out of the first 10 results contained highly useful information to self-diagnose medical symptoms,\textquotesingle Somewhat relevant\textquotesingle  documents contain information not only about the relevant symptom but also about other symptoms as well, \textquotesingle Highly relevant\textquotesingle documents contain information focused on the symptoms searched by users, such as descriptions of the symptoms and causes of those symptoms which are also supported by photographic material and \textquotesingle On topic but unreliable\textquotesingle  documents were considered as irrelevant for this evaluation. 
	
	\textbf{\item {Anushia Inthiran, Saadat M Alhashmi, and Pervaiz K Ahmed. Describing health querying behavior. In \textit{Proceedings of the 2nd SIGIR workshop on Medical Information Retrieval (MedIR),} 2016.}}
	
	This study investigated the querying behaviour of laypeople when searching for health information. The authors mainly aimed at evaluating query formulation and reformulation patterns used when performing health tasks with different levels of complexity. Participants were instructed to perform searches based on two simulated situations. These simulated situations were based on Clinical scenarios because laypeople tend to perform such health searches. The searches were also performed on MedlinePlus. The first simulated situation was related to finding reasons for kidney enlargement and the second simulated situation was related to finding treatment options for a swollen neck and the reasons for experiencing such a condition. Participants ratings of the task complexity, information about querying patterns and information about query reformulation patterns was also collected. The findings of this study revealed that, when performing easy tasks consumers start with more specific queries and move towards more broader aspects. Two query reformulation patterns, such as moving from switching topic to specialization and moving from specialization to parallel movement were used. When performing neutral tasks consumers start with broader topics and move towards more narrower aspects.  Two query reformulation patterns, such as moving from parallel movement to specialization and switching topics were used. When performing difficult tasks consumers start with specific queries, then move to more broader aspects, and again move back to specific queries at the end of the search tasks. Only one query reformulation pattern which is switching topics was used. 
	
	\textbf{\item {Carla Teixeira Lopes and Tiago Almeida Fernandes. Health suggestions: A chrome extension to help laypersons search for health information. In \textit{International Conference of the Cross-Language Evaluation Forum for European Languages,} pages 241-246. Springer, 2016.}}
	
	This study investigated how providing health query suggestions to consumers will be useful in successful health information retrieval. The approach followed in this study was to implement health query suggestions in Google Chrome as a Google Chrome extension. The suggestions were provided to users with the help of a health suggestion query panel and this was presented to participants when matches were identified between health suggestions and user queries. Two modules, such as a \textquotesingle Suggestion engine\textquotesingle and a \textquotesingle Login engine\textquotesingle  were used in the system. Suggestion engine generated suggestions based on Consumer Health Vocabulary (CHV) which contains links between everyday health-related terms and technical terms, and Login engine was used to study user health search behaviour by tracking their search process. Participants were asked to perform four tasks by formulating and issuing three queries per each task, and then they were asked to select the most reliable search results out of the top ten search results of each query. Prior conducting these health search tasks, the participants were divided into two groups and the first group received assistance via health suggestions, but the second group was unassisted. The findings of this study revealed that, participants\textquotesingle willingness to accept the health query suggestions from the tool (Google Chrome extension) was high, multilingual and multi-terminology suggestions are useful to retrieve more relevant documents, and query suggestions also lead to more successful searches.    
	
	\textbf{\item {D Thenmozhi, P Mirunalini, and Chandrabose Aravindan. Decision tree approach for consumer health information search. In \textit{FIRE (Working Notes),} pages 221-225, 2016.}}
	
	This study investigated how to determine whether sentences of a document is relevant, with the use of a consumer generated health search query and a document related to that user query. The approach of this study was to use a machine learning technique (a decision tree) to classify retrieved health-related information as relevant or irrelevant based on the issued query. Two variations of the decision tree, such as a decision tree with feature selection and a decision tree without feature selection were used to compare results of five queries. Training and test data was available for each of these five queries separately. The findings of this study revealed that, method with feature selection performs well and has a higher accuracy when determining the relevance of the search results compared to the method without feature selection and the feature selection model can be used to filter out irrelevant documents prior presenting search results to consumers, so as to aid them in retrieving more relevant health related documents. 
	
	\textbf{\item {Luca Soldaini, Andrew Yates, Elad Yom-Tov, Ophir Frieder, and Nazli Goharian. Enhancing web search in the medical domain via query clarification. \textit{Information Retrieval Journal,} 19(1-2):149-173, 2016.}}
	
	This study investigated the usefulness of query clarification when bridging the gap between lay terms and expert terms. As the methodology, the authors used three different types of synonym mappings, such as Behavioural mappings (mapping user expressions to medical symptoms), MedSyn mappings (mappings focused on diseases and symptoms, and removed terms which were not related UMLS semantic types) and DBpedia mappings (mapping lay terms to expert terminology based on Wikipedia redirect pages) to automatically improve user queries. In order to compare the performance of each synonym mapping, participants were asked to issue queries which were clarified with each synonym mapping and to answer some multiple-choice questions which were prepared based on these clarified queries. The findings of this study revealed that, Behavioural synonym mapping is the best performing synonym mapping because users are more likely to correctly answer questions with the use of results retrieved using queries clarified with Behavioural mapping, Behavioural mapping is the least preferred mapping for incorrectly answered questions, lay people prefer query clarifications more than experts because lay people are able to retrieve  more useful information to correctly answer health-related questions, the difference in success rates between lay people and experts is not significant, there is a strong correlation between lay people and experts in terms of the success rate of each health information search task because some questions were more difficult for both groups when compared to other questions, the number of correctly answered medical questions increased with the trustworthiness of the web pages used to answer the question, clarified queries are able to retrieve more trustworthy resources as results, the 'simple multinomial logistic regression classifier' performs the best when compared to all individual synonym mappings and unclarified queries, 
	the use of most suitable query clarification method (synonym mapping) to clarify each query has the ability to further improve the query clarification process 	  and implicit query clarifications are also highly useful and most importantly, users do not need to know the correct medical terminologies which are being added to user queries while clarifying them.	
	
	\textbf{\item {Ricardo Silva and Carla Lopes. The effectiveness of query expansion when searching for health related content: Infolab at clef ehealth 2016. In \textit{CLEF (Working Notes),} pages 130-142, 2016.}}
	
	This study investigated how query expansion (supplementing the original query with additional terms) will aid laypeople in improving initial queries and then the overall retrieval performance. Query generators were used to create initial queries based on consumer health posts. A set of different sources and methods, such as Medical Text Indexer, Wikipedia, MedlinePlus etc. were used to pick terms to be added to the initial queries. Three readability matrices, such as SMOG, FOG and Flesch-Kinciad and relevance scores were used to re-rank the documents retrieved using expanded queries. However, clear conclusions regarding the impact of query expansion on the improvement of initial queries and the overall retrieval performance were hard to make because of the lack of information about relevance and readability assessment for the test collection used to evaluate the retrieved documents. 
	
	\textbf{\item {Frances A Pogacar, Amira Ghenai, Mark D Smucker, and Charles LA Clarke. The positive and negative influence of search results on peoples decisions about the efficacy of medical treatments. In \textit{Proceedings of the ACM SIGIR International Conference on Theory of Information Retrieval,} pages 209-216. ACM, 2017.}}
	
	This study investigated the impact of search results on consumers\textquotesingle decisions of the effectiveness of medical treatments. Each participant was presented with search results pages which were either biased towards correct or incorrect search results and were asked to determine the effectiveness of ten medical treatments with the help of these search results. Another set of participants was not provided with search results and they were also asked to determine the effectiveness of two of the medical treatments with their own knowledge.  Participants were asked to categorise the medical treatments by labelling these treatments as \textquotesingle help\textquotesingle, \textquotesingle inconclusive\textquotesingle or as \textquotesingle does not help\textquotesingle. The findings of this study revealed that, when search results are bias towards correct information users are able to more accurately determine the efficacy of health treatments when compared to determining efficacy without any search results, when search results are bias towards inaccurate information users\textquotesingle make more harmful decisions which are even worse than the decisions made with no search results, this impact of search results bias on the number of accurate decisions and inaccurate decisions is statistically significant, users are less likely to label a health treatment as unhelpful because they are willing to see positive information in search results, users with more prior knowledge are less likely to classify a medical treatment as inconclusive and therefore, make more inaccurate and harmful decisions when determining the efficacy of health treatments, users who are less confident with the efficacy of the treatments labelled them as inconclusive, the top ranked search result (Rank 1) is important because users are more likely to click on that search result, the more participants interact with the search results the more accurate decisions they make, and search results notably affect people\textquotesingle s decisions about the effectiveness of medical treatments and have a higher potential to both aid and mislead users.    
	
	\textbf{\item {Zequn Ye, Jacek Gwizdka, Carla Teixeira Lopes, and Yan Zhang. Towards understanding consumers quality evaluation of online health information: A case study. In \textit{Proceedings of the Association for Information Science and Technology,} volume 54, pages 838-839. Wiley Online Library, 2017.} }
	
	This case study investigated how consumers evaluate the quality of online health information with the use of \textquotesingle eye-tracking\textquotesingle techniques. The participants were presented with five predefined tasks related to health information search and three pre-selected web pages per each task. They were asked to analyse and determine whether they would recommend those web pages to their family or friends. The findings of this study revealed that, consumers with lower eHealth literacy and familiarity with health topics spend a longer time to evaluate the quality of search results and exhibit F-shaped fixation patterns focused on the main text content of a webpage and consumers with higher eHealth literacy and familiarity with health topics spend much lesser time to evaluate the quality of search results and exhibit atypical fixation patterns scattered all-over the webpage.   
	
	\textbf{\item {Guido Zuccon, Bevan Koopman, and Jimmy. Choices in knowledge-base retrieval for consumer health search. In \textit{European Conference on Information Retrieval,} pages 72-85. Springer, 2018.}}
	
	This study investigated how to overcome consumer health search problems by expanding health queries, so as to contain more effective query terms. Tow knowledge bases, one as a specialised knowledge base (UMLS) and the other as a general knowledge base (Wikipedia) were used to expand consumer health queries. An \textquotesingle Entity Query Feature Expansion\textquotesingle model was used to retrieve health information from the two knowledge bases. Relevance feedback has also been used for the query expansion process. The impact of the choice in knowledge-based retrieval on query expansion and retrieval effectiveness of consumer health search was examined. The findings of this study revealed that, relevance feedback improves the effectiveness of query expansion, and overall, UMLS KB expanded more queries and retrieved a better set of documents compared to Wikipedia KB. Therefore, KB query expansion with UMLS was more effective than with Wikipedia. 	   
	
\end{enumerate}

\pagebreak
	
\section{Literature Review Results}

In this section, the literature review results are presented according to the identified 7 sub aspects of consumer health information search.   

\subsection{User Search Behaviours}

\begin{center}
	\begin{longtable}{||p{4cm} | p{10cm}||} 
		\hline
		\textbf{Name of the paper} & \textbf{Findings}\\ [0.5ex] 
		\hline\hline
		Seeking Treatment Options: Consumers’ Search Behaviours and Cognitive Activities (Broussard et al., 2013) & 1) Consumers submit short queries with inaccurately spelled keywords and select results from the very first page when searching for treatment options online; 2) Both medical-specific and general web sites are being used when performing the searches; 3) Users visit websites named with the medical condition searched; 4) Users start the search process with general concepts and move towards different treatments/ aspects of the treatments; 5) Search results are selected based on rankings and familiarity on top of trustworthiness, quality and usefulness of the information; 6) Users also search to obtain confirming and novel information \\ \hline Consumer Health Information Seeking as Hypothesis Testing (Keselman et al., 2006) & 1) Despite their web experience, users with incorrect and imprecise domain knowledge tend to search health information on irrelevant sites and are unsuccessful in evaluating retrieved information; 2) Consumers/ laypeople usually seek information to confirm their own incorrect initial assumptions; 3) Imprecise health search queries and poor web resource configurations are two main reasons for unsuccessful health information searches; 4) Users with better online search skills are able to perform efficient searches; 5) Resource knowledge is important for navigational actions; 6) Users with a higher education level tend to judge the authoritativeness of a source; 7) Features of website interfaces, such as not explicitly relating lay and professional terms and organization of search results lists influence users’ search process; 8) Laypeople with imprecise domain knowledge can be supported via information portals, individual websites and education tools \\ \hline Health Information Searching Behavior in MedlinePlus and the Impact of Tasks (Zhang et al., 2012) & 1) Session length of a search process increases with the complexity of the health information search task; 2) Most of the query reformations are performed by conducting conceptual changes; 3) Query re-executions are performed as query iterations; 4) Users with higher familiarity with the search task, also search for other concepts associated with the main concept; 5) Users prefer to use natural language while searching; 6) Both searching and browsing strategies are used when performing more complex tasks; 7) Task complexity also influences interaction strategies and patterns; 8) Resources, such as Encyclopedia and dictionary are used to gain a preliminary understanding when performing more complex health search tasks; 9) All these facts should be considered when designing consumer health information websites \\ \hline How consumers search for health information (Toms et al., 2007) & 1) Users’ domain knowledge, the way they assess credibility and familiarity with the health topic influence keyword searches; 2) Categories are not that useful when performing health information searches because consumers are not familiar with what a category contains and they do not like the fact that some categories having endless sets of category levels; 3) The information design of the web pages influences the time taken to select from the results pages; 4) Users tend to make erroneous decisions by only considering the web page appearances; 5) Credibility, reliability and trustworthiness are important factors when selecting web pages from results lists; 6) Formulation of good queries and retrieval of a results list with an appropriate design and standards are both important for a successful health search; 7) Consumer health information search (query formulation and efficient selection of appropriate results) still is a challenging task to many of the consumers; 8) Both information design and search engine technology are important when building better consumer health information systems; 9)  All these factors should be considered when designing consumer health information systems \\ \hline Towards Understanding Consumers’ Quality Evaluation of Online Health Information: A Case Study (Ye et al., 2017) & 1) Consumers with lower eHealth literacy and familiarity with health topics spend a longer time to evaluate the quality of search results and exhibit F-shaped fixation patterns focused on the main text content of a webpage; 2) Consumers with higher eHealth literacy and  familiarity with health topics spend much lesser time to evaluate the quality of search results and exhibit atypical fixation patterns scattered all-over the webpage \\ \hline Exploring User Navigation during Online Health Information Seeking (Graham et al., 2006) & 1) Majority of the users directly enter by accessing low-level pages (View Study) via web search engines or consumer health sites, and therefore, it is recommended to place links to background information and other search features on low-level pages; 2) Visible local maps on web sites are able to retain users \\ \hline How do patients evaluate and make use of online health information? (Sillence et al., 2007) & 1) Both design factors and content factors were considered when accepting or rejecting a website; 2) Credibility and personalized content were important when assessing the trustworthiness of a website;  3) Both online and offline information and advice are integrated when making decisions;  4) Online information is used to improve communications with physicians; 5) Information written by similar people as consumers is considered as trustworthy \\ \hline How Students Search for Consumer Health Information on the Web (Efthimiadis, 2009) & 1) Search engines was the most popular starting point of the searches; 2) Duration of the search and the count of webpages visited increases with the difficulty of the topic; 3) Undergraduates were more satisfied with the search results compared to the graduates; 4) Search planning is challenging for more complex tasks than for the easier tasks; 5) Overall, consumers have problems in formulating queries \\ \hline Characterizing the Effect of Consumer Familiarity with Health Topics on Health Information Seeking Behavior (Puspitasari et al., 2013) & The familiarity with the health topics influence consumers’ searching behavior; 1) Consumers use more specific and more varied vocabulary/query keywords when performing more familiar search tasks; 2) Different query reformulation patterns are being used by the consumers depending on the familiarity levels of the tasks, such as dynamic reformulation pattern for unfamiliar tasks and parallel reformulation pattern for familair tasks \\ \hline Describing Health Querying Behavior (Inthiran et al., 2016) & 1) When performing easy tasks consumers start with more specific queries and move to more broader aspects. Two query reformulation patterns, such as moving from switching topic to specialization and moving from specialization to paralllel movement are being used; 2) When performing neutral tasks consumers start with broader topics and move to more narrower aspects.  Two query reformulation patterns, such as moving from parallel movement to specialization and switching topics are being used ; 3) When performing difficult tasks consumers start with specific queries, then move to more broader aspects, and again move back to specific queries at the end of the search tasks. Only one query reformulation pattern which is switching topics is being used \\ \hline Effects of Topic Familiarity and Search Skills on Query Reformulation Behavior (Hu et al., 2013) & 1) Topic familiarity influences the number of query reformulations and time taken to perform each query reformulation type; 2) Users with high levels of topic familiarity initiate their search processes with specific terms and move towards using more general terms; 3) They also perform a lower number of query reformulations per session; 4) Users with lower topic familiarity tend to do more mistakes when issuing query terms, and start search process with more general terms and move towards more specific terms; 5) Comparatively these users perform a higher number of query reformulations per session; 6) Users with more search skills change query format more frequently to obatin relevant results and make less mistakes when issuing queries; 7) Users with less search skills are unable to issue efficient queries to obtain relevant results and make more mistakes when issuing queries; 8) Overall, users spend comparatively longer time when performing 'Error' (correcting queries) query reformulation type \\      		
		\hline
	\end{longtable}	
	
\end{center} 


\textbf{1. Complexity of health information search tasks} 
\\

The first main aspect of user search behaviour is the complexity of health information search tasks. The term \textquotesingle task complexity\textquotesingle represents the difficulty, mental effort required and the time needed to successfully perform and complete a health information search task. According to Zhang et al. and Efthimiadis \cite{zhang2012health,efthimiadis2009students} the session length of a search task and the total number of search results, for instance web pages, selected by users for viewing increases with the complexity of the health information search task. Zhang et al. \cite{zhang2012health} also state that, task complexity has the ability to influence interaction strategies because both searching and browsing strategies were used by consumers when performing more complex search tasks. As Zhang et al. \cite{zhang2012health} further mention, not only interaction strategies but also search patterns were influenced by the complexity of search tasks. This fact has also been supported by Inthiran et al. \cite{inthiran2016describing} stating that,  consumers start with specific queries, then move to more broader aspects, and again move back to specific queries at the end of search tasks, when performing difficult tasks. In addition to this, Inthiran et al. \cite{inthiran2016describing} also noticed that, the only query reformulation pattern used while performing complex tasks was \textquotesingle switching topics\textquotesingle. Furthermore, Zhang et al. and Sillence et al. \cite{zhang2012health,sillence2007patients} reported that, consumers use resources, such as Encyclopedia and dictionary to gain a preliminary understanding when performing more complex health search tasks and it is challenging to plan the search for more complex tasks. In contrast, Inthiran et al. \cite{inthiran2016describing} mentioned that, to accomplish much easier tasks, consumers started with more specific queries and moved towards more broader aspects, and to accomplish neutral tasks, consumers started with broader topics and moved towards more narrower aspects. According to Sillence et al. \cite{sillence2007patients} search planning was also identified as less challenging for easier health search tasks. Summary of how user search patterns varies with the task complexity is reported in Table 2. 

\begin{table}[b!]
	
	\includegraphics[width=1.0\textwidth]{Taskcomplexity.png}
	\caption{The influence of task complexity on user search patterns\label{tabel1}}
	
\end{table} 

\vspace{0.25cm}

\textbf{2. What do users search for, the purpose of the search and the use of information}
\\

The second aspect of user search behaviour focuses on the content users search for, purposes of the searches and the use of retrieved information. The observations are presented based on three different studies. Broussard et al. \cite{broussard2013seeking} have studied the purposes of consumer health information searches and have identified that, in addition to seeking information to gain general knowledge about different diseases and symptoms, treatment options, diagnostic information etc. users also tend to search health information to obtain confirming and novel information. However, according to Keselman et al. \cite{keselman2008consumer} some of these information searches were performed by consumers with the intention of confirming their own incorrect initial assumptions. Sillence et al. \cite{sillence2007patients} have stated that, online information is mainly used by consumers to improve communications with physicians. 

\vspace{0.25cm}

\textbf{3. Health search queries and query reformulations}\\


The third aspect of user search behaviour is the health search queries and query reformulations, performed by users when they search for health information. The impact of user queries on health searches, the reasons for unsuccessful health searches and different query reformulation patterns performed by users were identified based on five different investigations. According to Toms et al. \cite{toms2007consumers}, formulation of good queries and retrieval of well-designed, standard and reliable results lists are the most important factors for a successful health search. However, Keselman et al. \cite{keselman2008consumer} argue that, most of the consumers still issue imprecise health search queries and this has become one of the main reasons for unsuccessful health information searches. Inthiran et al. \cite{inthiran2016describing} noted that there are mainly two query reformulation patterns, such as moving from switching topic to specialization and moving from specialization to parallel movement which are being used by consumers when performing much easier tasks and also there are two main query reformulation patterns, such as moving from parallel movement to specialization and the use of switching topics which are being used for more difficult tasks. Hu et al. \cite{hu2013effects} mention that, users spend comparatively a longer time when performing 'Error' (correcting queries) query reformulation type and according to Zhang et al. \cite{zhang2012health}, most of the query reformations are performed by conducting conceptual changes. Zhang et al. \cite{zhang2012health} also state that, in addition to performing query reformulations, consumers also tend to perform query re-executions as query iterations. An example of a query iteration and a few examples of different query reformulation types are presented in Table 3.  

\begin{table}[t!]
	\includegraphics[width=1.0\textwidth]{queryreformulations.png}
	\caption{Examples of query reformulations and query iterations\label{tabel2}}
\end{table}
 
\vspace{0.25cm}

\textbf{4. Patterns of accessing resources}\\

The fourth aspect of user search behaviour is the different patterns followed by consumers when accessing resources. Basically, three main patterns were identified based on three different studies. According to Efthimiadis \cite{efthimiadis2009students}, search engines were the most popular starting point of the searches. As Broussard et al. \cite{broussard2013seeking} state, in general (irrespective of task complexity), users tend to start their search processes with general concepts and gradually move towards other aspects of those general concepts. In addition, Graham et al. \cite{graham2006exploring} mention that, the majority of users directly access low-level pages (View Study/ View Results) via web search engines or consumer health sites, and as a result, they have recommended to make important information, such as background information and other search features easily accessible to users by placing their links on low-level pages. 

\vspace{0.25cm}

\textbf{5. How consumers evaluate health search results}\\

The fifth aspect of user search behaviour is, how consumers evaluate health search results. The key factors, such as trustworthiness, quality, features of websites etc. which were used by consumers for the selection and determination of usefulness of search results were identified based on six different studies. According Broussard et al. \cite{broussard2013seeking}, search results are selected based on rankings and familiarity on top of trustworthiness, quality and usefulness of the information. However, as Toms et al. \cite{toms2007consumers} state, credibility, reliability and trustworthiness are also important factors when selecting web pages from results lists. Sillence et al. \cite{sillence2007patients} mention that websites/ search results which contained credible information, personalized content or information written by similar people to consumers were assessed as trustworthy. In addition, according to Broussard et al. \cite{broussard2013seeking}, both medical-specific and general web sites were used by consumers when performing health information searches. As Sillence et al. \cite{sillence2007patients} argue, consumers also use their prior knowledge when determining the usefulness of search results. Furthermore, it was shown that some other factors, such as websites which were named with the medical condition searched, websites with visible local maps, design and content of the websites, and explicitness of information in relating lay and professional terms were important when evaluating the usefulness of the search results \cite{broussard2013seeking,graham2006exploring,sillence2007patients,keselman2008consumer}. Moreover, as Toms et al. and Keselman et al. \cite{toms2007consumers,keselman2008consumer} state, the search engine technology used and the organization of search results lists were taken into account when assessing the efficiency of consumer health information systems. According to these observations Zhang et al. and Toms et al. \cite{zhang2012health,toms2007consumers} have concluded that, all these factors should be considered when designing consumer health information websites/systems, because as Toms et al. and Keselman et al. \cite{toms2007consumers,keselman2008consumer} state, the information design of a web page influences the time taken to select from its results pages, users tend to make erroneous decisions by only considering the web page appearances and poor web resource configurations lead to unsuccessful health information searches. Therefore, once a consumer health information system is well designed, it can lead users to successful health information searches. 

\vspace{0.25cm}

\textbf{6. The impact of domain knowledge and search skills}\\


The sixth aspect of user search behaviour is the impact of domain knowledge and search skills on health information searches. The observations of researchers are presented based on eight different investigations. According to Toms et al. and Puspitasari et al. \cite{puspitasari2013characterizing,toms2007consumers}, users\textquotesingle domain knowledge and familiarity with health topics influence consumers\textquotesingle keyword searches and health information search behaviours. This has been confirmed by Keselman et al. \cite{keselman2008consumer} stating that, despite their web experience, users with incorrect and insufficient domain knowledge tend to search health information on unrelated websites and are unsuccessful in evaluating retrieved health information. In contrast, Keselman et al. \cite{keselman2008consumer} also mention that users with higher education levels tend to judge the authoritativeness of search results prior determining them as useful. The impact of domain knowledge and familiarity with health topics has been further supported by Ye et al. \cite{ye2017towards} stating that consumers with lower eHealth literacy and familiarity with health topics spend a longer time to evaluate the quality of search results and they focus only on the central text content of a resulting webpage. On the other hand, Ye et al. \cite{ye2017towards} also state that consumers with higher eHealth literacy and  familiarity with health topics spend much lesser time to evaluate the quality of search results and their focus is scattered all-over the webpage. In addition to these observations, it was also shown that, users with higher health topic familiarity and domain knowledge also search for other concepts associated with the main concept, use more specific and more varied vocabulary/query keywords, and start search processes with specific terms and move towards using more general terms \cite{zhang2012health,puspitasari2013characterizing,hu2013effects}. 

The behaviour of users with lower topic familiarity was further discussed by Hu et al., Efthimiadis and Broussard et al. \cite{hu2013effects,efthimiadis2009students,broussard2013seeking} stating that they make more mistakes (spelling) when issuing query terms, start search processes with more general terms and move towards more specific terms, quickly get satisfied with the search results, submit short queries and tend to select results from the first results page. Furthermore, according to Keselman et al. \cite{keselman2008consumer}, resource knowledge is important for navigational actions and according to Zhang et al. \cite{zhang2012health}, in general users prefer to use natural language while performing health searches. As Keselman et al. \cite{keselman2008consumer} state, users with better online search skills are able to perform efficient health information searches. This fact has been supported by Hu et al. \cite{hu2013effects}, mentioning that users with more search skills change query format more frequently to obtain relevant results and make less mistakes when issuing queries. Opposed to this, Hu et al. \cite{hu2013effects} also mention that, users with less search skills are unable to issue effective queries to obtain relevant results and make more mistakes when issuing queries. In terms of query reformulations, Puspitasari et al. \cite{puspitasari2013characterizing} mention that, different query reformulation patterns, such as dynamic query reformulation pattern (unfamiliar tasks) and parallel query reformulation pattern (familiar tasks) are being used by the consumers depending on the familiarity level of the tasks. The behaviour of query reformulations has also been discussed by Hu et al. \cite{hu2013effects} stating that the rate (frequency) at which query reformulations are performed is influenced by topic familiarity, where users with higher topic familiarity perform a lower number of query reformulations per session and users with lower topic familiarity perform a higher number of query reformulations per session. According to Hu et al. \cite{hu2013effects}, topic familiarity also influences the time taken to perform each query reformulation type. Summaries of user behaviours and characteristics with respect to domain knowledge, topic familiarity and search skills are presented in Figure 1, Figure 2 and Figure 3 respectively.   

\begin{figure}[t!]
	\includegraphics[width=1.0\textwidth]{domainknowledge.png}
	\caption{User behaviours and characteristics with respect to domain knowledge\label{fig1}}
\end{figure}  

\begin{figure}[t!]
	\includegraphics[width=1.0\textwidth]{topicfamiliar.png}
	\caption{User behaviours and characteristics with respect to topic familiarity\label{fig2}}
\end{figure} 

\begin{figure}[t!]
	\includegraphics[width=1.0\textwidth]{searchskills.png}
	\caption{User behaviours and characteristics with respect to search skills\label{fig3}}
\end{figure} 


\vspace{0.25cm}

\textbf{7. Overall view of consumer health information searches}\\

The seventh aspect of user search behaviour is the overall view of consumer health information searches. The observations of two different studies are presented by highlighting two main factors of consumer health searches which are query formulation and selection of appropriate search results. According to Efthimiadis \cite{efthimiadis2009students}, consumers still have problems in formulating queries which is also supported by Toms et al. \cite{toms2007consumers} stating that health-related information searches, including query formulations and efficient selections of appropriate search results still remain as challenging to many of the consumers.  

\vspace{0.25cm}

\textbf{8. What can be done to support lay people when searching for health information}\\

The final aspect of user search behaviour is the actions that could be taken to support lay people when searching for health information. Three highly useful methods and one moderately useful method were identified based on two different investigations. According to Keselman et al. \cite{keselman2008consumer}, laypeople with imprecise domain knowledge can be supported via information portals, individual websites and education tools. However, Toms et al. \cite{toms2007consumers} state that directory categories were not that useful when performing health information searches because consumers were not familiar with what a category contains and they did not like the fact that some categories having endless sets of category levels. 

\subsection{Strategies to Improve User Queries}

\begin{center}
	\begin{longtable}{||p{4cm} | p{10cm}||} 
		\hline
		\textbf{Name of the paper} & \textbf{Findings}\\ [0.5ex] 
		\hline\hline
		Assisting Consumer Health Information Retrieval with Query Recommendations (Zeng et al., 2006) & 1) The use of query recommendations, increases the number of successful queries; 2) The impact of query recommendations on successfully completing a predefined health search task and the overall user satisfaction is not significant; 3) Query suggestions provided by the Health Information Query Assistant system are beneficial for consumers when searching health related information \\ \hline Enhancing web search in the medical domain via query clarification (Soldaini et al., 2016) & 1) Behavioural synonym mapping is the best performing synonym mapping because users are more likely to correctly answer questions with the use of results retrieved using queries clarified with Behavioural mapping.; 2) Behavioural mapping is the least preferred mapping for incorrectly answered questions; 3) Lay people prefer query clarifications more than experts because lay people are able to retrieve  more useful information to correctly answer health-related questions; 4) The difference in success rates between lay people and experts is not significant; 5) There is a strong correlation between lay people and experts in terms of the success rate of each health information search task because some questions were more difficult for both groups when compared to other questions; 6) The number of correctly answered medical questions increased with the trustworthiness of the web pages used to answer the question; 7) Clarified queries are able to retrieve more trustworthy resources as results; 8) The \textquotesingle simple multinomial logistic regression classifier\textquotesingle performs the best when compared to all individual synonym mappings and unclarified queries; 9) The use of most suitable query clarification method (synonym mapping) to clarify each query has the ability to further improve the query clarification process; 10) Implicit query clarifications are also highly useful and most importantly, users do not need to know the correct medical terminologies which are being added to user queries while clarifying them.\\ \hline Health Suggestions: A Chrome Extension to Help Laypersons Search for Health Information (Lopes et al., 2016) & 1) Good acceptance of the health query suggestions from the tool (Google Chrome extension); 2)  Multilingual and multi-terminology suggestions are useful to retrieve more relevant documents; 3) Query suggestions also lead to more successful searches \\ \hline The effectiveness of query expansion when searching for health related content: InfoLab at CLEF eHealth 2016 (Silva et al., 2016) & Clear conclusions regarding the impact of query expansion on the improvement of initial queries and the overall retrieval performance are hard to make because of the lack of information about relevance and readability assessment for the test collection used to evaluate the retrieved documents \\      		
		\hline
	\end{longtable}	
	
\end{center}  


\textbf{1. How query recommendations, clarifications, suggestions and expansions impact health information searches}\\


The first aspect of query improvements is how query recommendations/ clarifications impact health information searches. The different ways in which query recommendations and clarifications impact health searches have been identified based on four different studies. According to Zeng et al. \cite{zeng2006assisting}, the use of query recommendations, increases the number of successful queries and query suggestions provided by different systems are beneficial for consumers when searching health related information. This fact has been supported by Lopes et al. \cite{lopes2016health} mentioning that, query suggestions lead to more successful searches.  However, Zeng et al. \cite{zeng2006assisting} also state that, the impact of query recommendations on successful completion of a predefined health search task and the overall user satisfaction is not significant. The importance of query clarifications when performing health information searches was highlighted by Soldaini et al. \cite{soldaini2016enhancing} stating that, the number of accurately answered medical questions increases with the trustworthiness of the web pages used to answer the questions and with the use of clarified queries, consumers are able to retrieve more trustworthy resources as results. According to Soldaini et al. \cite{soldaini2016enhancing}, implicit query clarifications which are performed on websites without the involvement of users, are also highly useful and most importantly, users do not need to know the correct medical terminologies which are being added to user queries while clarifying them. In addition, Lopes et al. \cite{lopes2016health} argue that, multilingual and multi-terminology suggestions are also useful to retrieve more relevant documents. Silva et al. \cite{silva2016effectiveness} investigated how query expansion aids laypeople in improving initial queries and the overall retrieval performance. However, they have not been able to make clear conclusions because of the lack of information about relevance and readability assessment of the test collection they have used to evaluate retrieved documents. 

\vspace{0.25cm}
\textbf{2. The impact of different synonym mappings on query clarification}\\

The second aspect of query improvements is the impact of different synonym mappings on query clarification. The observations are presented based on one study which utilized three main synonym mappings for the analysis. According to Soldaini et al. \cite{soldaini2016enhancing}, Behavioural synonym mappings which mapped user expressions to medical symptoms is the best performing synonym mapping compared to MedSyn mappings (mappings focused on diseases and symptoms, and removed terms which were not related to UMLS semantic types) and DBpedia  mappings (mapping lay terms to expert terminology based on Wikipedia redirect pages), because users are more likely to accurately answer health-related questions with the use of results retrieved using queries clarified with Behavioural mapping and Behavioural mapping is the least preferred mapping for incorrectly answered questions. In addition, Soldaini et al. \cite{soldaini2016enhancing} also state that, \textquotesingle simple multinomial logistic regression classifier\textquotesingle (an ensemble of three query clarification methods and no query clarification method) is the best performing query clarification method compared to all individual synonym mappings and any unclarified queries. Furthermore, Soldaini et al. \cite{soldaini2016enhancing} have also shown that, the selection of the most appropriate query clarification method (synonym mapping) to clarify each query, further improves the clarification process. 

\vspace{0.25cm}
\textbf{3. Consumers' preference for query recommendations, clarifications and suggestions when performing health information search tasks} \\

The final aspect of query improvements is consumers\textquotesingle preference for query recommendations, clarifications and suggestions when performing health information search tasks. Consumers\textquotesingle different preferences are presented based on two different investigations. According to Soldaini et al. \cite{soldaini2016enhancing}, comparatively lay people prefer query clarifications more than experts because lay people are able to retrieve  more useful information to successfully complete a health-related search. Lopes et al. \cite{lopes2016health} identified consumers\textquotesingle higher preference for query suggestions and they have mentioned this as good acceptance of health query suggestions from the tool (Google Chrome extension) was observed. In addition, according to Soldaini et al. \cite{soldaini2016enhancing}, a strong correlation has been identified between lay people and experts in terms of the success rate of each health information search task,  because some tasks were more difficult for both groups when compared to other health information search tasks, no matter which query clarification method has been used. However, Soldaini et al. \cite{soldaini2016enhancing} also have identified that, the difference in success rates between lay people and experts is not significant. 


\subsection{Methods to Enhance Consumers' Understandability of Health Information}

\begin{center}
	\begin{tabular}{||p{4cm} | p{10cm}||} 
		\hline
		\textbf{Name of the paper} & \textbf{Findings}\\ [0.5ex] 
		\hline\hline
		Current challenge in consumer health informatics: Bridging the gap between access to information and information understanding (Alpay et al., 2009) & 1) The use of contextualization significantly increases lay users\textquotesingle understanding of health information by raising users\textquotesingle health literacy; 2) Contextualization helps to improve users\textquotesingle understanding of different health concepts by reducing communication complexity; 3) Users\textquotesingle cognitive style (the way they organize, filter, transform and process information) is also a significant factor in thoroughly understanding health content; 4) Consumers also use their own contexts (social and psychological) as a support to understand health information well \\		
		\hline
	\end{tabular}	
	
\end{center}  


\textbf{The importance of methods which enhance consumers' understandability of health information}\\

Studies have been conducted in order to investigate the importance of different methods which enhance consumers\textquotesingle understandability of health information. The observations are presented based on both system side approaches and user side approaches which have been used to enhance consumers\textquotesingle understandability of health information. According to Alpay et al. \cite{alpay2009current}, the use of supportive information to thoroughly explain a health concept (contextualization of health information), significantly increases lay users\textquotesingle understanding of health information by raising users\textquotesingle health literacy and reducing communication complexity. In addition, Alpay et al. \cite{alpay2009current} mention that, users\textquotesingle cognitive style, including the way they organize, filter, transform and process information is also a significant factor in thoroughly understanding health content. Apart of that, as Alpay et al. \cite{alpay2009current} mention, consumers also use their own contexts (social and psychological) as a support to understand health information well. 

\subsection{Retrieval Models}  

\begin{center}
	\begin{tabular}{||p{4cm} | p{10cm}||} 
		\hline
		\textbf{Name of the paper} & \textbf{Findings}\\ [0.5ex] 
		\hline\hline
		 Choices in Knowledge-Base Retrieval for Consumer Health Search (Zuccon et al., 2018) & 1) Relevance feedback improves the effectiveness of query expansion; 2) Overall, UMLS KB expanded more queries and retrieved a better set of documents compared to Wikipedia KB. Therefore, KB query expansion with UMLS is more effective than with Wikipedia \\		
		\hline
	\end{tabular}	
	
\end{center}  

\textbf{The impact of retrieval models on consumer health information search}\\ 


Researches have also been conducted to study the impact of retrieval models on consumer health information search. The observations are presented based on a study which was conducted to investigate the performance of two different knowledge bases. According to Zuccon et al. \cite{zuccon2018choices}, UMLS Knowledge Base has the ability to effectively expand more queries and retrieve more reliable and useful documents compared to Wikipedia Knowledge Base. In addition, Zuccon et al. \cite{zuccon2018choices} also state that, relevance feedback improves the effectiveness of these query expansions.  
	
\subsection{Problems faced by Consumers when Searching for Health Information Online}

\begin{center}
	\begin{longtable}{||p{4cm} | p{10cm}||} 
		\hline
		\textbf{Name of the paper} & \textbf{Findings}\\ [0.5ex] 
		\hline\hline
		 Internet healthcare: do self-diagnosis sites do more harm than good? (Ryan et al., 2008) & Consumers face several problems when using self-diagnosis websites\textquotesingle information to diagnose symptoms; 1. Experience anxiety after reading inaccurate or false diagnoses information about serious conditions; 2. Unreliability of available information with no involvement of a health professional; 3. Making users buy prescribed drugs from Internet pharmacies; 4. Ignore professional assistance and diagnosis by believing the false reassurance provided by the self-diagnosis websites. Therefore, essentially information available on self-diagnosis websites should be well structured, reliable, evidence-based and presented in lay language which will then be easier for lay people to understand \\ \hline Current challenge in consumer health informatics: Bridging the gap between access to information and information understanding (Alpay et al., 2009) & 1) There are gaps between informational contexts provided to users by health information websites and users\textquotesingle personal contexts that they use to enhance their understandability of health information \\ \hline Diagnose this if you can (Zuccon et al., 2015) & 1) Current retrieval techniques are not able to retrieve relevant health information when queries which describe symptoms in a circumlocutory manner are issued; 2) There is a higher possibility of users retrieving misleading advice and irrelevant information when searching for self-diagnosis information which will also lead to inaccurate self-diagnosis decisions and ultimately cause harm \\ \hline The Positive and Negative Influence of Search Results on People's Decisions about the Efficacy of Medical Treatments (Pogacar et al., 2017) & 1) When search results are bias towards accurate information users are able to more accurately determine the efficacy of health treatments when compared to determining efficacy without any search results; 2) When search results are bias towards incorrect information users\textquotesingle make more harmful decisions which are even worse than the decisions made with no search results; 3) This impact of search results bias on the number of accurate decisions and inaccurate decisions is statistically significant; 4) Users are less likely to label a health treatment as unhelpful because they are willing to see positive information in search results;  5) Users with more prior knowledge are less likely to classify a medical treatment as inconclusive and therefore, make more inaccurate and harmful decisions when determining the efficacy of health treatments; 6) Users who are less confident with the efficacy of the treatments labelled them as inconclusive; 7) The top ranked search result (Rank 1) is important because users are more likely to click on that search result; 8) The more participants interact with the search results the more accurate decisions they make; 9) Search results notably affect people\textquotesingle s decisions about the effectiveness of medical treatments and have a higher potential to both aid and mislead users \\		
		\hline
	\end{longtable}	
	
\end{center}  

\textbf{Problems faced by consumers when searching for self-diagnosis information, symptoms and efficacy of health treatments}\\


Researchers have conducted studies in order to understand different problems faced by lay people when they search for health/ medical information, as a means of improving health information search systems. The problems faced by consumers when searching for self-diagnosis information, symptoms and treatments are presented in this section based on four studies. According to Ryan et al. \cite{ryan2008internet}, consumers face different problems when using self-diagnosis websites\textquotesingle information to diagnose symptoms. They state these problems as experiencing anxiety after reading inaccurate or false diagnosis information about serious conditions, unreliability of available information because there is no involvement of a health professional, the possibility of making users buy prescribed drugs from Internet pharmacies, and the potential to ignore professional assistance and diagnosis by believing false reassurance provided by self-diagnosis websites. This has also been confirmed by Zuccon et al. \cite{zuccon2015diagnose} stating that, there is a higher possibility of users retrieving misleading advice and irrelevant information when searching for self-diagnosis information which will then lead to inaccurate self-diagnosis decisions and eventually cause harm to people. One of the main reasons for users obtaining irrelevant health information has also been highlighted by Zuccon et al. \cite{zuccon2015diagnose} mentioning that, current retrieval techniques are not able to retrieve relevant health information when queries which describe symptoms in a circumlocutory manner are issued. Therefore, Ryan et al. \cite{ryan2008internet} have concluded that information available on self-diagnosis websites should be well structured, reliable, evidence-based and presented in lay language. 

In addition to problems faced by consumers when searching for self-diagnosis information and symptoms, according to Pogacar et al. \cite{pogacar2017positive}, consumers also face similar problems when searching and determining the efficacy of health treatments. They have mainly observed that users\textquotesingle decisions are highly dependent upon search results\textquotesingle bias. Therefore, when search results are bias towards correct information, users were able to accurately determine the efficacy of health treatments. However, when search results were bias towards incorrect information users\textquotesingle made more inaccurate and harmful decisions which were even worse than the decisions made without any search results. Other than that, users who were less confident with the efficacy of the treatments labelled them as inconclusive. Pogacar et al. \cite{pogacar2017positive} also claim that the impact of search results bias on the portion of accurate decisions and harmful decisions is statistically significant. A few of the main reasons for users making inaccurate decisions regarding the efficacy of health treatments were identified. They are, users less likely behaviour to label a health treatment as unhelpful because they are willing to see positive information in search results, users\textquotesingle prior health knowledge which make them think that most of the medical treatments are useful, users\textquotesingle bias towards clicking on the top ranked (rank 1) search result and the lower number of search results interacted by users. Therefore, they have concluded that, search results significantly affect people\textquotesingle s decisions of the effectiveness of medical treatments and have a great potential to assist users as well as to harm them. Furthermore, as Alpay et al. \cite{alpay2009current} have observed, providing supportive information (contextualization) has the ability to reduce communication complexity and increase users\textquotesingle understanding of different health concepts. However, they have also identified that there is a gap between this additional/ supportive information, which is known as \textquotesingle informational context\textquotesingle, provided to users by health information websites and users\textquotesingle own personal contexts (social and psychological) which are used as a support to better understand health-related information.  

\vspace{0.5cm}

\subsection{Methods used to Evaluate the Relevance of Health Search Results}

\begin{center}
	\begin{longtable}{||p{4cm} | p{10cm}||} 
		\hline
		\textbf{Name of the paper} & \textbf{Findings}\\ [0.5ex] 
		\hline\hline
		Diagnose this if you can (Zuccon et al., 2015) & Assessors determined the relevance of retrieved search results. 1) On average only about 4 to 5 results out of the first 10 results are considered as containing relevant information that was helpful to people; 2) If users seek highly relevant information, only 3 out of the first 10 results contained highly useful information to self-diagnose medical symptoms; 3) \textquotesingle Somewhat relevant\textquotesingle documents contain information not only about the relevant symptom but also  about other symptoms as well; 4) \textquotesingle Highly relevant\textquotesingle documents contain information focused on the symptoms searched by users, such as descriptions of the symptoms and causes of those symptoms which are also supported by photographic material; 5) \textquotesingle On topic but unreliable\textquotesingle documents were considered as irrelevant for this evaluation\\ \hline Decision Tree Approach for Consumer Health Information Search (Thenmozhi et al., 2016) & 1) Method with feature selection performs well and has a higher accuracy when determining the relevance of the search results compared to the method without feature selection; 2) The feature selection model can be used to filter out irrelevant documents prior presenting search results to consumers, so as to aid them in retrieving more relevant health related documents \\		
		\hline
	\end{longtable}	
	
\end{center} 

\textbf{How humans assess the relevance of health search results and the use of a Decision Tree model to classify search results as relevant and irrelevant}\\ 

Retrieved results for a health search task are evaluated in order to determine whether those results are relevant to what a user has searched for. Researchers have used different methods in their studies to decide the relevance of retrieved search results. Zuccon et al. \cite{zuccon2015diagnose} recruited human assessors to evaluate the relevance of retrieved search results. According these assessors, documents which contained information focused on the symptoms searched by users, such as descriptions of the symptoms and causes of those symptoms which are also supported by photographic material were considered as \textquotesingle Highly relevant\textquotesingle. Therefore, they have observed that, when seeking for highly relevant information, on average, only three out of top ten results contain highly useful information to self-diagnose medical symptoms. In addition, approximately four or five results of the top ten results, which contained little information that was helpful to people, were considered as relevant (provide useful information to people in order to self-diagnose themselves). Furthermore, documents which contained information not only about relevant symptoms but also about other symptoms were considered as \textquotesingle somewhat relevant\textquotesingle. Moreover, \textquotesingle on topic but unreliable\textquotesingle documents were considered as irrelevant in this evaluation. Thenmozhi et al. \cite{thenmozhi2016decision} have used a Decision Tree approach in their study to categorize retrieved health search results as relevant or irrelevant based on users\textquotesingle queries. According to their observations, the method with feature selection (included only the important features for determining the relevance) performs well and has a higher accuracy when determining the relevance of the search results compared to the method without feature selection. Therefore, they concluded that, this feature selection model can be used to filter out irrelevant documents prior presenting search results to consumers, so as to aid them in retrieving more relevant health related documents.     

\subsection{Techniques used to Map User Queries to Medical Symptoms}

\begin{center}
	\begin{longtable}{||p{4cm} | p{10cm}||} 
		\hline
		\textbf{Name of the paper} & \textbf{Findings}\\ [0.5ex] 
		\hline\hline
		Circumlocution in diagnostic medical queries (Stanton et al., 2014) & 1) Users are able to generate more successful queries to search for the name of a symptom, when they are presented with images and videos of a medical symptom; 2) The features, such as Encyclopedias, anatomy (body parts), synonyms, medical dictionaries, Greek and Latin roots and paraphrases are useful when attempting to match a user query with its corresponding symptom name; 3) This approach exhibits improved performance when matching queries with symptom names compared to random guessing\\		
		\hline
	\end{longtable}	
	
\end{center} 

\textbf{What aids consumers in describing a medical symptom successfully and the features used in systems to map user queries with medical symptom names}\\  


In particular, when users want to search for information about medical symptoms, it is really important that they are able to identify proper terms to address these symptoms, in order to retrieve accurate information about these symptoms. Researchers have used different techniques in their studies to determine the impact of those techniques on accurately mapping user queries to medical symptoms. Therefore, according to Stanton et al. \cite{stanton2014circumlocution}, users are able to generate more successful queries to search for the name of a symptom, when they are presented with images and videos of different medical symptoms. This indicated that users are more likely to correctly describe a symptom when they actually experience it. In addition, these researchers also observed that, from health information search systems\textquotesingle side, the use of features, such as Encyclopedias, anatomy (body parts), synonyms, medical dictionaries, Greek and Latin roots and paraphrases is useful when attempting to match a user query with its corresponding symptom name. Therefore, they have concluded that, the use of previously mentioned features has the ability to improve the performance when matching queries with symptom names compared to randomly guessing symptoms names from user queries. \\	

	
\section{Discussion}

This section critically analyses the key findings, significance of this literature review, limitations associated with this literature review, and key recommendations arising from this literature analysis. In addition to that this section also highlights some of the aspects which are uncovered by the considered set of research papers. \\

\subsection{Crtical Analysis of Key findings}

The key findings of this literature analysis are presented with respect to the 7 sub aspects of consumer health search which were identified by analysing the 22 recent research papers on consumer health information search.\\ 

The first key aspect of consumer health search identified was user search behaviours. According to the findings, the complexity or difficulty of a consumer health search influences the overall health search process, including session length, number of web pages visited, interaction strategies, search patterns, query reformulation patterns and search planning. When considering the purposes of consumer health searches, it was identified that, in addition to searching information about diseases, symptoms, treatments etc., users also search for confirming and novel information, and some of this information is also used by consumers to improve communications with physicians. In addition, formulation of effective queries and the use of most appropriate query reformulation types lead consumers to successful health information searches. In relation to the starting point of a search, it was identified that, at the beginning of a search process consumers search for more general concepts via search engines. Moreover, consumers tend to evaluate the usefulness of search results based on factors, such as trustworthiness, quality, features of websites, reliability, ranking, familiarity, personalized content and prior knowledge. The organization of the search results lists is important, because consumers use it to assess the efficiency of health information systems. Furthermore, topic familiarity, domain knowledge and search skills have a higher impact on health information searches performed by consumers. Therefore, consumers who have higher levels of domain knowledge, topic familiarity and more search skills are able to perform successful health searches. Hence, health information search still remains as a challenging task to most of the consumers because consumers have problems with query formulations and selection of useful search results. As a solution, information portals, individual websites and education tools are suggested to be used as a support for consumers when searching for health information.\\ 

The second key aspect of consumer health search identified was strategies used to improve user queries. According to the findings of this aspect, query recommendations, clarifications and suggestions, all are able to increase the number of successful queries and therefore, positively impact consumer health searches. The use of different synonym mappings, such as Behavioural, MedSyn and DMpedia has the ability to improve query clarification processes with varying levels of performance. In addition, query clarification process can be further improved by using a logistic regression classifier which ensemble all three synonym mapping methods and a method with no query clarification. Furthermore, lay people prefer to have query recommendations, clarifications and suggestions while searching for health information, because they are able to retrieve more useful information with the assistance of these techniques. \\

The third key aspect of consumer health search identified was methods used to enhance consumers\textquotesingle understandability of health information. The findings of this aspect relieved that, both user side approaches, such as users\textquotesingle cognitive style and the use of their own contexts (social and psychological), and system side approaches, such as contextualisation of health information (providing supportive information to thoroughly explain a health concept) which are used to enhance consumers\textquotesingle understandability of health information have a positive influence on overall consumer health information searches, because they assist consumers\textquotesingle to increase their understandability of health information.\\

The fourth key aspect of consumer health search identified was retrieval models. According to the findings of retrieval models, the use of most suitable retrieval model is able to effectively expand more queries and retrieve highly useful information. In addition, relevance feedback can be used to further improve the effectiveness of these query expansions.\\

The fifth key aspect of consumer health search identified was problems faced by consumers when seeking for health-related information online. The findings of this aspect revealed that, consumers face various problems when searching for different types of health-related information. When they search for self-diagnosis information they face many problems, such as experiencing anxiety after reading inaccurate or false diagnosis information, unreliability of available information, making users buy drugs from Internet pharmacies, ignore professional assistance by believing everything on self-diagnosis websites, and making inaccurate and harmful self-diagnosis decisions. In addition, they also face problems when determining the efficacy of health treatments, because their decisions are highly dependent on search results\textquotesingle bias. In other words, if search results are bias towards inaccurate information consumers tend to make erroneous and harmful decisions about the efficacy of health treatments. Furthermore, there are gaps between the supportive information provided to users by health information websites and users\textquotesingle own personal contexts (social and psychological) which they use to support their understanding of health information. \\ 

The sixth key aspect of consumer health search identified was methods used to assess the relevance of health search results. The findings of the methods used to assess the relevance of health search results revealed that, search results which contain information directly relevant to what user has searched are considered as highly relevant by human assessors. In addition, when using an approach, such as Decision Tree model to determine the relevance of health search results it is important to use a method with feature selection, because it improves the accuracy of decisions made on the relevance of search results.\\

The final key aspect of consumer health search identified was techniques used to map user queries to medical symptoms. According to the findings of techniques used to map user queries to medical symptoms, it was understood that, users are able to correctly describe and search for a symptom, when they actually experience it. In addition, the use of features, such as Encyclopedias, anatomy (body parts), synonyms, medical dictionaries etc. is highly useful because these features improve the performance of health information search systems, when matching user queries with symptom names. \\   

\subsection{Uncovered Aspects} 

There are a few uncovered aspects which were noticed while analysing the 22 recent research papers. One of the main aspects was that, studies have not been conducted by performing the same consumer health information search task in several different consumer health information websites. It is important to conduct the same health information search task in several health information websites, in order to understand how different architectures, structures and interfaces of websites influence, consumer health information searches. In another study, which aimed at investigating how query expansions impact on improving initial user queries and the overall retrieval performance, authors were not able to provide clear conclusions, because of the lack of some important information, such as relevance assessments for the test collection used for the study. In addition to these some of the researches were conducted using simulated health search tasks and some environments did not represent an actual environment which a user is more likely to search for health-related information. Furthermore, there were comparatively a very few number of researches which have investigated aspects, such as techniques to map user queries to medical symptoms, methods to evaluate the health search results, retrieval models and methods to enhance consumers\textquotesingle understandability of health information. Because of the availability of very little information on those aspects, they also can be identified as aspects which are not properly covered in recent research papers.     


\subsection{Significance of the Literature Analysis}

The main purpose of this literature review was to contribute to the existing findings of consumer health search presented in recent research papers by summarising the findings, categorising them into various aspects of consumer health information search and by analysing each aspect separately. Hence, this literature analysis has thoroughly analysed the findings from 22 recent research papers on consumer health information retrieval/search by categorising them into 7 different sub aspects of consumer health search. Therefore, the importance of this analysis is that it clearly synthesize the existing knowledge (user search behaviours,retrieval models) of consumer health search presented in the research papers and also it aids in identifying areas which are pertinent for future studies. \\

\subsection{Limitations of the Project}

There are also a few limitations of this literature analysis. One limitation is the number of research papers which have been analysed. It would have been better if this literature analysis could cover some more research papers, so as to synthesize more findings covering other aspects of consumer health information search. Another limitation is that the unbalanced number of research papers covered in each aspect of consumer health search. For instance, a higher number of research papers have been covered for the aspect \textquotesingle user search behaviours\textquotesingle, but comparatively the number of research papers covered for the aspects, such as retrieval models, methods to enhance consumers understandability of health information and methods to evaluate the relevance of health search results were very low. \\

\subsection{Key Recommendations arising from this Literature Analysis}

Based on this literature analysis, it is highly recommended for consumer health information system designers to consider above mentioned findings as much as possible when designing these systems, because it will favour consumers with low levels of domain knowledge to retrieve more useful health information as required by formulating effective queries and selecting the most useful health search results. It is also recommended for researchers to conduct more research in this area, because still there are gaps in knowledge remaining in this area. It is really important to conduct more research in this area, because unlike other information retrievals if consumers retrieve inaccurate health information and if they make use of it, then there is a higher risk of them causing harm to their own health. In addition to the recommendations based on this literature review analysis, it is also recommended to conduct another literature analysis in order to extend the knowledge synthesising process by covering other aspects of consumer health information retrieval.  

\pagebreak

\section{Conclusion}

In conclusion, the aim of this literature analysis was to contribute to the various aspects of consumer health information searches, investigated in recent research papers, by synthesizing this information and presenting it as a literature analysis. Therefore, this literature analysis mainly focused on four key aspects of consumer health information search, such as user search behaviours, consumer health information retrieval models, strategies to improve user queries and different problems faced by consumers when searching for health information online. The significance of this literature review project was to present the scattered knowledge of each of these main aspects as a well-structured analysis. In addition to the different aspects/ topics covered in each paper, this literature analysis has also highlighted different methods used to investigate each of these aspects and their corresponding findings. Furthermore, all the identified uncovered aspects of consumer health information search was also highlighted in the discussion section.\\ 

This literature review has thoroughly analysed 22 research papers which cover above-mentioned four main aspects. Firstly, annotations were written for all the chosen research papers. Secondly, a table of topics was prepared which categorized findings of each research paper. As a result, this literature analysis has identified 7 sub aspects of consumer health information search while analysing those key findings. They are user search behaviours, strategies used to improve user queries, methods used to enhance consumers' understandability of health information, retrieval models, problems faced by consumers when searching for health-related information online, methods used to evaluate the relevance of health search results and techniques used to map user queries to medical symptoms. All these identified aspects play an important role in consumer health information search process.\\ 
  
Therefore, this analysis is really useful for health information search system designers because they are able to find heaps of helpful information and facts which can be embedded when designing such systems. For instance, with the use of information about user search behaviour, health information system designers can design interfaces in such a way that it will be really appealing to a lay person. In addition, they can add other features, such as links to background information on the most viewed pages, so that it will be very convenient for the users of the system.  As it has been mentioned in the introduction, the use of online resources for retrieving consumer health information has highly increased in past few years and most importantly the possibility of users retrieving inaccurate and unreliable health information is high because they are not familiar with proper medical terms. Therefore, the importance of thinking health information retrieval in favour of users is that, the potential negative impact on users is very high if they make use of inaccurate and unreliable health information that they have obtained. For instance, if a cancer patient search for medical treatments and obtain inaccurate information about medical treatments, it can lead them to worsen their illness or even can cause death as it is a serious illness.\\
  
In terms of potential future work, as it has been mentioned before, a few uncovered aspects of consumer health information search have been highlighted in the discussion section. This information as well as the findings of the considered studies will be really useful and important for researchers who are keen to conduct researches on consumer health information search. They can get a basic idea of what has been studied in this field and then clearly identify areas which are pertinent for future studies. It is really important to fill this gap (uncovered aspects) by conducting more research on such uncovered aspects of consumer health search, so as to further improve consumer health information search systems in favour of lay people. 

\section{Reflections on My Learning}

During the 13 weeks period time, while undertaking my project, I had to face several challenges. The first one was to define a proper protocol to conduct this literature analysis. While defining the protocol the hardest part was to decide the keywords and phrases to be used to search for relevant literature. For instance, \textquotesingle consumer health information retrieval\textquotesingle was one of the most effective phrases that I used to retrieve relevant literature. However, in some cases I was faced with lack of keywords and phrases. As a solution to this, I used some other options to obtain relevant research papers. These were the two options provided in Google Scholar as \textquotesingle Cited by\textquotesingle and \textquotesingle Related articles\textquotesingle which were really useful in findings more relevant research papers, and also, I used reference lists in articles that I have identified as useful to obtain more research papers. \\
  
The second challenge I faced was to understand the content presented in prior research related to consumer health information retrieval. The reason is because I did not have any information retrieval background. As a solution to this, I read a few articles which covered concepts behind information retrieval including health information retrieval. The third challenge which also can be stated as the hardest part of this project was to properly write annotations for the selected research papers. As I did not have prior experience in writing annotations I first wrote annotations which were insufficient (too shallow) to cover all the important facts presented in the research papers. Then my second approach was too broad where I started to add unnecessary information as well. With the feedback of my supervisor I was able to adjust my annotations so as to only include the most important facts presented in each research paper.  \\
   
In terms of the things that I did best (strengths) in this literature analysis, I was able to successfully prepare the table of topics in which I categorized findings of each research paper according to different aspects of consumer health information search. Then I was able to thoroughly analyse the findings for each aspect and identify patterns within each aspect of consumer health search. Therefore, I was able to present my literature review results in a well-organized manner. In terms of the things that I did least well (weaknesses) in this literature analysis, some of my attempts to search for relevant research were not really successful at the very beginning because I retrieved some researches which were published prior 2005 (according to my protocol I was supposed to select papers published 2005 onwards) and a very few of them were not actually related to consumer health search. In addition, as I have mentioned before my first attempts to write annotations for research papers were not that successful. However, fortunately with the help of my supervisor\textquotesingle s feedback I was able to fix these issues as soon as they were identified. \\ 
    
According to my opinion, I feel that it is required for me to further develop my knowledge of information retrieval systems and its associated concepts, such as stopping and stemming. In addition, I need to further develop my reading skills so as to read research papers much faster and to extract the most important information presented in those research papers. In terms of my writing skills I should further improve them to write annotations which highlight only the most important information and to explain each fact in a simple language. \\
      
The most important thing I learned by conducting this literature review project was that how essential it is for lay people to retrieve reliable and accurate information when searching for health-related information. Even though health information search can be seen as just another type of information search, the negative impact of retrieving unreliable and inaccurate health information is significantly high. For instance, retrieving information about an inaccurate medical treatment and making use of it can cause terrible harm to lay people. In addition, this literature review project on consumer health search made me realise how hard and crucial it is to design a successful health information retrieval system. \\
     
According to what I have learnt, the use of the most suitable project management approach is essential to conduct a successful project. For this project I have used Dynamic Systems Development Method (DSDM) as my project management approach. Even though it was not possible to develop my literature analysis in phases as it is normally done in a development project, I was able to make use of the important features of DSDM while conducting this literature analysis. For instance, the iterative development was utilized when preparing annotations and I was able to obtain continuous feedback from my supervisor which lead me to deliver a quality product on time. In addition, according to what I have learnt about professional research, I feel that at the beginning it is quite a hard task but with experience it can be improved and also become less harder. \\

The learnings of this project mainly provide opportunities to consumer health information system designers. All these findings which are categorised into different aspects of consumer health search is really useful for system designers and they can put their maximum effort to embed features as much as possible in favour of consumers or laypeople. This will allow lay people to retrieve more useful and accurate health information by issuing effective queries and selecting the most reliable health search results. In terms of the next steps which are possible with the use of the learnings of this project, I believe it is essential to continue research on consumer health search, so as to fill in the gaps (some are presented in the discussion section) and to identify more user search behaviours and problems they encounter when searching for health information. Then all these findings can be utilized to develop more strategies which improve user queries and retrieval models that will retrieve more useful information. The development of such effective strategies and models will then aid lay people in satisfying their health information needs. \\
      
In addition, these learnings also provide opportunities for me to conduct successful literature reviews in the future based on any related topic. For instance, if I will have to do a research project on an IT related topic, one of my main requirements would be to conduct a literature review on prior related work to gain basic knowledge about concepts associated with that topic and to identify gaps in knowledge to conduct further research. Therefore, with the learnings of this project I will be able to conduct a successful literature review which is an essential component in both research projects and development projects.

\pagebreak  

\bibliographystyle{plain}

\bibliography{reference}  

\pagebreak 

\addcontentsline{toc}{section}{Appendix A: Annotations of Research Papers}
\section*{Appendix A: Annotations of Research Papers} 

\vspace{0.5cm}


\begin{enumerate}
 \renewcommand\labelenumi{\bfseries\theenumi .}
\item { \textbf{Guido Zuccon, Bevan Koopman, and Joao Palotti. Diagnose this if you can. In \textit{European on Information Retrieval,} pages 562-567. Springer, 2015.}}

This study investigated the effectiveness of current web search engines in retrieving relevant and useful information for diagnostic medical queries written in a circumlocutory manner. \textquotesingle Circumlocutory queries\textquotesingle are queries in which the user observed symptoms are described in a long winded form, rather than using the specific medical terms for the symptoms. In this study the authors mainly investigated eight symptoms. In order to investigate these symptoms, they considered three or four queries per each symptom. As a result, in total twenty six queries were investigated. The  medical circumlocution diagnostic queries were generated based on a method proposed in a prior study. These diagnostic queries represented query terms that may be issued by users when searching for self-diagnosis information.  Along with the queries, the names of the symptoms that users referred were recorded in a table. This information was used for relevance assessment. This study used Google and Bing search engines to issue all the twenty six queries and to obtain the top ten search results for each of these queries. URLs of all these top-ranked results were also recorded. The authors recruited researchers from Queensland University of Technology and also another eight higher degree students to evaluate the relevance of each of these retrieved result. Most importantly, none of these participants were medical experts. As a result, the authors were able to observe the actual behaviours of lay people (no medical knowledge) when they search for health information online. Each participant was shown web pages only related to a single symptom. 

Then participants determined the relevance of each web page by considering whether a user is able to properly self-diagnose (find the accurate medical term of the symptom) a medical symptom by using the information on that web page. Participants were given four labels, which are \textquotesingle Not relevant\textquotesingle, \textquotesingle On topic but unreliable\textquotesingle, \textquotesingle Somewhat relevant\textquotesingle and \textquotesingle Highly relevant\textquotesingle, and were instructed to assign one label to each of the search result. The effectiveness of Google and Bing search engines was assessed using precision (relevance of documents) and nDCG (ranks of relevant documents). The authors observed that on average only about four or five results of the top ten results, which were retrieved using Google and Bing, consisted of useful information for people in order to self-diagnose medical symptoms. In particular, if highly relevant information was sought, on average, only three out of top ten results contained highly useful information to self-diagnose medical symptoms. When analysing the documents/results which were judged as \textquotesingle Somewhat relevant\textquotesingle, it revealed that these documents contained information not only about the relevant symptom but also it contained information about other symptoms as well. For instance, some documents which were categorized as \textquotesingle Somewhat relevant\textquotesingle had definitions of several symptoms including the definition of the targeted symptom. When analysing \textquotesingle highly relevant\textquotesingle document category, it revealed that, those documents contained information focused on the symptoms searched by users, such as descriptions of the symptoms and causes of those symptoms which were also supported by photographic material. The category of documents \textquotesingle on topic but unreliable\textquotesingle was considered as irrelevant in this evaluation.  

In conclusion, according to the results of this study, it was revealed that currently used retrieval techniques in search engines are not able to retrieve useful results for queries issued with circumlocutory or colloquial descriptions of symptoms. Hence, there are possible risks associated with lay people when they search for self-diagnosis information online, because they are more likely to retrieve misleading information that would lead them to inaccurate self-diagnosis decisions and eventually cause harm to them.   

\textbf{Limitations:} Only a small amount of queries were considered in this study. 
During the evaluation performed by each participant, only one query was taken into account, but in reality it is important to consider the whole search session, including all the health queries issued by users. Other factors, such as reliability and understandability of the obtained information which are necessary to evaluate the relevance of search results were not considered in this study.\\

\item {\textbf{Isabelle Stanton, Samuel Ieong, and Nina Mishra. Circumlocution in diagnostic medical queries. In \textit{Proceedings of the 37th international ACM SIGIR conference on Research \& development in information retrieval,} pages 133-142. ACM, 2014.}}

The objective of this study was to find the corresponding professional medical terms for medical signs and symptoms that are referred by user generated \textquotesingle colloquial health search queries\textquotesingle. For instance, the colloquial name for \textquotesingle cephalalgia\textquotesingle is headache and a circumlocutory for \textquotesingle cephalalgia\textquotesingle is \textquotesingle my head is pounding\textquotesingle. The solution suggested in this paper ignored query terms that were not helpful to determine the proper medical symptom and picked up the terms which were more likely to matter. Therefore, this study aimed at solving three main problems:

\begin{enumerate}
	\item How to generate training data for this experiment (queries which colloquially explain medical signs or symptoms)
	
	\item How to automatically represent the similarity between a user query and a medical symptom
	
	\item How to automatically infer the actual symptom name from a circumlocutory expression

\end{enumerate}


The authors used a reversed approach of crowd sourcing in conjunction with images and videos to obtain colloquial variants for a medical symptom (training data). Then the main aim was to map these colloquial variants and documents to a particular symptom name in order to improve search engines\textquotesingle ability to retrieve more useful content. Colloquial variants for a medical symptom were collected by presenting fixed symptoms to participants and asking them to generate queries that they would run to search and then determine the health problem. Because users tend to describe a symptom according to what they see and feel, a set-up was used in this study to provide participants with simulated experiences of having a given symptom. Several knowledge sources, such as Greek and Latin roots, medical dictionaries, Encyclopedias, synonyms, paraphrases, anatomy (body parts) and colours were used to understand a user\textquotesingle s colloquial query. Each query and its associated symptom were converted to two vectors based on the feature selected. A vector for a particular query was built by considering the query terms. The cosine similarity was measured to compare the similarity between a user query and its associated symptom. The labelled data was obtained by crowd sourced labels and Wikipedia redirects (approximately twelve redirects for each medical symptom). Two approaches were followed during the experiments.In approach 1 training and testing both were performed on the same fixed set of symptoms where images representing 31 symptoms and videos representing 10 symptoms were used by the classifier to choose the most appropriate class for a query. In approach 2 training (crowd sourced data) and testing (Wikipedia redirects) were performed on two different sets of symptoms.     

The authors observed that, during the first approach the improvement to the micro-average (total fraction of correct symptoms) was 61\% when images were presented to participants for generating queries (33\% improvement over the baseline). The improvement to the micro-average (total fraction of correct symptoms) was 85\% when videos were presented to participants for generating queries (26\% improvement over the baseline). During the second approach overall accuracy was obtained as 59\%. Therefore, overall this approach exhibited improvements over random guessing. The features, such as Encyclopedias, anatomy (body parts), synonyms, medical dictionaries, Greek and Latin roots and paraphrases were identified as top ranked features because they were able to match queries with their corresponding symptom name. 

\textbf{Limitations:} 

The authors observed several limitations of this study. Firstly, the classifier was unable to relate some of the truly positive query, symptom combinations. Secondly, this classifier was unable to better perform on some redirects from professional to highly professional language and on foreign language redirects. Thirdly, crowd sourcing participants of this study did not properly represent a sample of Internet users. Finally, it was kind of impossible for someone to correctly describe how a symptom really feels when that person is actually not experiencing the symptom. \\ 

\item {\textbf{Frances A Pogacar, Amira Ghenai, Mark D Smucker, and Charles LA Clarke. The positive and negative influence of search results on people’s decisions about the efficacy of medical treatments. In \textit{Proceedings of the ACM SIGIR International Conference on Theory of Information Retrieval,} pages 209-216. ACM, 2017.}}

This paper investigated the impact of search results on consumers\textquotesingle decisions of the effectiveness of medical treatments. 60 people participated in the study. They were presented with search results biased towards either accurate or inaccurate medical treatment information. Search results pages which were biased towards correct information had 8 out of 10 results correct. Search results pages which were biased towards incorrect information had 8 out of 10 results incorrect. Each search result/ document was presented with its title, url, and a snippet. In addition, the top most rank of a correct result was also controlled (either 1 or 3) to analyse the effect of ranks. The levels 1 and 3 were chosen because, according to prior research the mostly viewed search results were the first two results. Participants were asked to determine the effectiveness of ten medical treatments either with the help of search results or in a controlled condition (no search results). The two independent variables considered were the search results bias (correct and incorrect) and the topmost accurate search result\textquotesingle s rank (level 1 or level 3). The two dependent variables considered were the portion of accurate decisions and the portion of inaccurate (detrimental) decisions made by the participants. Data was also collected via questionnaires and as feedback on each decision made. Three categories, such as Helps (effective treatment), Inconclusive (unsure about the medical treatment) and Does not help (ineffective treatment) were used to categorise the efficacy of each treatment. Participants who performed under the control condition had to decide the effectiveness of two out of ten treatments. 

Two models, such as a complete model and a null model (no variable of interest) were used to analyse the significance of the dependent variables. The authors observed that when the rank 1 document was correct and the whole results page was biased in favour of accurate information the accuracy of decisions increased approximately up to 70\% and lowered the percentage (from 20\% to 6\%) of harmful decisions. This accuracy was significantly higher than the accuracy obtained while determining the efficacy of treatments under the control condition. In contrast, when results were biased in favour of incorrect information it notably reduced the accuracy of decisions (from 43\% to 23\%) and doubled the rate of harmful decisions. This performance with the presence of incorrect results was even worse than determining the effectiveness of treatments in the control condition (without search results). The impact of search results bias on the portion of accurate decisions and inaccurate (harmful) decisions was identified as statistically significant. The impact of the topmost correct rank on the portion of inaccurate (harmful) decisions was nearly statistically significant. While performing the tasks under the control condition, participants tended to categorize truly unhelpful treatments as inconclusive treatments. This clarified that participants were looking for positive information and did not want to classify a treatment as unhelpful. Hence, it was understood that search results with incorrect information can influence users\textquotesingle decision making. A positive correlation was identified between participants\textquotesingle prior knowledge of different health problems and medical treatments. It was also observed that more prior knowledge of the health issues led users to make more harmful decisions because users with prior knowledge were less likely to classify a medical treatment as inconclusive. Participants who classified medical treatments as inconclusive were identified as less confident of their final decision compared to classifying a treatment as unhelpful or helpful. Rank 1 search result of a results page was identified as important because some users clicked on it multiple times during the same session. It was also observed that, the more participants interact with the search results the more accurate decisions they make.

In conclusion, searchers performed better when they were exposed to correct information. On the other hand, they performed worse than not having search results, when they were exposed to incorrect information. Hence, it was concluded that search results notably affect consumers\textquotesingle decisions of the effectiveness of medical treatments. Therefore, the results of this study demonstrated that search engines are able to both assist and harm users. \\

\item {\textbf{Luca Soldaini, Andrew Yates, Elad Yom-Tov, Ophir Frieder, and Nazli Goharian. Enhancing web search in the medical domain via query clarification. \textit{Information Retrieval Journal, } 19(1-2):149-173, 2016.}}

This study investigated the usefulness of \textquotesingle query clarification\textquotesingle  when bridging the gap between lay terms and expert terms. Query clarification was performed by adding the most suitable expert expressions to search queries issued by lay people. Three different types of synonym mappings, such as Behavioural mappings (mapping user expressions to medical symptoms), MedSyn mappings (mappings focused on diseases and symptoms, and removed terms which were not related UMLS semantic types) and DBpedia  mappings (mapping lay terms to expert terminology based on Wikipedia redirect pages) were used to automatically improve user queries. Each synonym mapping mapped lay terms to expert expressions. For each of the user query three clarified queries were generated using the three different types of synonym mappings. Bing was used to retrieve search results for each of the four versions of the query. The most appropriate clarification (the clarification which best represented the medical concept expressed in user query) for a user query was selected by identifying the query clarification which was most likely (maximum probability) to appear in health-related Wikipedia pages. The Wikipedia pages with information boxes containing medically-related identification codes, such as MedlinePlus, DiseasesDB, eMedicine, MeSH and OMIM were considered as health related Wikipedia pages. In a situation where it was possible to map multiple query expressions to expert terms, the longest one was selected assuming that it entirely captured the information requirements of the user. In a situation where it was possible to map multiple query expressions with the same length, the one with the \textquotesingle highest conditional probability\textquotesingle was selected.  

The similarities and differences of each of the three different synonym mappings were analysed based on three factors, such as their different sizes, common terms between each set of synonyms and overlap between the results obtained by issuing each query prior and after query clarification.  The authors observed that Behavioural was the smallest in size and DBpedia was the largest in size. The synonym list of Behavioural mapping was almost completely contained within the DBpedia which had the largest synonym list.  Both Behavioural and DBpedia synonym mappings performed similar query clarifications. Queries clarified with MedSyn synonym mapping and unclarified queries retrieved quite similar results. Therefore, Behavioural and DBpedia mapping were identified as the most similar synonym mappings. Eighty lay people and  twelve medical experts participated in the study.  Fifty clarified queries were generated for this analysis. Each of these queries consisted of a symptom name, a drug name or a disease name. Each participant was instructed to answer twenty medical related multiple choice questions which were prepared based on each clarified query. Participants were also instructed to answer questions based on search results which were obtained by issuing clarified  and unclarified queries.

The authors observed that Behavioural synonym mapping was identified as the best performing synonym mapping because users were more likely to accurately answer questions with the use of results which were retrieved using queries clarified with Behavioural mapping. In addition, Behavioural mapping was the least preferred for incorrectly answered questions. Lay people preferred query clarifications more than experts because lay people were able to retrieve  more useful information to correctly answer health-related questions. In contrast, experts preferred to have their own original queries when retrieving information. However, the difference in success rates between the lay people and experts was identified as not significant. A strong correlation was also been identified by authors, between lay people and experts in terms of the success rate of each health information search task. The reason for this was because some questions were more difficult for both laypeople and experts when compared to other questions. The number of accurately answered medical questions increased with the trustworthiness of the web pages used to answer the question. In addition, clarified queries were able to retrieve more trustworthy resources as results. A logistic regression classifier was used to identify the optimal synonym mapping to perform query clarification. Furthermore, the authors also used a \textquotesingle simple multinomial logistic regression classifier\textquotesingle which was an ensemble of the methods of different query clarification types (synonym mappings) and the method with no query clarification. This was considered as an additional baseline and according to the observations of the authors, this logistic regression classifier performed the best when compared to all individual synonym mappings and unclarified queries. Hence, it was concluded that by selecting the most appropriate query clarification method (synonym mapping) to clarify each query, the clarification process can be further improved. 

In conclusion, users are able to retrieve more reliable results by clarifying their initial queries. These reliable results can then be used to correctly answer health related questions. Implicit query clarifications are also highly useful and most importantly, users do not need to know the correct medical terminologies which are being added to user queries while clarifying them. \\       
    	
\item {\textbf{Qing T Zeng, Jonathan Crowell, Robert M Plovnick, Eunjung Kim, Long Ngo, and Emily Dibble. Assisting consumer health information retrieval with query recommendations. \textit{Journal of the American Medical Informatics Association,} 13(1):80-90, 2006.}}


This study investigated how the use of a novel system known as \textquotesingle the  Health Information Query Assistant (HIQuA)\textquotesingle  impact on consumers\textquotesingle health information retrieval by providing supplementary query terms, which are related to consumers\textquotesingle  initial queries, to construct more effective queries. Three sources, such as consumer usage patterns, medical vocabularies and concept co-occurrences in medical publications were used to identify related terms which were then suggested to consumers. In order to provide query recommendations, the system mapped a user query to a medical concept defined in the \textquotesingle Unified Medical Language System (UMLS)\textquotesingle  and then identified other concepts which are related to the initial concept. These related concepts were identified based on three sources, such as \textquotesingle medical vocabulary\textquotesingle  source, \textquotesingle literature co-occurrence\textquotesingle source and \textquotesingle concept co-occurrences in consumer HIR sessions\textquotesingle. The degree of relevance was calculated by considering the frequency of concept co-occurrences in publications or frequency of relation occurrences in different medical vocabularies. The \textquotesingle semantic distance\textquotesingle between two health-related concepts was calculated using their co-occurrence frequency. The concepts with shortest semantic distances to the initial query concept were considered as related concepts. In addition to suggesting concepts based on consumers\textquotesingle  initial query concepts, the system also suggested concepts based on the \textquotesingle type\textquotesingle of initial query concepts, such as disease and procedure.      

213 people participated in the study. They were instructed to perform one of the two pre-defined health search tasks. The first task was related to finding five factors which increase the possibility of having a heart attack and the second task was related to finding three treatments for baldness. 50\% were randomly assigned with the first search task and the other 50\% were assigned to the second search task. Then 50\% were provided with the query recommendation function and the rest were not provided with it. Each participant was instructed to perform a health search task which was defined by themselves and search for a health related question according to their task definition. At the end of searches, data regarding participants\textquotesingle  satisfaction of the search tasks were collected based on a scale of 1 to 5. All the queries issued by participants and their recommendation selections were recorded. The impact of query recommendation function was evaluated by comparing 3 outcomes between the groups with and without the query recommendation function. It was identified that, health associated web experience and health status are statistically significant, therefore, they were used for further analysis between the two groups.

The authors observed that 85.2\% with query recommendations were satisfied with the search tasks and 80.6\% without query recommendations were satisfied with the search tasks. The satisfaction rate increased by 79\% for the participants with recommendations. However, it was identified that the association between the groups with and without the query recommendation function and user satisfaction are not statistically significant. The percentage of successful query submissions (a query which retrieves one or more relevant search results) increased by 79\% for the participants with recommendations. Therefore, the association between two groups and query success rate was identified as statistically significant. Participants who used suggested queries exhibited a higher success rate compared to the participants who typed-in queries. When considering the score of pre-defined tasks a higher normalized mean score was exhibited by the group without recommendations, but the difference in scores was not statistically significant.                     

Overall, the use of query recommendations, increased the number of successful queries. However, the impact of query recommendations on successfully completing a predefined health search task and the overall user satisfaction was not clearly identified, because all the participants did not make use of the query recommendations, the provided query recommendations were not useful for some of the participants with very low health literacy and the measurement of user satisfaction was very subjective. In general, this Health Information Query Assistant system was identified as beneficial for many users performing health related search tasks because the system was able to provide meaningful and user-centred query suggestions.

\textbf{Limitations}

Because of the diversity of the participants\textquotesingle population, the semantic distance measurements between concepts were less precise. It is not always feasible to map user query concepts to UMLS concepts, because UMLS concepts are presented in health care professionals\textquotesingle language. Some of the rules which were used in this study are not equivalent with some of the ranking methods, such as using an algebraic mean. The query success rates were determined by the authors instead of the participants and therefore, there is a possibility that authors may make mistakes while interpreting participants\textquotesingle retrieval goals. Time spent for each task was not analysed assuming that there can be many reasons for spending large amounts of time on some specific tasks. It is hard for this system to recognize users\textquotesingle  real information needs when a very short query is issued. The aspects, such as the quality and the credibility of the material used were not considered when developing the Health Information Query Assistant system. \\

\item {\textbf{Angela Ryan and Sue Wilson. Internet healthcare: do self-diagnosis sites do more harm than good? \textit{Expert opinion on drug safety,} 7(3):227-229, 2008.}}

Self-diagnosis sites usually consist of information, such as diagnosed conditions, possible diagnoses for different symptoms and advices to aid consumers in deciding whether to self-treat themselves or consult a doctor. These self-diagnosis websites suggest different diagnoses options for different symptoms reported by consumers. Usually these diagnoses options are provided with the help of private companies and recognised healthcare providers. Self-diagnosis websites were identified as different compared to the traditional text books because  accuracy and quality of content available in such websites are comparatively low, anyone can edit the information available on a website, there is no guarantee that the information available on a website is well edited and checked, some websites might financially exploit its users by trying to sell products and the risks associated with the home treatments and diagnostic acids suggested by these websites. 

There are several advantages of using well constructed quality information about symptoms provided by self-diagnosis websites. 

\begin{enumerate}
	\item The ability of the websites to provide immediate information compared to meeting a doctor by appointment which might take several days
	 
	\item Reduce the number of visits to the doctor by making use of home remedies
	
	\item Gain additional knowledge regarding when it is necessary to visit a doctor
	
	\item Gain precise knowledge about the symptoms prior meeting a doctor and clarify treatment options suggested by doctors with the use of that prior knowledge 
	
	\item Other subsidiary benefits, such as the availability of tips to stay healthy  
	
\end{enumerate}

There are several disadvantages or harms of using self-diagnosis websites. 

\begin{enumerate}
	\item Users of self-diagnosis websites can experience anxiety after reading inaccurate or false diagnoses information about serious conditions
	
	\item Unreliability of information available on self-diagnosis websites because the information might not have been reviewed by a health professional
	
	\item The possibility of making users buy prescribed drugs from Internet pharmacies
	
	\item Possibility of ignoring professional assistance and diagnosis because of the false reassurance provided by the self-diagnosis websites	  
	
\end{enumerate}      

At present consumers heavily use self-diagnosis websites to search and diagnose certain symptoms that they face and to identify different options available to treat those symptoms. Therefore, it is essential to make sure that information available on self-diagnosis websites to be well structured, edited and evidence-based. Reliable self-diagnosis websites can be emphasized by providing them with a \textquotesingle quality mark\textquotesingle which makes it easier for the users to choose high quality self-diagnosis websites. In addition, it is convenient for the users if healthcare, diagnostic and treatment services are provided in lay language because it is much easier for the less educated people to understand. \\      

\item {\textbf{Laurence Alpay, John Verhoef, Bo Xie, Dov Te’eni, and JHM ZwetslootSchonk. Current challenge in consumer health informatics: Bridging the gap between access to information and information understanding. \textit{Biomedical informatics insights,} 2:BII–S2223, 2009.}}

This study investigated how tailoring of health information (contextualized) impact on the access and understanding of the health information. The term \textquotesingle Contextualization\textquotesingle in health information stands for providing supportive information so as to thoroughly explain a health message. Hence, contextualization helps to improve users\textquotesingle understanding of different health concepts by reducing communication complexity. A website known as \textquotesingle SeniorGezond Website\textquotesingle was used for the evaluation. The study consisted of three main parts, such as a pre-test, tasks involved with the website and a post-test. Forty people participated in the study. Participants were randomly assigned to two groups in order to investigate their understandability of the health information. One group was presented with contextualized information and the other group did not receive any contextualized information. This website consisted of structured information about \textquotesingle fall\textquotesingle incidences with four levels of information. Level 1 consisted of \textquotesingle causes of falls\textquotesingle, level 2 consisted of \textquotesingle solutions\textquotesingle, level 3 consisted of \textquotesingle products and services\textquotesingle to aid people who experience this issue and level 4 consisted of \textquotesingle supportive facts\textquotesingle, such as information about insurance and addresses of the places where products and services which were mentioned in level 3 are available. The information consisted in level 2 was removed for the participants with no contextualisation because level 2 acted as the middle layer between level 1 (causes) and level 3 (products and services).

The authors observed that the use of contextualization significantly increased the lay users\textquotesingle understanding of health information by raising users\textquotesingle health literacy. Participants\textquotesingle cognitive style (the way they organize, filter, transform and process information) was also identified as a significant factor on thoroughly understanding health content. Participants also used their personal contexts (social and psychological) as a support to understand health information well. Gaps between informational contexts provided to users by health information websites and users\textquotesingle personal contexts that they use to enhance their understandability of health information were identified. 

According to these observations the importance of embedding contextualization techniques to health-related websites or tailoring of health information was identified. \\

\item {\textbf{Zequn Ye, Jacek Gwizdka, Carla Teixeira Lopes, and Yan Zhang. Towards understanding consumers’ quality evaluation of online health information: A case study. In \textit{Proceedings of the Association for Information Science and Technology,} volume 54, pages 838-839. Wiley Online Library, 2017.}}

This case study investigated how consumers evaluate the quality of online health information. This study used \textquotesingle eye-tracking\textquotesingle techniques and conducted post interview sessions in order to investigate different interface elements which are being used by consumers\textquotesingle to evaluate quality and also to investigate whether differences in individuals, such as eHealth literacy, personality, demographic facts and familiarity with health topics have any influence on consumers\textquotesingle behaviour. 12 lay people participated in the study. They were presented with five predefined tasks related to health information search. For each task they were shown 3 preselected web pages and asked if they would recommend those to their family or friends. 

The authors hand-picked results for two participants (because of opposite patterns shown in eye glaze and fixation). They observed that, the first participant had viewed the web page content as a \textquotesingle F-shaped\textquotesingle pattern with fixations focused on the main text content. In contrast the second participant had viewed the web page content as an \textquotesingle atypical pattern\textquotesingle with fixations scattered all over the web page. Different choices made on this web page, time taken to complete the task, time spent on the web page, number of links clicked by the participants and their background information, such as \textquotesingle eHEALS scores\textquotesingle and \textquotesingle TIPI scores\textquotesingle were compared to analyse the performance of the two participants. In conclusion, the authors noted that the participant with lower eHEALS score tend to spend a longer time on the task and on the web page than the other participant with a higher eHEALS score. Eye-movement patterns were dependent on facts, such as familiarity with the health topic, health literacy and demographic factors.               

\textbf{Limitations}

Their observations are limited by the fact that just 2 participants were studied. Also, in their analysis the authors did not account for the possibility that a participant with a non-focused eye glazing was distracted and not interested in the task. \\

\item {\textbf{Carla Teixeira Lopes and Tiago Almeida Fernandes. Health suggestions: A chrome extension to help laypersons search for health information. In \textit{International Conference of the Cross-Language Evaluation Forum for European Languages,} pages 241-246. Springer, 2016.}}

This study investigated how providing health query suggestions to consumers will be useful in successful health information retrieval. The study provided query suggestions in both Portuguese and English based on the initial query\textquotesingle s language. Health suggestions were provided in three search engines which are Google, Bing and Yahoo. The ultimate aim was to provide consumers with a mechanism which aids consumers in retrieving high-quality health information which will also fit with their health expertise. The approach of the study was to implement health query suggestions in Google Chrome as a \textquotesingle Google Chrome extension\textquotesingle and then to reach the three search engines. A health suggestion query panel was presented when matches were identified between health suggestions and user queries. Users were able to search for suggested queries (a new search), switch search engines (same search in a new search engine) and navigate to an options page using this query panel. Two modules were used in the system. \textquotesingle Suggestion engine\textquotesingle generated suggestions based on Consumer Health Vocabulary (CHV) which contains links between everyday health-related terms and technical terms. A \textquotesingle Login engine\textquotesingle was used to study user health search behaviour by tracking facts, such as the amount of time spent on the health suggestion query panel, different web pages visited, time spent on each web page, number of scrolls and number of clicks performed on the query panel. 

36 students were recruited to participate in this study. The first group of students received assistance via health suggestions. Second group was unassisted. Health topics for the task were selected randomly from the topics mentioned by 20 laypeople (no medical expertise). Participants performed 4 tasks. For each task they formulated 3 queries. Then they were asked to issue each query and download the most useful/relevant documents out of the top 10 search results of each query. 

After completing each task assisted participants explained their utilization of health suggestions (in which way they used health suggestions), the reasons for determining those health suggestions as useful  and assessed the utility of those health suggestions. The number of relevant documents retrieved and users\textquotesingle opinions towards the task were used to determine the success of the search. User satisfaction was scaled from 1 to 5. According to the observations, 4 students (11\%) categorized satisfaction as 3, 26 students (72\%) categorized satisfaction as 4 and 6 students (17\%) categorized satisfaction as 5. The authors also evaluated health suggestions based on 4 research questions:

\begin{enumerate}
	\item \textbf{How are suggestions used?:} According to assisted participant, they were able to obtain query suggestions for 71\% of the executed initial queries. Out of these query suggestions, 27\% was used by directly clicking on the health suggestion, 15\% was used to extract query terms from health suggestions and 4\% was used to extract query terms from several health suggestions
	
	\item \textbf{Why are suggestions used?:} Five main reasons were observed. In 35\% of the cases because they presented synonyms; in 37\% of the cases because they presented alternatives in medico-scientific terminology; in 24\% of the cases because they suggested English terms; in 3\% of the cases because of the lay terminology 
	
	\item \textbf{What is the method used by users to evaluate the usefulness of the suggestions provided by the system?:} A scale in a range of 1-3 was used. 35\% of the queries were considered useful, 33\% partially useful and 29\% were assessed as useless 
	
	\item \textbf{Will the suggestions lead users to a more effective (successful) search process?:} The average number of useful (relevant) documents downloaded by participants were observed. The average was 16.3 for the assisted group and 14.1 for the unassisted group. The overall task success was scaled from 1 to 5. The mean value was 4 for the assisted group and 5 for the unassisted group. However, this difference was not considered as significant 
\end{enumerate}

In conclusion, health query suggestions from the tool were well accepted by the users. Both lay terms and medical-scientific terms which were provided with health suggestions increased the number of successful searches. The utilization of an approach which uses both multilingual and multi-terminology resources is important and useful to retrieve a huge number of relevant documents. \\    

\item {\textbf{Ricardo Silva and Carla Lopes. The effectiveness of query expansion when searching for health related content: Infolab at clef ehealth 2016. In \textit{CLEF (Working Notes),} pages 130-142, 2016.}}

This study investigated how query expansion (supplementing the original query with additional terms) will aid laypeople in improving initial queries and then the overall retrieval performance. The impact of query expansion can be determined by evaluating the set of ranked list of documents retrieved from a recommended test collection. Two sub tasks, ad-hoc search (treating each query individually) and query variation (treating a group of query variations as one query) were performed. ClueWeb12 B13 Dataset was used as the document collection. Query generators created initial queries based on consumer health posts. \textquotesingle Terrier\textquotesingle was used as the indexing technique. A set of different sources and methods were used to pick terms to be added to the initial queries. \\        

\textbf{Baseline:} \textquotesingle BM25 term weighting model\textquotesingle was used to score and rank medical documents. Ranking was performed based on documents\textquotesingle relevance to the issued search query. 

\textbf{Pseudo Relevance feedback:} This method modified queries based on the top-ranked documents retrieved by the baseline approach. Existing query terms were re-weighted to pick useful terms and to drop useless terms. 

\textbf{Query expansion using the Medical Text Indexer:} This method linked query text to the Medical Subject Headings (MeSH) vocabulary. This provided additional related concepts. All these concepts were appended to the initial query.  

\textbf{Query expansion using Wikipedia:} Wikipedia was chosen as a source because it contains health information and medical terms in lay language. Two methods were used to obtain expansion terms from Wikipedia. Term frequency was used to extract the most frequent terms related to the MTI concepts, from the Wikipedia articles. For each article the most frequent terms (5, 10, 15) were chosen. Link analysis was used to recognize matching articles and expansion terms were extracted from their titles. Jaccard similarity coefficient was used to filer out all the irrelevant articles. Titles of the articles with Jaccard similarity coefficient value above 0.25, 0.50 and 0.75 were chosen.             

\textbf{Query expansion using MedlinePlus:} Information from the infoboxes of the Wikipedia pages were used to select related MedlinePlus pages. The sections, such as Causes, Symptoms, Treatment, Possible Complications and Alternative Names of the MedlinePlus pages were used for the query expansion process. The 5, 10 and 15 most frequent terms from each relevant section were chosen.       

\textbf{Query expansion using the ICD-10:} Information from the infoboxes of the Wikipedia pages were used to select related ICD-10 pages. Information regarding diseases or symptoms which are related to initial query concepts were utilized for query expansion process. For each ICD-10 page the most frequent terms (5, 10, 15) were chosen to be appended with the initial query.
 
\textbf{Query expansion using Latent Dirichlet Allocation over Wikipedia:} This method generated different latent topics represented by the text in Wikipedia articles with different MTI concepts. A combination of 3 topics with 1, 5, 10 words and 1, 5, 10 topics with 5 words were chosen.     

\textbf{Query expansion using Unified Medical Language System:} This method extracted terms from the UMLS definitions which were associated with the MTI concepts. For each MTI concept the most frequent terms (5, 10, 15) were chosen from the UMLS definitions for query expansion.  

\textbf{Readability:} Three readability matrices, such as SMOG, FOG and Flesch-Kinciad representing the \textquotesingle educational grade level\textquotesingle needed to understand a document were used. Three formulas where each formula was a combination of one readability matrix and relevance scores were used to re-rank the documents retrieved using expanded queries. 

Three runs were submitted. Firstly the Baseline approach; secondly the Wikipedia Link Analysis with a Jaccard similarity coefficient above 0.50; finally the Latent Dirichlet Allocation with 3 topics and 5 words. SMOG readability metric and one combination formula were used for re-ranking. However, because of the lack of information about the relevance and readability assessments for the test collection, it was hard for the authors to compare results with the baseline for each sub task and to make any conclusions regarding the results which have been obtained by following different approaches.            

\textbf{Limitations}

One major limitation was the unavailability of the relevance and readability assessments for the test collection at the time of reporting the study. Another limitation was that Medical Text Indexer results were machine generated, so there was a possibility of generating irrelevant concepts. \\    

\item{\textbf{Laurel Graham, Tony Tse, and Alla Keselman. Exploring user navigation during online health information seeking. In \textit{AMIA Annual Symposium Proceedings,} volume 2006, page 299. American Medical Informatics Association, 2006.}}

This study investigated user navigation behaviour while seeking for online health-related information. ClinicalTrials.gov web site\textquotesingle s log data were obtained over a three month period. Data was extracted from log files using Transaction Log Analysis (TLA). Investigation focused on online search behaviour, query failures, navigations, browsing strategies, clicks performed on links, initial query generations and other task related actions (logins performed). The obtained log data included information about clients (unique IP addresses), search sessions (sequential actions performed during each search) and requests made by those clients (eg: web page request). Web pages were aggregated into functional categories, such as search, browse, view results, view study etc. for the analysis. Page reviews, referral frequencies, page transition frequencies (presented as a single transitions table by computing the frequencies), navigation path frequencies (determined using algorithms), click stream data, participant comments and usage statistics were assessed. Then a pilot user study was conducted. 7 lay consumers were presented with 2 hypothetical scenarios; \textquotesingle sleep apnea\textquotesingle and \textquotesingle Parkinson\textquotesingle s disease\textquotesingle. They were assigned to either one of the scenarios to perform the search. Any online resource could be used as their choice. 

By analysing ClinicalTrials.gov log data the authors observed that clients had frequently requested (40\%) \textquotesingle View Study\textquotesingle  pages. In addition, \textquotesingle View Study\textquotesingle  pages were also commonly used by clients to enter and exit ClinicalTrials.gov. \textquotesingle View Results\textquotesingle  pages were the second most frequently requested (25\%). In 69\% of the sessions clients have used search engines or government sites, which were considered as external web sites, to commence their searches. The top five referring web sites were Google (41\%), NIH.gov domain (18\%), Yahoo, MSN and AIDSinfo which accounted for 66\% of all referrals. View Study (39\%), View Results and Opening Screen (homepage)  (24\%) pages served as web site entry and exit points. In 40\% of the cases users either have viewed two studies consecutively or have viewed a study and the exited. In 9\% of the cases users have clicked on one of the studies from the results list. These were identified as the most frequent moves between pages indicating that users were moving among a limited set of pages (only a few user online navigation activities). 8 common user navigation patterns were identified. These patterns revealed that users were tend to access ClinicalTrials.gov directly via \textquotesingle View Study\textquotesingle  pages. Search and browse features were directly used. Time was spent on exploring the site and refining search queries. Other available site features \textquotesingle Search within results\textquotesingle  and \textquotesingle Resources\textquotesingle  were not used.

The analysis of the pilot study revealed that, during their search session, 5 out of 7 consumers were keen to look at, at least one web page on ClinicalTrials.gov. Consumers also used search engines (two out of five), MedlinePlus (two out of five) and health web sites (one out of five) to access ClinicalTrials.gov.   

According to this study, the \textquotesingle View Results\textquotesingle  page was frequently used by consumers to enter and exit, in contrast to TLA data with most frequent entry/exit point as \textquotesingle View Study\textquotesingle  page. For search queries with general terms users navigated to \textquotesingle View results\textquotesingle  page and for search queries with specific terms users navigated to \textquotesingle View Study\textquotesingle  page. The name recognition, \textquotesingle dot-gov\textquotesingle domains and keywords were used by participants to judge the relevance of the web sites. All the participants\textquotesingle were satisfied about the overall search task. Pilot user study\textquotesingle s entry and exit points were very much alike to the entry and exit points which were identified by analysing log data. Most moves were occurred within the pages opening screen, view results and view study. It was also observed that, search engines were able to direct consumers to more pages with results and individual studies. 

In conclusion, the majority of the consumers use web-based search engines to access low-level pages (View Study). Therefore, it is recommended to make important information, such as background information and other search features easily accessible to users by placing their links on low-level pages. In addition, visible local maps are able to retain users on web sites. \\ 

\item {\textbf{D Thenmozhi, P Mirunalini, and Chandrabose Aravindan. Decision tree approach for consumer health information search. In \textit{FIRE (Working Notes),} pages 221-225, 2016.
}}

This study investigated how to determine whether sentences of a document is relevant, with the use of a consumer generated health search query and a document related to that user query. The approach of this study was to use a machine learning technique to categorize retrieved health-related information as relevant or irrelevant based on the issued query. A decision tree was created by following 4 steps. 

\begin{enumerate}
		\item Preprocessing the given text by removing punctuations and adding annotations for parts of speech (verbs, nouns, adjectives). Nouns and adjectives in training data were selected to use as the features of the decision tree
		 
		\item Feature selection or extracting features for training data. Two variations were used. In the approach without feature selection, the linguistic features were used without explicit feature selection. The features were lemmatized to obtain the root form of each feature. In the approach with x2 feature selection, a chi-square value was computed to pick the important features from the linguistic features. A maximum x2 statistic value was calculated to select the features with a strong dependency on the categories.The occurrence or non-occurrence of each feature in relevant and irrelevant instances were determined.The expected frequency for each feature was also calculated using the observed frequency
		
		\item Using the selected features of training data to build the decision tree which is a classification model. For the approach without feature selection the features vector of training data was used to extract features from the test data with unknown labels. For the approach with feature selection, the features with x2 statistic value greater than 3.841 were considered as significant features and were used to build the classification model
		
		\item Predict the relevance of a document either as \textquotesingle relevant\textquotesingle or \textquotesingle irrelevant\textquotesingle using the built decision tree model   
\end{enumerate}

The data set consisted of 5 queries. Training and test data was available for each of these 5 queries separately. The number of nodes for the x2 feature selection tree were considerably lower than that of tree without feature selection. The reduction in size of the tree for x2 feature selection was statistically significant. The authors observed that the approach without feature selection had higher cross-validation accuracy for each query than for the approach with x2 feature selection. However, x2 feature selection method had a 2.23\% higher test data accuracy compared to the method without feature selection. The improvement in performance for the x2 feature selection method was statistically significant. It was shown that the feature selection approach is able to significantly reduce the size of the decision tree with no change to the performance of the decision tree. In conclusion, since x2 feature selection model can categorize health-related documents as relevant and irrelevant successfully, it can be used for filtering out irrelevant health related documents prior presenting the results to consumers/ laypeople. This would help consumers in retrieving more relevant health related documents according to their issued query. \\

\item {\textbf{Ramona Broussard and Yan Zhang. Seeking treatment options: consumers’ search behaviors and cognitive activities. In \textit{Proceedings of the Association for Information Science and Technology,} volume 50, pages 1-10. Wiley Online Library, 2013.
}} 

This study investigated consumers\textquotesingle exploration of treatment options in both behavioural and cognitive perspectives. Two research questions were addressed;

\begin{enumerate}
		\item The behaviours of consumers when seeking for medical treatment options online
		
		\item The cognitive activities performed by consumers during their searches: finding, selecting and evaluating information  
\end{enumerate}

The approach of the study had two parts; 1. participant observation; 2. post-session interview. 40 people participated in the study. Two interfaces were used; 1. a simple web-based search engine interface with a basic search box and the results were presented as a ranked list; 2. a \textquotesingle Scatter/Gather enabled\textquotesingle  search interface with a basic search box, but the results were grouped into a number of clusters. These clusters were ranked based on their size. 20 participants were assigned to each interface and were asked to search treatment options for migraines. Four different categories of data were collected:

\begin{enumerate}
		\item Participants\textquotesingle demographic factors and experiences in searching health-related information
		
		\item Transaction log data including, length of sessions, queries issued, visited web sites, participants\textquotesingle evaluation of the utility and relevance of the visited web sites
		
		\item Participants\textquotesingle ratings for the required mental efforts to successfully perform the task and satisfaction of the search performance 
		
		\item Information gathered via playback interviews (playback of the search process): selection of keywords, query reformulation and examination, selection of particular search results, and evaluation of search results   
\end{enumerate} 

Two different interface groups were pooled together for further analysis because they had no difference in information, such as session length, number of queries submitted, and sites visited.                       

The authors observed that the mean values for web experience, health search experience and health search frequency were 13.2 (years), 3.7 (years) and 2.8 (times per month) respectively. Average ratings for the perceptions of the task were; familiar: 2.3 (somewhat familiar); easy: 2.5 (somewhat difficult); effortful: 2.7 (medium amount of effort) and satisfied (with results): 4.2 (generally satisfied). Three aspects were analysed for search behaviours:

\begin{enumerate}
	\item Session length: average session length was 10.25 minutes
	
	\item Basic query behaviour: average number of queries issued was 4.2 and average query length was 3.8 terms
	
	\item Web sites visited (60\% of all visits): frequently visited medical-specific websites: WebMD (26 visits), livestrong (22 visits), mayoclinic (17 visits), migraines.org (19 visits) and migraines.com (10 visits); frequently visited general-purpose sites:  ehow (22 visits), Wikipedia (18 visits) and buzzle.com (5 visits); sites with checkout behaviour (73.2\% of all the visits): medical-specific, evidence-based and general-purpose sites; sites visited twice: 50\% general purpose sites and 50\% medical-related sites  
	
\end{enumerate}	 

44 unique websites were visited only once. Four aspects were analysed for cognitive activities. 

\begin{enumerate}
		\item Query formulation and reformulation (89 times): specification (22), generalization (9), parallel move (25), new concept (27) and rephrase (6)
		
		\item Examination of search results: trial and error approach was taken to decide whether to select a search result or not. The rank of a result, familiarity of websites and the validity of information were important when selecting a search result
		
		\item Judgement of sources: the design (clarity, simplicity, clean and well-structured text with images), readability (text without jargon), completeness (pros and cons, side effects and expensiveness of the treatments), and credibility (depending on participants\textquotesingle experience and the content of websites) were used to judge sources
		
		\item Search timeline -cognitive development: one pattern was to start with a general search and gradually move towards more specific information; another pattern was to start from confirming information and gradually move towards novel information and validating that novel information (double check a fact)    
\end{enumerate}

In conclusion, consumers\textquotesingle exhibited behaviours, such as submitting short queries consisted of misspelled query terms and selecting results from the first results page when they seek for medical treatment options online. In terms of the frequently visited web sites (20\%), participants relied on both medical-specific and general web sites, and viewed web sites with its name containing the medical condition searched. Consumers performed cognitive activities, such as starting the search with general concepts and gradually moving towards different treatments and aspects of each treatment, and selecting search results based on familiarity while they searched for different treatment options. Search engine rankings and familiarity of the web sites were more important when selecting results when compared to trustworthiness, quality or usefulness of the information contained in the websites. Therefore, search engines need to be designed in a way that they will support different consumer behaviours.  

\textbf{Limitations}

The task used in this study was artificially set-up by the authors and therefore, did not demonstrate users\textquotesingle real needs. Only one treatment option based task was used in this study and therefore, users\textquotesingle behaviours might have been influenced by the task\textquotesingle s nature. A lab setting was used for this study and it was not a real environment where treatment options are actually being searched.\\ 

\item {\textbf{Alla Keselman, Allen C Browne, and David R Kaufman. Consumer health information seeking as hypothesis testing. \textit{Journal of the American Medical Informatics Association,} 15(4):484-495, 2008.}} 

This study investigated the most common patterns consumers follow when searching health information, depending on their initial theories, search strategies and comprehension. These patterns were then categorized as successful and failed. A framework was used for the study. This framework had two perspectives, such as hypothesis testing perspective and Human Computer Interaction (HCI) perspective to obtain further insight regarding difficulties faced by consumers when searching and understanding health information. Four states were considered in this study. 

\begin{enumerate}
		\item Beginning state which constituted of background knowledge and primary hypotheses (perceived information need)
		
		\item Search goal
		
		\item Shaping the search goal by search action steps
		
		\item Evaluation of retrieved information    
\end{enumerate}


20 lay individuals with different levels of education participated in the study. They were presented with a hypothetical scenario which described symptoms of \textquotesingle stable angina\textquotesingle. They were asked to do two tasks: 1. Semi-structured interviews where possible causes of the symptoms were discussed and 2. Search MedlinePlus to seek information on the disease. Information gathered via semi-structured interviews was compared with a reference model built in association with \textquotesingle stable angina\textquotesingle. The search processes were categorized as action-related and competency processes. Action-related represented actions performed or steps of information seeking path. Competency processes included facts, such as domain Knowledge, strategies to perform searches, knowledge about resources, meta-knowledge and knowledge about language, including spelling and vocabulary. Each action step was assigned with one of these competencies. Each search process was integrated with the corresponding verbal transcripts obtained during semi-structured interviews for further analysis (examine search strategies and visualize trends across participants’ behaviours). 

The authors observed that overall participants\textquotesingle understanding of the scenario was incorrect and imprecise. Three aspects in which participants\textquotesingle understanding was differed from the reference model was identified. Firstly, the main concepts were analysed. \textquotesingle Coronary artery disease(CAD)\textquotesingle  and \textquotesingle angina\textquotesingle were in the reference model, but no participant mentioned them. The third concept was atherosclerosis in the reference model, but only 3 participants mentioned a lexical form of this term. For the majority of the participants \textquotesingle heart attack\textquotesingle was the primary hypothesis. Blockage of blood vessels\textquotesingle  was the primary cause of such heart problems. Some other causes, such as \textquotesingle irregular heart beat\textquotesingle  and \textquotesingle electrical problem with the heart\textquotesingle  were also mentioned by the participants. In addition, some other health issues, such as \textquotesingle arthritis\textquotesingle, \textquotesingle asthma\textquotesingle and \textquotesingle diabetes\textquotesingle, which were not directly related to heart problems, were also mentioned by the participants. Secondly, symptoms\textquotesingle  grouping was analysed. Reference model indicated that all symptoms were related to one health condition. In contrast, participants mentioned \textquotesingle nausea\textquotesingle and \textquotesingle dizziness\textquotesingle as irrelevant to a heart (cardiac) problem. Finally, symptoms\textquotesingle  characteristics were analysed. The short length of time the pain lasted, its connection with exertion and its responsiveness to rest were not noted by the participants. In contrast the reference model identified the significance of all these factors. 

Participants were categorized into three main groups, such as \textquotesingle Verification-First\textquotesingle, \textquotesingle Problem Area Search-First\textquotesingle and \textquotesingle Bottom-Up\textquotesingle for the analysis of Information-Seeking Processes. Firstly, the \textquotesingle Verification-First\textquotesingle  group was analysed. It represented 8 participants (40\%). They started the search process by trying to substantiate a particular illness related to heart attack. The only strategy used was \textquotesingle verification\textquotesingle. One participant correctly concluded the scenario as angina. Majority of them (7) incorrectly concluded the scenario as related to heart attack by only considering the similarities between descriptions of heart attacks and symptoms in the scenario. Participants tend to ignore characteristics of symptoms which were seen as unimportant and this behaviour was identified as \textquotesingle selective perception bias\textquotesingle. The behaviour of stating that information confirmed their hypothesis depending on information of a site they visited, was identified as \textquotesingle confirmation bias\textquotesingle. The behaviour of stopping search just after reviewing only one content topic was identified as \textquotesingle premature search termination bias\textquotesingle. 

Secondly, the \textquotesingle Problem Area Narrowing-First\textquotesingle group was analysed. It represented 5 participants (25\%). They started the search process with problem area search. Therefore, participants in this group had both \textquotesingle Area\textquotesingle hypothesis (start searching with queries) and \textquotesingle Assorted\textquotesingle hypothesis (browsing the web site index). A behaviour of switching approaches (\textquotesingle Problem Area Narrowing-First\textquotesingle approach to \textquotesingle bottom-up\textquotesingle approach) was exhibited by one participant. One participant had switched to a \textquotesingle bottom-up\textquotesingle approach during this search process. Sites with specific disease information were visited. Participants had left without a conclusion rather than providing incorrect conclusions. The behaviour of picking text from the scenario (pain episode only lasts 2–3 minutes) and making conclusions (very minor heart attacks) based on that was identified as \textquotesingle selective perception bias\textquotesingle. Finally, the \textquotesingle Bottom-Up First\textquotesingle  Cluster was analysed. It represented 7 participants (35\%). Some started the search process without a specific hypothesis. The attempts to search for a general-purpose diagnostic tool were unsuccessful. Some participants had  switched to other \textquotesingle hypothesis-driven strategies\textquotesingle, but made incorrect heart attack conclusion. Neither of the participants in this cluster had made the exact correct conclusion as angina. The behaviour of trying to match description of a heart attack with the facts mentioned in the scenario was identified as \textquotesingle selective perception bias\textquotesingle.                      

Other factors, such as domain knowledge was important for setting goals and information evaluation. Domain understanding was important for determining the path of the search and for better understanding of the results. Knowledge about resources, strategies used to perform searches and meta-knowledge were important for navigational actions. Participants were able to understand the medical symptoms explained in the search scenario disregard of their education level. Highly educated participants were more familiar with MedlinePlus, used efficient search strategies and made constructive comments (eg: authoritativeness of a source).  

In conclusion, insufficient domain knowledge (setting information goals and evaluating retrieved information), imprecise search queries entered and the inconvenient configuration of the web resources were identified as main causes of the problems faced by participants when searching for information and selecting correct results. An incorrect hypothesis can cause the search of irrelevant resources. Prior hypothesis and background knowledge influences hypothesis generation, evidence interpretation and evaluation of retrieved information. Two aspects of MedlinePlus interface influenced participants\textquotesingle  search process. 1. Its\textquotesingle  index was not able to explicitly relate lay and professional terms and 2. organization of search results lists. Searching processes with a specific preconception ended with incorrect conclusions. Searching processes without a specific preconception ended without any conclusion.The importance of identifying the difficulties faced by consumers\textquotesingle  is that health information website(consumer health sites) designers can try eliminate these difficulties by providing support in places where consumers were tend to behave erroneously. In information portals like MedlinePlus it is worth providing query suggestions, presenting information in an organized manner and suggesting consultation with a health professional. In individual websites it is worth addressing needs of targeted users with consumer-friendly terminology. In education tools it is worth to educate consumers on how to properly formulate search queries, how to evaluate retrieved information and inform them to prevent terminating searches prematurely.

\textbf{Limitations}

The hypothetical nature of the scenario (possibility of affecting users\textquotesingle  motivation), the difficulty of the search task and the likeness of angina\textquotesingle s symptoms and heart attack\textquotesingle s symptoms (ambiguity in the description of the scenario) were identified as limitations and potential reasons which led this study to result in a very low success rate. \\

\item {\textbf{Yan Zhang, Peiling Wang, Amy Heaton, and Heidi Winkler. Health information searching behavior in medlineplus and the impact of tasks. In \textit{Proceedings of the 2nd ACM SIGHIT International Health Informatics Symposium,} pages 641-650. ACM, 2012.}} 

This study investigated health-related information search behaviour of consumers in MedlinePlus which consisted of a basic search engine, and the impact of the number of concepts involved in the search behaviours. Three research  questions were examined:

\begin{enumerate}
		\item The way lay people search in MedlinePlus?
		\item The way lay people browse in MedlinePlus?  
		\item The influence of search tasks on users' interaction strategies?
\end{enumerate}

The difficulties faced by consumers when searching and how they handled these problems were also examined. 20 undergraduate students who have not used MedlinePlus before participated in the study. Three search tasks were given to each participant. The first task was associated with finding positive and negative arguments in favour of the use of \textquotesingle marijuana\textquotesingle . The second task was associated with finding connections among two types of diabetes (Type I and Type II) and hypertension. The third task was related to finding various information associated with liver and kidney, such as functions of them, the role and purpose of insulin and liver and kidney diseases caused by insulin if there are any. Prior performing the tasks information regarding participants\textquotesingle spatial ability and demographic factors were collected. Participants\textquotesingle   searching and browsing behaviours were examined during the search tasks. After each task participants\textquotesingle opinions on difficulty of the task, required mental effort and satisfaction of the search performance in MedlinePlus website were collected.                

The authors observed that participants\textquotesingle  spatial ability score ranged between 6.8 to 17.6. Internet experiences ranged between 6 to 13 years. Online health information searching frequency was on a yearly or monthly basis. Session lengths varied between 11.63 to 29.01 minutes. The time spent, difficulty and the required mental effort for task 3 were significantly higher than the others. Satisfaction of task 3 was the lowest among three tasks. Four aspects were considered for the analysis of search behaviour. 

\begin{enumerate}
		\item \textbf{Different features of queries:} task 3 had the maximum number of queries per person which was statistically significant and task 2 had the maximum average query terms which was not statistically significant
		
		\item \textbf{Search terms:} three types of search terms were identified; (1) meaningful search terms/ keywords; (2) stop words; (3) search operators, such as AND and OR
		
		\item \textbf{Query reformulation:} two factors were considered. (1) Executed actions which modified the initial queries. Three actions were identified. Concept related changes: add (28.9\%), delete, repeat/re-execution, replace (30\%) and change to a new concept/ switching topic (20.7\%). Task 3 had the maximum count of query reformulations. Terms\textquotesingle form related changes: A term\textquotesingle s form was changed accordingly and misspelled query terms were corrected. Conceptual relationships (boolean operators): changing boolean operators and altering the order of the words. (2) Subsequent conceptual changes to the queries. Four categories were identified; specification; generalization; parallel movement (two aspects of the same query); substitute query terms with synonyms (words with similar meaning)
		
		\item  \textbf{Accessing and evaluation of results:} two options, such as directly access the results (results from Health Topics section and Encyclopedia) and filter out results. Majority directly accessed the results rather than filtering the results. A few also used the options to filter out the results. The results were evaluated by scanning, checking the authors, checking the source of information and the structure of the content in the resources 
		 
\end{enumerate}

Two aspects were considered for the analysis of browse behaviour. 

\begin{enumerate}
	\item \textbf{Accessing different resources:} participants accessed various resources, such as \textquotesingle Drugs \& Supplements\textquotesingle list, \textquotesingle Encyclopedia\textquotesingle, dictionary, health Topics section, news section and directories
	
	\item \textbf{Accessing related topics:} in-text and related topics list hyperlinks available on MedlinePlus health topic pages were used for task 3  
\end{enumerate}

In conclusion, session length of a search process increases with the complexity of the task. Most query reformations were related to conceptual changes (85\% ) while making queries specific,  followed by making queries more general, switching to a new topic and making parallel movements. Query re-executions were performed the most as query iterations. Participants with higher understanding of the scenario explained in the search task also searched for concepts associated with the concepts mentioned in the scenarios. The use of stop words demonstrated participants\textquotesingle preference of using natural language during search. Generally searching and browsing strategies were used as a combination mostly when performing task 3 (relatively higher complexity). Only one strategy was used mostly when performing the first two tasks (relatively lower complexity). Task 3\textquotesingle s search process patterns were identified as more diverse and complex compared to other two tasks and the searches were performed iteratively as well. This indicated that task difficulty has the ability to impact users\textquotesingle interaction strategies and patterns. Searching strategies were preferred the most because it was easier to perform than browsing strategies. 

Encyclopedia, Health Topics, dictionary and links to related topics were used to gain preliminary understanding of the scenarios in complex tasks, such as task 3. Therefore, the need for functions to detect misspellings and provide query suggestions with automatic synonym expansions, a hierarchical terminology structure to assist participants in selecting query terms for query construction, suggestions for alternative moves when reformulating queries, the visibility of information architecture to users, advanced search function to support exploring relationships between multiple health concepts, a backtrack function to access search histories, techniques to match lay terms with corresponding medical terms and encouragement for evaluating search results were identified when designing consumer health-related information websites, so as to aid consumers when performing search tasks with varying complexity.         

\textbf{Limitations}

This study recruited undergraduate students who had vast web search skills but low experience in health information search. Predefined tasks were used for the study rather than focusing on users\textquotesingle real needs. It is important to perform the same tasks in different consumer health information websites to understand how different information architectures or structures and interfaces influence interaction behaviour. \\

\item {\textbf{Elaine G Toms and Celeste Latter. How consumers search for health information. \textit{Health informatics journal,} 13(3):223-235, 2007.}}

This study investigated consumer health information search behaviour online. Three aspects were considered. 

\begin{enumerate}
	\item The way people state their information needs
	\item The way people choose useful resources from result lists
	\item The way people evaluate web pages which were determined as relevant to the health search task  
\end{enumerate}

48 young, educated adults who have used the web before participated in the study. Search tasks were performed in Google. A set of directory categories were appended to Google so as to provide participants with an additional scan option. Four tasks were given to participants. The first two tasks were specified for the participants; the first task was related to finding groups of people to whom taking flu shot might be acceptable or unacceptable; 2. the second task was related to finding a website which consists of information about consequences of second-hand smoke. The participants were able to personalize the second two tasks; the first personalised task was related to finding two treatments for any health-related matter as per participants\textquotesingle interest; the second personalised task was related to finding two advantages and disadvantages of consuming large doses of any drug/ treatment as per participants\textquotesingle interest. Information regarding participants\textquotesingle  demographic factors and web search experience were collected. Each participant was assigned to one of the four tasks. Information, such as participants\textquotesingle familiarity and expertise with the search topic, actions performed while searching for the topic (issuing queries, selecting categories, examining pages and selecting results), and participants\textquotesingle  perception on completing the task was collected. Judges were recruited to assess the pages which were noted as relevant by the participants. Completeness of the tasks were assessed by examining the pages which were recommended as useful by the participants.

The authors observed that the majority of the participants were either \textquotesingle somewhat familiar\textquotesingle or \textquotesingle highly familiar\textquotesingle (66\%) with the topics of the fully defined search tasks. Average number of queries issued per each task was 1.3 queries. Average number of keywords in each query: 4.3 keywords. 63\% used only the search box to type a query. 6\% of the participants used the categories. The rest of the participants used both queries and categories. On average 4.5-9 minutes were spent per task. Time was mostly spent on interpreting results page to comprehend the information presented on the web page. Formulation of queries and selection of categories were analysed. To perform the first two tasks participants issued various queries and selected various categories. For task 1, 12 participants issued 23 entries. However, for task 2 less variability in query content was observed. 

Query formulation was a quick process. The most common query formulations was to search for keywords either via a search engine or on a website. Keyword were also searched as a \textquotesingle trial and error\textquotesingle approach. With the use of keyword search participants were able to control the search process and results. They made searches more robust (adding specific words while searching), restricted the search by specific sources of information and the reliability of the sources of information, used a plus sign to make sure the results will appear on a common page and used quotation marks to encapsulate search terms to make sure that all query terms appear together in resultant web sites. Therefore, searching for keywords was mostly preferred by participants over categories because categories were seen as an indirect route (included more steps) to find information, information contained in categories was too general, increased exposure to advertisements and increased time taken to find relevant information.

How participants selected from the results list was analysed. The average number of results list pages examined by participants was 5.4. The average rank of a page containing a list of results was observed as 4 therefore, each result\textquotesingle s probability of being selected from a list of results was equal. Summaries/ descriptions, URLs and titles were the mostly used to select relevant links. In addition, dates, web site\textquotesingle s size and file\textquotesingle s type were not much used when selecting appropriate links. The retrieval of numerous results was indicated as the search needed to be refined. URLs were used to evaluate the trustworthiness of the information found. For instance, a URL associated with a university was considered as reliable. Some participants viewed links on a few of the results pages and some chose the topmost link on a results page. How participants identified appropriate websites was analysed. Average number of web pages selected was 2.6 web pages. The average relatedness of a web page was 4.5 out of 5. An average score of 4.5 out of 5 was assigned to the pages which were identified as relevant to the tasks by the participants. 

Therefore, the pages analyzed by participants were highly related to the topic of search and the task completion was between 80-100\% of each task. 44\% of the relevant pages did not contain any advertisement. 40\% of those relevant pages were published by a government agency. These relevant pages belonged to one of the categories out of four categories, such as \textquotesingle informational articles\textquotesingle, \textquotesingle journal articles\textquotesingle, \textquotesingle fact sheets\textquotesingle and \textquotesingle newspaper articles\textquotesingle. The use or rejection of a web site was determined by considering three main factors:

\begin{enumerate}
	\item Whether the web site is able to satisfy users\textquotesingle information needs
	
	\item Quality of the available information, such as the author and purpose of a web site, and the method used to present information
	
	\item Higher accessibility to relevant information and the understandability of information   
\end{enumerate}

Factors, such as what participants searched for, their domain knowledge, their bias towards credibility of information and their individual understanding of topics influenced keyword searches. The system limited the improvements that could be performed while constructing queries and the apparent system improvements were not realized by the participants. Not knowing what a category contains and having an endless set of category levels were identified as barriers when using categories. The information design of web pages created a notable barrier to consumers when they search health information online because time taken to choose from a search results list was high. Erroneous decisions were made by considering the web page appearances. Credibility, reliability and trustworthiness were important when selecting pages from results lists. The major challenges for a successful search were formulating good queries and having a results list with appropriate design and standard (easy scanning and efficient recognition).  

In conclusion, consumer health information search still remains as a challenging task for the average person because they have different comprehension levels, searching abilities and levels of information needs. Both information design and search engine technology were identified as important to build good consumer health information systems. Therefore, the need for a system which provides assistance to query development, which assesses information that is being retrieved, which will have a flexible and responsive terminological infrastructure to support the search process and which will contain summaries of each result on the results pages indicating the reliability/ credibility of information, type and information content of the documents that they represent was identified. \\ 

\item {\textbf{Guido Zuccon, Bevan Koopman, and Jimmy. Choices in knowledge-base retrieval for consumer health search. In \textit{European Conference on Information Retrieval,} pages 72-85. Springer, 2018.}}  

This study investigated how to overcome consumer health search problems by expanding health queries, so as to contain more effective query terms. Specialised health knowledge bases (MeSH and UMLS) and general knowledge bases (Wikipedia) were used to 
expand consumer health queries. The \textquotesingle Entity Query Feature Expansion\textquotesingle model was used in this empirical evaluation to retrieve health information either on Wikipedia or on UMLS knowledge bases. In Wikipedia KB page title, categories, links, aliases, and body were the useful features for a retrieval scenario. In UMLS KB concept unique identifier, aliases, body, parent concepts and related concepts were the useful features for a retrieval scenario. The impact of different choices in knowledge based retrieval on query expansion and retrieval effectiveness of consumer health search was examined in five different aspects. 

\begin{enumerate}
	\item Firstly, knowledge based construction was examined. For Wikipedia three types of pages were considered as health related pages; Type 1 was pages containing a Medicine infobox; Type 2 was pages containing a Medicine infobox and hyper-links to medical terminologies provided in Mesh or UMLS; Type 3 was pages containing titles which match with an UMLS entity. For UMLS two choices were made; 1. all entities; 2. entities associated with four main aspects (symptoms, diagnostic test, diagnoses and treatments) of medical decision making
	
	\item Secondly, entity mention extraction was examined. Text from the query which can be mapped to entities were identified
	
	\item Thirdly, entity mapping was examined. Exact matches between mentions and entities were identified to map a mention to an entity
	
	\item Fourthly, source of expansion was examined. The sources from the knowledge bases were selected to use for drawing terms to perform query expansions
	
	\item Finally, the utilization of explicit relevance feedback and Pseudo Relevance Feedback was examined 
	
\end{enumerate}

A term which matched with either a title or an entity of the knowledge base considered (Wikipedia/ UMLS) were defined as health related. The preference to use a specialised knowledge base over a general knowledge base was also investigated. 

300 query topics were considered for this study. A baseline was used for the evaluation, and the title field and the body field were considered for the evaluation. Wikipedia knowledge base consisted of a set of candidate pages and they were indexed using field-based indexing. Therefore, the fields were used as a source of query expansion terms. UMLS knowledge base consisted of English terms indexed with fields, such as title, aliases, body, parent and related. The average count of terms added during query expansions and the total count of expanded queries were recorded.

\begin{enumerate}
	\item \textbf{Knowledge Base Construction:} for Wikipedia knowledge base, web pages with Medicine infobox and web pages which had links to medical terminologies exhibited the highest effectiveness in information retrieval; for UMLS knowledge base, all entities had the highest effectiveness in information retrieval; performance of baseline method was significantly better than any of the knowledge base\textquotesingle s performance; the reason is because top 10 results retrieved using knowledge base approaches included many unjudged documents for a considerable number of queries compared to the lower count of unjudged documents retrieved as top ranked results by the baseline method 
	
	\item \textbf{Entity Mentions Extraction:} for Wikipedia knowledge base, the entity mentions which matched entities (Wikipedia pages) in the CHV (Consumer Health Vocabulary) exhibited the highest effectiveness in information retrieval; for UMLS knowledge base, the entity mentions with all types of query terms exhibited higher effectiveness in information retrieval
	
	\item \textbf{Entity Mapping:} for both Wikipedia knowledge base and UMLS knowledge base, mapping entities to Aliases had the highest retrieval effectiveness
	
	\item \textbf{Source of Expansion:} for both knowledge bases the selection of titles as the source of expansion had the highest retrieval effectiveness
	
	\item \textbf{Relevance Feedback:} Wikipedia knowledge base produced a combination of positive and negative results with the incorporation of feedback; for UMLS knowledge base explicit relevance feedback had produced the best performance; however the baseline method had performed worse when compared to the knowledge base methods because only explicit relevance information was able to improve its retrieval performance 
	   
\end{enumerate} 

In conclusion, pseudo relevance feedback did not improve results, independently of the knowledge base. In contrast, relevance feedback provided improved effectiveness for query expansions and the extent of query expansions. UMLS knowledge base generally provided better improved effectiveness when compared to Wikipedia knowledge base. UMLS knowledge base expanded a higher number of queries compared to the Wikipedia knowledge base because the Wikipedia knowledge base was comparatively incomplete by only having pages containing a Medical infobox and hyper-links to related medical terminologies. The two approaches had provided dissimilar query expansions with an average of only 8.9\% of common expansion terms. The two methods retrieved comparatively different documents with an average overlap of 61\% of the top 1,000 documents. The choice of most appropriate knowledge base for query expansion resulted in the lowest number of query expansion terms and expanded a very few user queries. Relevance feedback had appended a considerable count of expansion terms and expanded a large number of queries.  In total by both the knowledge bases, 183 queries were expanded. 16 of these expansions did not exhibit any changes related to effectiveness in comparison with the baseline approach. 92 showed improvements and 75 showed losses. Overall, UMLS knowledge base was more effective when compared to Wikipedia knowledge base. UMLS knowledge base also performed better than the baseline. Therefore, it was confirmed that the use of a knowledge-base retrieval approach has the ability to translate well into the challenging consumer health search domain.                 

\textbf{Limitations}

The count of unjudged documents retrieved by issuing expanded queries was the major limitation of this study, which made it challenging to fairly evaluate the methods. In spite of that, this study which investigated the impact of two different knowledge bases on consumers\textquotesingle health-related information retrieval was able to highlight both pitfalls and payoffs. 

\item {\textbf{Elizabeth Sillence, Pam Briggs, Peter Richard Harris, and Lesley Fishwick. How do patients evaluate and make use of online health information? \textit{Social science \& medicine,} 64(9):1853-1862, 2007.}}\\ 

This study investigated how menopausal women search the Internet for information and medical advice on hormone replacement therapy (HRT) and how they determine which sites/information to trust. A stepwise model which represented users\textquotesingle gradual development of trust was used for the evaluation. This model represents the way users firstly assess sites by screening them (heuristic stage/ initial searching process), then move on to in-depth evaluation of available information (analytic stage) and finally how they develop a long-standing relationship with some specific web sites based on trust. 15 menopausal women participated in the study. All these women were interested in assessing costs and advantages associated with HRT. The authors aimed at evaluating how Internet advice influence decisions taken by these consumers and their communication with physicians. The research was divided into 3 phases:

\begin{enumerate}
		\item \textbf{Phase 1:} Online search behaviours were evaluated. 4 weekly sessions were conducted: during each session, consumers searched the Internet for information and advice regarding menopause and had a discussion with a facilitator. Two searches were freely performed on the Web and the other two were performed within specific websites. These specific websites represented a range of information providers, different content and design features to evaluate the impact of them on trust of the sources. The authors observed that participants efficiently recognized and rejected web pages which linked them to other sites, websites unrelated to menopause, websites which were not menopause specific, websites with broken links and sales sites. According to the analysis of participants interviews, design factors such as complex layout, large amount of text, look and feel of a website, and content factors such as irrelevant material and inappropriate material were the main reasons for rejecting and determining the websites as untrustworthy. Content factors (unbiased information and information appended with illustrations) were more important than design factors (simple layout and interactive features) when accepting and trusting a website. Content factors, such as source expertise and credibility, accessibility and consistency, and social identification and personalization (written in familiar language and include highly relevant content) were the most important in terms of influencing the trust of a website. The risk information contained in websites including figures were found to be confusing and misleading
		
		\item \textbf{Phase 2:} Internet usage and the utilization of information from various resources over a long time-scale was evaluated. At the end of Phase 1, participants were instructed to perform searches on health information and advice, and record their findings for the subsequent 6 months. The authors observed that only two-three sites in addition to a few new sites were visited for seeking advices during this time period; therefore, no factors which made participants revisit specific web sites were identified using this recorded information. However, facts related to the integration of information from various sources was evident from the recorded information
		  
		\item \textbf{Phase 3:} Consumers\textquotesingle opinions of the usefulness of online menopause-related health information for decision making and communication with physicians were evaluated; a telephone interview was conducted one month after completing phase 2. The authors observed that only a few  participants had repeated visits to the sites for further information and to obtain information to share with their friends, family and physicians. Participants reported that Internet was an essential source of information in the early stages of decision making; the Internet was also used for investigating specific concerns, comparing sites and assessing risks. Online information was found to affect almost all participants\textquotesingle thinking and decision making behaviours   
\end{enumerate}  

In conclusion, according to these findings it is important to note that, if the relevant content to a person\textquotesingle s information seeking is buried deeply in a website or if the website has a poor design, the information seeker will reject these resources earlier on in their search. This work also found that consumers are interested in seeking information and advice which will support their own viewpoints, because this will help them to build confidence when making decisions. They do so by accessing online information; however, they tended to integrate online and offline information and advice when making decisions. It was also found that online information is used to commence and improve discussions with friends and physicians. It was also noted that only a few consumers reported on the long-standing development of trusting relationships with specific websites.     

\textbf{Limitations}

The number of participants (15) of this study was relatively small, and only a very few participants recorded their information for the subsequent 6 months. Therefore, this information was not enough to allow for conclusions.  Asking consumers to use specific websites may have influenced their decision making. \\


\item {\textbf{Efthimis N Efthimiadis. How students search for consumer health information on the web. In \textit{System Sciences, 2009. HICSS’09. 42nd Hawaii International Conference on,} pages 1–8. IEEE, 2009.}}

This study investigated the search behaviours exhibited by students (undergraduate and graduate) while searching for consumer health information online. 32 students participated in the study. They were instructed to search and find answers for 4 health associated questions: 

\begin{enumerate}
	\item Related to flu shot
	\item Related to second-hand smoking
	\item Related to major depression
	\item Related to taking large doses of aspirin    
\end{enumerate}

Data related to students\textquotesingle demographic information, educational background, search experience on the Web, the searching process and behaviours while performing the tasks, decision making steps, including search strategies, website evaluation and question interpretations which were explained verbally, satisfaction with each task, and the reliability of the information sources were collected. The facts, such as the starting point of the search (Internet Explorer (IE) by default), query formulation and refinement, evaluation of the results, evaluation of the websites, various sources used and how the familiarity of the subject impacts on search behaviours and the satisfaction of search results were analysed and evaluated.\\

The authors analysed the results in different perspectives.

\begin{itemize}
	\item Firstly, participants\textquotesingle  characteristics were analysed. Undergraduates\textquotesingle  age ranged from 19-27 and majority of them (87.5\%) conducted a search daily. Graduates\textquotesingle  age ranged from 21-52 and 72.2\% of them conducted a search daily as well. The familiarity of web search engines for both groups was very high. Graduates were more familiar with scientific databases compared to undergraduates. Both groups tended to search information for themselves and others
	
	\item Secondly, searcher's first destination was analysed. All participants started the search task from the default Internet Explorer (IE) web browser and navigated to a web resource to commence their search (first search strategy destination choice). Four main categories (commercial sites, government sites, organizational sites and resources from University of Washington) of first search strategy destinations were identified. Majority from both groups commenced search tasks with the use of a search engine. Directories (commercial or government) was the second most popular starting point for the graduates group, but the undergraduates did not use them much
	
	\item  Thirdly, search length and website visitations were analysed. Authors identified that the search length increased with the complexity of the question. Therefore, for the questions related to flu and smoke the average time taken was quite similar (5.4 and 4.4 mins), the time was twice as long for the question on depression (8.1 mins) and had a threefold increase for the question on aspirin (13.5 mins). Overall the number of websites and web pages visited increased with the complexity of the question. Therefore, searches on flu and smoke had the minimum number of website visitations (3.9 and 3.8) followed by searches on depression (6.1 mins) and aspirin (7.8 mins). Undergraduates visited more web pages, such as 50\% and 100\% for depression and aspirin related questions respectively indicating that the higher the complexity of the question, the more difficult it is for them to answer the question. Also, the number of multiple windows opened increased with the difficulty of the question
	
	\item Fourthly, query formulation patterns and search terms were analysed. The more the complexity of questions increased, the more query formulations and search terms needed
	
	\item Finally, familiarity with the topics and searcher satisfaction were analysed. The concepts flu and smoke were quite familiar to both groups. However, graduates had more knowledge on the topics depression and aspirin. Search planning was more easier for the topics flu and smoke, but was more challenging for the topics depression and aspirin for both groups. Both groups found the first two concepts as much easier to search compared to the last two concepts which required more effort to search. Undergraduates' satisfaction of search results was extremely high and the majority of the graduates were also satisfied, but were more conservative compared to undergraduates. In terms of time taken to complete each task, satisfaction of both groups were at high levels for the first two concepts but they were more concerned about the time required for the last two concepts. All undergraduates mentioned they found answers to all the questions, but 12.5\% of graduates mentioned they did not find answers to the last two questions. 25\% of the undergraduates mentioned they were not sure about the answers to last two questions and certain percentages of graduates (5.9\% for smoke and 25\% for aspirin) mentioned they were not sure about the answers to all the questions. Undergraduates mentioned  the websites used for the first two questions as reliable and certain percentages of graduates were not sure about the reliability of the websites across all four questions     
\end{itemize} 

In conclusion, searching consumer health information online still remains as a challenging process, because many consumers have problems in query formulations. \\

\item {\textbf{Ira Puspitasari, Roberto Legaspi, and Masayuki Numao. Characterizing the effect of consumer familiarity with health topics on health information seeking behavior. In \textit{The 27th Annual Conference of the Japanese Society for Artificial Intelligence,} volume 27, pages 1–5, 2013.}}


This study investigated the impact of health topic familiarity, on consumers\textquotesingle health information seeking behaviour. 10 volunteers, including undergraduates, graduates and a postdoctoral researcher participated in the study. All participants had prior experience in searching for medical information online. They reported that when searching online for health information, they mostly searched for information related to diseases and health problems.

Participants were instructed to perform four health-related search tasks: 

\begin{itemize}
	\item \textbf{Task 1:} An exploratory task where participants had to search for the reasons for having a swelled red big toe on the right leg
	
	\item \textbf{Task 2:} A specified task where participants had to search for information about rheumatoid arthritis and osteoporosis, the affect of rheumatoid arthritis on osteoporosis, and treatments of rheumatoid arthritis which will not cause osteoporosis
	
	\item \textbf{Task 3:} A specified task where participants had to search for information about hydrochlorothiazide and the negative impacts of taking it with decongestant drug   
	
	\item \textbf{Task 4:} A personalized task where participants had to search for information about two medical treatments for any particular health concern, such as a disease or a symptom as per their interest	
	
\end{itemize}

The choice of search engine, utilization of relevant websites and the speed of search were all decided by participants themselves. Rated information regarding search performance, the effort needed to accomplish the task, the complexity and the familiarity of the task, several characteristics of search behaviour, such as different types of query terms used, query reformulation patterns and explanations about search strategies used, and impressions on the search processes were collected. 

The authors observed that according to the ratings of the participants, task 2 and task 3 were notably more challenging than task 1 or task 4, hence on average they have allocated 70\% of their time  to complete task 2 and task 3. Participants were more significantly unfamiliar with health terms on task 2 and task 3. lt reported search performance for task 2 and task 3 were comparatively much lower (somewhat disappointed) than for task 1 and task 4 (satisfied). The cognitive effort needed for completing task 2 and task 3 was much higher than the effort needed to complete task 1 and task 4. Task 3 and task 2 had the most submitted queries. The formulation of the queries and the selection of the relevant results for both these tasks were difficult because of the unfamiliarity with the health terminologies on these two tasks. Task 2, task 1 and task 3 had the longest average query length; task 1's query length was longer than task 3 because more stop-words were included in task 1's queries. 

Google search engine was used by all the participants to commence their searches. The most number of results/ web pages were accessed for task 4 followed by task 2, task 1 and task 3. The issued query keywords were classified as general consumer terms and medical specified terminology. Task 1 contained common consumer terms, and task 2 and task 3 contained medical specified terms. Query keywords issued  by the participants for task 1 contained common consumer terms whereas query keywords issued for task 2 and task 3 contained medical specified terms, because participants used keywords from the task descriptions. For task 4 participants used both common consumer terms and medical specified terms because the search was performed either based on their interest or concern. 

A medical vocabulary thesaurus, Medical Subject Headings (MeSH) was used to perform keywords classification. 24 out of 29 queries contained phrases from MeSH. Both syntactic and semantic changes were identified when performing query reformulations. Different reformulation patterns, such as generalized (generalising the meaning of previous query), specified (specifying the meaning of previous query), parallel (modifying queries from one aspect of a concept to another), building block (combing concepts from the previous queries) and dynamic (performing inconsistent query reformulation patterns) were identified. Only 3 participants performed query reformulations for task 1 and the reformulation pattern was 'specified'. For the other tasks all participants reformulated their queries. The majority performed dynamic reformulation when completing task 2 and task 3 and the majority performed parallel reformulation when completing task 4. 

In conclusion, the familiarity with the health topics affects consumers\textquotesingle searching behaviour. Participants used more specific and more varied vocabulary (query keywords selection) when performing more familiar tasks. Participants different levels of familiarity lead them to perform different types of query reformulation patterns. For unfamiliar tasks they performed dynamic reformulation patterns (task 2 and task 3) by relying only on the task descriptions for formulating queries. Selection of the results was also more difficult to perform in unfamiliar tasks (task 2 and task 3). For familiar tasks they performed the parallel reformulation pattern (task 4).

The importance of estimating consumers\textquotesingle  familiarity with the health topics is because this information can be used by a health information retrieval system to better assist users and to retrieve a larger number of understandable results.  

Therefore, a health information retrieval system can classify query keywords as common consumer terms and medical specified terms using Consumer Health Vocabulary and Medical Subject Headings, detect query reformulation patterns, including semantic changes using Unified Medical Language System (UMLS) Semantic Network and track the site selection in order to estimate consumers\textquotesingle  familiarity with the health-related search tasks. Once familiarity is estimated, the system is able to supply query suggestions and suggest recommendations to lay people and provide personalized search results. \\	
	
\item {\textbf{Anushia Inthiran, Saadat M Alhashmi, and Pervaiz K Ahmed. Describing health querying behavior. In \textit{Proceedings of the 2nd SIGIR workshop on Medical Information Retrieval (MedIR),} 2016.}} 

This study investigated the querying behaviour of laypeople when searching for health information. The querying process of a whole search process is important because it exhibits searchers\textquotesingle  primary understanding of the health aspect they search for. The improvement of this understanding can also be understood by evaluating the query reformulation patterns. Therefore, the authors aimed at evaluating query formulation and reformulation patterns used when performing health tasks with different levels of complexity. 

20 laypeople (undergraduates, postgraduates and staff members) participated in the study. Searching was performed on MedlinePlus. Two simulated situations were given to each participant; search session time was unlimited. Information, such as demographic factors, web search experience, health-related search experience were gathered via a pre-experiment interview. Clinical scenarios were used because laypeople tend to perform such health searches. Simulated situation 1 was related to searching reasons for experiencing kidney enlargement followed by urine retention and whether there are any alternative treatments or is surgery the only option available to treat this condition. Simulated situation 2 was related to searching treatment options for a swollen neck on the left side which cannot be moved left or right and the reasons for experiencing this condition. After completing each task, participants rated task complexity via a post-experiment interview. Querying patterns were classified as informational directed (related to knowing about a particular topic) and informational undirected (related to knowing everything/ anything about a topic). Query reformulation patterns were analysed using semantic analysis methods.

For simulated situation 1, 5 participants rated it as easy, another 5 rated as neutral and 10 rated as difficult. For simulated situation 2, 15 rated it as easy, another 5 rated as neutral and 5 rated as difficult. When performing easy tasks, participants started their search process by searching answers to specific questions (informational directed) and then moved towards searching more broader aspects (informational undirected). When performing neutral tasks, participants started their search process by searching more broader topics (informational undirected) and moved towards more narrower aspects (informational directed). When performing difficult tasks, participants started their search process by searching for more specific aspects (informational directed), then moved towards more broader aspects (informational undirected) and then again moved back to more specific aspects at the end of the search (no much focus). This behaviour was called as participants performing \textquotesingle unsystematic movements\textquotesingle. 

Two query reformulation patterns were demonstrated by the participants when performing easy tasks. The first one was to start with \textquotesingle switching topic\textquotesingle and end with \textquotesingle specialization\textquotesingle. The second one was to start with \textquotesingle specialization\textquotesingle and end with \textquotesingle parallel movement\textquotesingle (previous and the new queries having partial overlaps). When performing neutral tasks, again two query reformulation patterns were demonstrated by the participants. The first one was to start with \textquotesingle parallel movement\textquotesingle and end with \textquotesingle specialization\textquotesingle. The second one was the use of \textquotesingle switching topics\textquotesingle. When performing difficult tasks also, participants tended to use \textquotesingle switching topics\textquotesingle. Therefore, participants tended to use varying query reformulation strategies when performing much easier tasks and did not change the reformulation strategies (switching topic only) when performing more difficult tasks.

In conclusion, based on these results, two recommendations were proposed, so as to aid laypeople to perform well in querying and reformulating (high querying efficacy). The first recommendation is to develop an algorithm which classifies different query and reformulation patterns and the second recommendation is to suggest personalized query recommendations for users. As participants performed different querying patterns for tasks with different levels of difficulty, querying patterns are useful to detect task difficulty. Therefore, such an algorithm will be able to detect task difficulty based upon query and reformulation patterns. In addition, this information can also be used to provide relevant assisting features (based on different health domain) to users when they perform difficult tasks. Hence, if a switching topic pattern was detected, query suggestions could be provided to laypeople so as to help them in keeping the focus on the search goal. With the use of such query suggestions, laypeople will be able to issue efficient and relevant queries. Health domains can also be used to personalize assisting features according to the health search task performed. \\     

\item {\textbf{Rong Hu, Kun Lu, and Soohyung Joo. Effects of topic familiarity and search skills on query reformulation behavior. In \textit{Proceedings of the Association for Information Science and Technology,} volume 50, pages 1-9. Wiley Online Library, 2013.}}

This study investigated how query reformulation will be influenced by topic familiarity/ domain knowledge and search skills when searching for health-related information. Four research questions were considered in this study:

\begin{enumerate}
	\item The influence of topic familiarity on the choice of query reformulation type 
	
	\item The influence of users' search skills on the choice of query reformulation type 
	
	\item The influence of topic familiarity on the number of query reformulations performed within one session 
	
	\item The difference in time taken to perform each query reformulation type 
\end{enumerate}

A health information retrieval system with a search interface was used for this study. Six search topics were selected from the medical information database MEDLINE to be used in the user studies. 45 graduate students participated in the study. Users were instructed to perform two pre-experiment search tasks. In the first task participants were instructed to perform searches to find the definitions of medical terms by picking one from each search topic. In the second task participants were instructed to seek for relationships between medical concepts and to find answers to each search topic. The use of search system was not allowed while answering questions. Participants\textquotesingle demographic information, familiarity with each search topic and information about their major was collected. Each participant performed three tasks using health information search system. Data, such as user queries, timestamps of searches, IDs of the documents viewed, and the topic of search task were collected. 135 search sessions were collected. 

Two groups of sessions were created by categorizing sessions based on participants\textquotesingle  topic familiarity. Therefore, the two categories were expert (higher topic familiarity) sessions (34 out of 135 sessions) and novice (lower topic familiarity) sessions (101 out of 135 sessions). 45 participants were divided into two groups based on their major. Therefore, the two groups were Library and Information Science (LIS) major (28 out of 45 participants) and other majors (17 out of 45 participants). Query reformulations were categorized into six types. 4 main facets were considered for the analysis. 

\begin{enumerate}
	\item \textbf{Content changed:} included 4 sub-facets; (i) specification: adding or using more specific terms to specify a query; (ii) generalization: replacing specific terms with general terms to generalize a query; (iii) parallel movement: shifting query terms to another aspect of the topic
	
	\item \textbf{Content unchanged:} included 1 sub-facet; (i) synonym: replacing terms with more common terms which have the same meaning of the previous terms
	
	\item \textbf{Format:} changes done to the format (variations of terms and search operators)
	
	\item \textbf{Error:} correcting typos and wrong formats in previous terms 
\end{enumerate}


According to authors\textquotesingle observations, participants had an average topic familiarity of 2 (based on a nine-level Likert scale) prior performing searches. The majority (more than 65\%) of the participants occasionally searched health information online and about 11\% of them never searched health information online. 112 out of 135 sessions had at least one query reformulation action. A total of 334 query reformulation actions were performed. The reformulation types \textquotesingle Specification\textquotesingle, \textquotesingle Generalization\textquotesingle and \textquotesingle Parallel movement\textquotesingle were performed in 78\% of the query reformulations. \textquotesingle Format\textquotesingle, \textquotesingle Error\textquotesingle and \textquotesingle Synonym\textquotesingle were performed in 16.8\%, 3.9\% and 0.3\% of the query reformulations respectively. Specification query reformulation type was identified as most frequently being used and synonym was the least frequently used query reformulation type. 

During expert sessions the most frequently applied query reformulation types were \textquotesingle Generalization\textquotesingle (22.4\%) and \textquotesingle Format\textquotesingle (22.4\%). Users who were more familiar with the topics started their the search process with more specific terms and extended search results with the use of \textquotesingle Generalization\textquotesingle reformulation type. \textquotesingle Specification\textquotesingle (35.3\%), \textquotesingle Parallel movement\textquotesingle (25.6\%) and \textquotesingle Error\textquotesingle (4.1\%) were the most frequently applied query reformulation types during novice sessions.  

This revealed that participants who were less familiar with the topics (low domain knowledge) are prone to make typos and spelling mistakes because of the insufficient knowledge about health topics. They also started their search process with general terms and gradually had moved towards specific terms in order to retrieve accurate and more useful information. However, according to the statistical analysis there was no significant association between topic familiarity (domain knowledge) and the choice of query reformulation types. \textquotesingle LIS major\textquotesingle group more frequently applied \textquotesingle Format\textquotesingle(17.7\%), \textquotesingle Generalization\textquotesingle (21.7\%) and \textquotesingle Specification\textquotesingle (35.9\%) during their search process. 

LIS major users tended to change the format of the queries very frequently so as to retrieve more precise search results and made comparatively less errors. \textquotesingle Parallel movement\textquotesingle (27.4\%) and \textquotesingle Error\textquotesingle (7.4\%) were the most frequently applied query reformulation types by the other majors group during their search process. Hence, this group was unable to issue efficient queries to obtain relevant results and was not good at moving from one aspect to another while searching. However, according to the statistical analysis, no significant association was found between search skills and the choice of query reformulation type.    
 
The average query reformulations per session was lower for the \textquotesingle expert\textquotesingle group (under 2 query reformulations) compared to the \textquotesingle novice\textquotesingle group (approximately 2.7 query reformulations). Therefore, participants with higher topic familiarity/ domain knowledge were able to complete search tasks with lesser effort in query reformulation. 

According to the statistical analysis a significant association between topic familiarity and the number of query reformulations per session was identified. In terms of time taken to perform each query reformulation type, participants spent comparatively a longer time to perform \textquotesingle Error\textquotesingle (correct queries) query reformulation type. However, according to the statistical analysis no significant difference was found among different times taken to perform each query reformulation type. 

In conclusion, topic familiarity has the ability to impact on the number of query reformulations performed per session and the time required to perform each query reformulation type. Users with higher topic familiarity made a lower number of spelling mistakes and preferred applying general terms to modify their queries. Therefore, they tended to issue more correct search statements, and initiated their search processes with more specific terms and gradually moved on to issuing more general terms to obtain broader search results. In addition, these users performed less number of query reformulations to complete their search tasks (more efficient). 

In contrast, users with lower topic familiarity were less likely to apply \textquotesingle Format\textquotesingle and \textquotesingle Generalization\textquotesingle query reformulation types when modifying queries. They initiated their search processes with more broader terms and gradually moved on to issuing specific terms to obtain accurate and useful search results. Because these users obtained irrelevant search results with their inadequate initial queries, they applied a \textquotesingle Parallel reformulation\textquotesingle to move from one search aspect to another aspect. Therefore, such users had more difficulty in selecting proper search strategies when initiating a search process. As a result, they performed more number of query reformulations to complete their search tasks (less efficient). 

Users with higher search skills (LIS major) made less errors and applied \textquotesingle Format\textquotesingle, \textquotesingle Generalization\textquotesingle or \textquotesingle Specification\textquotesingle query reformulation types. \textquotesingle Error\textquotesingle and \textquotesingle Parallel movement\textquotesingle query reformulation types were not much used by these users. In contrast, users from other major group did not very frequently apply any of the \textquotesingle Format\textquotesingle, \textquotesingle Generalization\textquotesingle or \textquotesingle Specification\textquotesingle query reformulation types and made more errors when issuing queries to the system. Therefore, users with higher search skills applied various query reformulation types. Overall, users spent more time when conducting \textquotesingle Error\textquotesingle query reformulation type because they had to correct terms from previous queries. Therefore, it is said that query expansion techniques, query suggestion techniques, controlled vocabularies and auto filling techniques can be used in IR systems to aid users with lower topic familiarity.                   

\textbf{Limitations}

Authors reported that 135 search sessions from 45 participants were insufficient to generalize the findings of this study. Level of topic familiarity was measured on the basis of user perception, so the information might not be reliable. Information about participants\textquotesingle search skills were not obtained in person for this study (although LIS major participants were assumed to have advanced search skills compared to other participants).  

\end{enumerate}

\addcontentsline{toc}{section}{Appendix B: Project Log Sheets}
\section*{Appendix B: Project Log Sheets} 

	
\end{document}

